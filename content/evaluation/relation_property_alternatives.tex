\subsection{Alternative Approaches to Relation Properties}
\label{sec:alt_approaches_to_relation_properties}
Relation properties are determined using a soft label approach. This section details alternative approaches and their advantages and disadvantages.

Ideally, we would extract the properties of relations directly from metadata about the source graphs, as this would accurately depict the properties of the relations. However, most \glspl{kg} do not contain this metadata, and none of the \glspl{kg} used in this paper has complete metadata information, making this approach unfeasible.

Another alternative approach is manual assignment of properties based on notions about what each relation models. This approach risks misinterpreting the purpose of the relations, resulting in property assignment that do not reflect the data, and involves additional work to include each dataset.

We chose to do soft label assignment, as it depicts the relevant data, does not require a complete \gls{kg} unlike hard label assignment, and makes it easy to add new datasets.

Based on the labels assigned to each relation, we considered using data materialization, to create more complete \glspl{kg}. This would include modifying the dataset with new facts that can be inferred from the relation properties e.g. adding $(e_1, r, e_2, \tau)$ if $(e_2, r, e_1, \tau)$ exists and $r$ has the symmetric property.
We chose not to do this, as we still could not guarantee a complete \gls{kg} after data materialization, and it would make it impossible to compare our results to other works that was learned on the original dataset.

The soft label thresholds were selected empirically. We identified some relations that we expect to have certain properties, and the thresholds were selected such that those relations were assigned the expected relation properties.