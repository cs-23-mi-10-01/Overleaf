\section{Conclusion}
\label{sec:conclusion}

We have analyzed the \gls{tkg} embedding methods DE-TransE, De-DistMult, DE-SimplE, ATiSE, TeRo and TimePlex to evaluate their performance depending on the charactersitics of queries.
%hypoteser
The inspected characteristics are the temporal density of data, the relation properties symmetry, antisymmetry, and inversion, and prediction target which has been included and expanded upon from a previous set of findings.

%hypoteser
We find that the overall score of predictions is not particularly impacted by the temporal density of the data, however time predictions specifically score higher in temporally dense partitions.
Our findings also support that DE-TransE, which theoretically cannot model symmetry, is indeed worse at symmetric relations than other models, while DE-DistMult, which theoretically cannot model antisymmetry and inversion, is not worse at antisymmetric or inverse relations than other models.

%ensemble
The findings are used in a new ensemble model ORB-E where the results are used to calculate query-specific weights for each model. ORB-E achieves better performance than each individual model, as well as an ensemble model that assignes the same static weight to each model.
This suggests that it is possible to achieve a better performance by taking a model's advantages and disadvantages into account.

%time accuracy
Finally, we use the continuous nature of timestamps to examine the accuracy of time predictions and improve them.
We find that the difference between the predicted and correct timestamp is generally too large to be useful in most \gls{qa} systems.
However, if all models are equally accurate in their top timestamp answer, the average of all predictions is more accurate than each indvidual model.