\section{Methods}
\label{sec:methods}

The selection of methods have been expanded from \cite{P9} to include DE-TransE, DE-DistMult, DE-Simple \cite{goel19diachronicemb}, ATiSE \cite{xu19atise}, TeRo \cite{xu2020tero}, and TimePlex. In this section, a general outline of the method approaches are given. In \autoref{tab:overview_of_models}, an overview of what each method is designed for is provided.

\begin{table}[htb]
\centering
\begin{minipage}{\columnwidthcaption}
\centering
\caption{Overview of method and characteristics. T: Transformation, TD: Tensor decomposition.}
\label{tab:overview_of_models}
\end{minipage}

\vspace{-3mm}
%DE-TransE, DE-DistMult, DE-SimplE: \cite{goel19diachronicemb} ATiSE: \cite{xu19atise} TeRo: \cite{xu2020tero} TimePlex: \cite{jain2020timeplex}
\resizebox{\columnwidth}{!}{
\begin{tabular}{r|cccc}
\hline
Method & Category & {Symmetry} & {Anti-Symmetry} & {Inversion} \\ \hline
DE-TransE & T & $\times$ & \checkmark & \checkmark \\
DE-DistMult & TD & \checkmark & $\times$ & $\times$ \\
DE-SimplE & TD & \checkmark & \checkmark & \checkmark \\
ATiSE & TD & \checkmark & \checkmark & \checkmark \\
TeRo & T & \checkmark & \checkmark & \checkmark \\
TimePlex & TD & \checkmark & \checkmark & \checkmark \\ \hline
\end{tabular}
}
% \begin{tblr}{colspec={r|cccc},rowspec={Q[b]QQQQQ}}
% \hline
% Model & Category & {Symmetry} & {Anti-\\Symmetry} & {Inversion} \\ \hline
% DE-TransE & T & $\times$ & \checkmark & \checkmark \\
% DE-DistMult & TD & \checkmark & $\times$ & $\times$ \\
% DE-SimplE & TD & \checkmark & \checkmark & \checkmark \\
% ATiSE & TD & \checkmark & \checkmark & \checkmark \\
% TeRo & T & \checkmark & \checkmark & \checkmark \\
% TimePlex & TD & \checkmark & \checkmark & \checkmark \\ \hline
% \end{tblr}

\end{table}

DE-TransE is based on the TransE \cite{bordes2013transe} method, a method used as a benchmark for many embedding approaches. TransE uses vectors in a Euclidean vector space to make entity and relation embeddings, and Euclidean transformations and distance calculations to evaluate the score facts. The TransE method has been modified with diachronic embeddings, resulting in the entity embeddings being modified with temporal information, thereby embedding temporal information.

DE-DistMult is based on the DistMult \cite{yang2015distmult} method, a method that uses vectors to make embeddings, and element wise products of the embedding vectors to score facts. Similarly to DE-TransE, DE-DistMult has been modified to add temporal information in the entity embeddings.

DE-SimplE is based on SimplE \cite{kazemi2018simple} which in turn is based on Canonical Polyadic \cite{hitchcock1927cp} embedding. Canonical Poliadic embedding embed relations using single vectors, and entities using two vectors each, one for that entity in the head position and one for tail positions in facts. SimplE then embeds relations using two vectors, one for the forward relation $r$ and one for the reverse relation $r^{-1}$. The score function is then defined as being the average of the element wise product of the embeddings of the elements involved with two facts, namely $(h, r, t)$ and $(t, r^{-1}, h)$. This is then modified with diachronic embeddings, including temporal information in entity embeddings.

ATiSE is a method that uses additive time series to model changes in the data over time. The additive time series is a combination of a linear function, a sine function and a random function, to describe trends, seasonal changes and noise respectively. Each entity and relation has a seperate additive time series to represent how those elements change over time, following the patterns that those elements demonstrate. Entities and relations are embedded as vectors, and they follow Gaussian probability distributions. The score function considers the difference between the probability distribution of entities in head and tail positions of a fact, and their similarity to the probability distribution of the relation of the fact at a certain time. To compare the probability distributions, ATiSE uses Kullback–Leibler divergence.

TeRo is a method that is similar to TransE, but uses rotations in complex space to model entities at specific timestamps instead of Euclidian space. Entities and relations are embedded using vectors in complex space, and entities are defined with time information by an element-wise rotation of the embeddings in complex space. The scoring function is defined as the distance between the time-specific head entity plus the relation, and the conjugate of the time-specific tail entity.

TimePlex is a model that has a base version and a full version. Both versions include embeddings of entities and timestamps in complex vector space, and the relations are embedded with three vectors in complex space each. The first relation embedding vector represents a relation that is true for a head entity at a specific timestamp, the second relation embedding represents a relation that is true for a head entity and a tail entity, the third relation represents a relation that is true for a tail entity and a specific timestamp. The score function of the base version uses element-wise products between the head, relations, tail and timestamps, the latter embeddings in the products being the conjugate of those vectors. The full version then adds two additional scores to the score function of the base version, one additional score considers recurring events and the other considers time constraints between pairs of relations.
