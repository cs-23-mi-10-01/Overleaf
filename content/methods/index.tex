\section{Methods}
\label{sec:methods}

The selection of methods has been expanded from \cite{P9} and consists of DE-TransE, DE-DistMult, DE-Simple, ATiSE, TeRo, and TimePlex. 
In \autoref{tab:overview_of_models}, an overview of what each method is designed for is provided. Methods have been selected such that they are diverse in the relation properties that they are theoretically capable of modelling and overall approach.
In this section, a general outline of the method approaches are given. 

\begin{table}[htb]
\centering
\begin{minipage}{\columnwidthcaption}
\centering
\caption{Overview of method and characteristics. T: Transformation, TD: Tensor decomposition.}
\label{tab:overview_of_models}
\end{minipage}

\vspace{-3mm}
%DE-TransE, DE-DistMult, DE-SimplE: \cite{goel19diachronicemb} ATiSE: \cite{xu19atise} TeRo: \cite{xu2020tero} TimePlex: \cite{jain2020timeplex}
\resizebox{\columnwidth}{!}{
\begin{tabular}{r|cccc}
\hline
Method & Category & {Symmetry} & {Antisymmetry} & {Inversion} \\ \hline
DE-TransE & T & $\times$ & \checkmark & \checkmark \\
DE-DistMult & TD & \checkmark & $\times$ & $\times$ \\
DE-SimplE & TD & \checkmark & \checkmark & \checkmark \\
ATiSE & TD & \checkmark & \checkmark & \checkmark \\
TeRo & T & \checkmark & \checkmark & \checkmark \\
TimePlex & TD & \checkmark & \checkmark & \checkmark \\ \hline
\end{tabular}
}
% \begin{tblr}{colspec={r|cccc},rowspec={Q[b]QQQQQ}}
% \hline
% Model & Category & {Symmetry} & {Anti-\\Symmetry} & {Inversion} \\ \hline
% DE-TransE & T & $\times$ & \checkmark & \checkmark \\
% DE-DistMult & TD & \checkmark & $\times$ & $\times$ \\
% DE-SimplE & TD & \checkmark & \checkmark & \checkmark \\
% ATiSE & TD & \checkmark & \checkmark & \checkmark \\
% TeRo & T & \checkmark & \checkmark & \checkmark \\
% TimePlex & TD & \checkmark & \checkmark & \checkmark \\ \hline
% \end{tblr}

\end{table}
DE is a model-agnostic method, meaning it adapts existing non-temporal methods to the use of temporal information. Here we use it on the methods TransE, DistMult and SimplE.

DE-TransE is based on the TransE \cite{bordes2013transe} method, a method used as a benchmark for many embedding approaches. 
TransE embeds entities and relations as vectors in Euclidian space, 
models relations as a translation between head and tail entity and scores the facts by Euclidian distance calculation.
TransE is unable to model symmetry as that would require embedding both entities with the same vector and the symmetric relation to be a 0 distance vector. This would make it impossible to differentiate between the entities in all other relations.

DE-DistMult is based on the DistMult \cite{yang2015distmult} method, a method that like TransE uses a single vector to embed each relation and entity. Unlike TransE it scores facts as summations of element wise products of the embedding vectors.
As the ordering of elements in the element-wise product does not affect the result, DistMult cannot model edge direction, which in turn means that it cannot model anti-symmetry or inversion.

DE-SimplE is based on SimplE \cite{kazemi2018simple} which in turn is based on Canonical Polyadic \cite{hitchcock1927cp} embedding. 
SimplE embeds relations and entities using two vectors each. Relation embeddings have a vector $r$ for forward relations and a vector $r^{-1}$ for reverse relations. Entity embeddings have a vector $e_h$ for head entities, and a vector $e_t$ for tail entities.
The score function is then defined as the average of the element wise product of the embeddings of the elements in two facts, namely $(e_h, r, e_t)$ and $(e_t, r^{-1}, e_h)$.

Methods modified with DE \cite{goel19diachronicemb} are extended such that the base method includes temporal information by making the entity embeddings dependent on time.
Each model has a part of their entity embedding vector values that are impacted by timestamp.
DE adds two learnable embedding vectors for each entity, one for dependency on time, and one for a bias.
The resulting entity embedding values are found by multiplying the time-dependent vector values with the timestamp, adding them to a bias value, using an activation function on them, and then multiplying them with the non time-dependent embedding. 
The part of embedding vector values that are not dependent on the timestamp are simply the unaltered original embedding vector values.

ATiSE \cite{xu19atise} is a method that uses additive time series decomposition to model changes and uncertainty in the data over time. The additive time series is a combination of a linear function, a sine function and a random function, to describe trends, seasonal changes and noise respectively. Each entity and relation has a seperate time series to represent how those elements change over time, following the patterns that those elements demonstrate. Entities and relations are embedded as vectors, and they follow Gaussian probability distributions. The score function considers the difference between the probability distribution of entities in head and tail positions of a fact, and their similarity to the probability distribution of the relation of the fact at a certain time. To compare the probability distributions, ATiSE uses Kullback–Leibler divergence.

TeRo \cite{xu2020tero} embeds entities and relations as vectors in complex space and uses element-wise rotations in complex space to model entities at specific timestamps.
The scoring function is defined as the distance between the time-specific head entity plus the relation, and the conjugate of the time-specific tail entity.

TimePlex \cite{jain2020timeplex} is a model that embeds entities and timestamps in complex vector space, and each relation with three vectors in complex space. The first relation embedding vector represents a relation that is true for a head entity at a specific timestamp, the second relation embedding represents a relation that is true for a head entity and a tail entity, the third relation represents a relation that is true for a tail entity and a specific timestamp. The score function uses element-wise products between the head, relations, tail and timestamps, the latter embeddings in the products being the conjugate of those vectors. In addition, the score function also includes two additional scores, one of them consider recurrences of events, the other considers time contraints between pairs of relations.