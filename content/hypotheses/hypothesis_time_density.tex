\subsection{Hypothesis on Time Density}
\label{sec:hypothesis_time_density}

As a dataset has varying amounts of data in different time periods, we considered whether the prediction quality is affected by the temporal density of the data. This led to the hypothesis:

\begin{hypothesis}
\label{hyp:time_density}
Prediction quality is significantly higher on queries where the timestamps are in a temporally dense partition of the training dataset, compared to a sparse partition.
\end{hypothesis}

The significance of results refer to differences in \gls{mrr} scores that are above a certain threshold, as defined in \autoref{sec:background-and-notation}.

%\begin{figure}[htb]
\centering
\begin{minipage}{\columnwidthcaption}
\centering

\begin{tikzpicture}
\begin{axis}[
    ymin=0,
    ymax=165.6,
    ylabel={No. of facts},
    xlabel={Time},
    %height=5cm,
    %width=5cm,
    area style,
    scaled y ticks=false
    ]

\addplot[
    fill=tikzgrey,
] coordinates {
(2014.0, 0)
(2014.004213483146, 30.75)
(2014.0154494382023, 56.25)
(2014.0252808988764, 51.0)
(2014.0351123595506, 70.5)
(2014.0463483146068, 53.25)
(2014.056179775281, 83.0)
(2014.066011235955, 60.75)
(2014.0758426966293, 78.0)
(2014.0856741573034, 54.0)
(2014.0969101123594, 92.25)
(2014.1067415730338, 45.0)
(2014.116573033708, 82.5)
(2014.1264044943819, 58.0)
(2014.1362359550562, 102.75)
(2014.1474719101125, 60.0)
(2014.1573033707864, 82.0)
(2014.1671348314608, 56.25)
(2014.176966292135, 76.0)
(2014.186797752809, 53.25)
(2014.1980337078653, 83.25)
(2014.2078651685395, 67.0)
(2014.2176966292136, 81.75)
(2014.2275280898878, 52.0)
(2014.2373595505617, 87.0)
(2014.248595505618, 57.75)
(2014.2584269662923, 73.0)
(2014.2682584269662, 69.0)
(2014.2780898876406, 74.0)
(2014.2879213483147, 61.5)
(2014.2991573033707, 64.5)
(2014.3089887640451, 57.0)
(2014.318820224719, 72.0)
(2014.3286516853932, 74.0)
(2014.3384831460676, 66.0)
(2014.3497191011236, 83.25)
(2014.3595505617977, 72.0)
(2014.369382022472, 79.5)
(2014.3792134831463, 87.0)
(2014.3890449438202, 77.25)
(2014.4002808988766, 71.25)
(2014.4101123595508, 80.0)
(2014.4199438202247, 67.5)
(2014.4297752808989, 100.0)
(2014.439606741573, 52.5)
(2014.4508426966293, 82.5)
(2014.4606741573034, 48.0)
(2014.4705056179776, 108.75)
(2014.4803370786515, 62.0)
(2014.4901685393259, 95.25)
(2014.501404494382, 68.25)
(2014.511235955056, 96.0)
(2014.5210674157304, 66.0)
(2014.5323033707866, 81.75)
(2014.5421348314605, 68.0)
(2014.5519662921347, 86.25)
(2014.561797752809, 49.0)
(2014.5716292134832, 78.75)
(2014.5828651685392, 58.5)
(2014.5926966292134, 62.0)
(2014.6025280898878, 66.0)
(2014.6123595505617, 59.0)
(2014.6221910112358, 93.0)
(2014.633426966292, 69.75)
(2014.6432584269662, 65.0)
(2014.6530898876404, 63.0)
(2014.6629213483145, 71.0)
(2014.672752808989, 71.25)
(2014.683988764045, 87.75)
(2014.693820224719, 77.0)
(2014.7036516853932, 63.75)
(2014.7134831460673, 78.0)
(2014.7233146067415, 94.5)
(2014.7345505617977, 71.25)
(2014.7443820224719, 93.0)
(2014.754213483146, 57.75)
(2014.7640449438202, 100.0)
(2014.7738764044943, 70.5)
(2014.7851123595506, 92.25)
(2014.7949438202247, 46.0)
(2014.8047752808989, 95.25)
(2014.814606741573, 80.0)
(2014.8244382022472, 102.0)
(2014.8356741573034, 67.5)
(2014.8455056179773, 83.0)
(2014.8553370786517, 71.25)
(2014.8651685393259, 76.0)
(2014.875, 84.0)
(2014.8862359550562, 117.0)
(2014.8960674157304, 88.0)
(2014.9058988764045, 111.75)
(2014.9157303370787, 62.0)
(2014.9255617977528, 92.25)
(2014.936797752809, 61.5)
(2014.9466292134832, 84.0)
(2014.9564606741571, 86.25)
(2014.9662921348315, 93.0)
(2014.9761235955057, 63.75)
(2014.9873595505617, 60.0)
(2014.997191011236, 73.0)
(2015.0070224719102, 43.5)
(2015.0140449438202, 45)
(2015.0168539325844, 138)
(2015.0196629213483, 96)
(2015.0224719101125, 27)
(2015.0224719101125, 0)
} ;

\addplot[ %Dense
    color=tikzdarkred,
    pattern=north east lines,
    pattern color=tikzlightred,
    opacity=0.8,
] 
coordinates {
(2013.9991517410167, -10)
(2013.9991517410167, -10)
(2014.0092752252754, -10)
(2014.1314662365112, -10)
(2014.131174212927, 1000)
(2014.1412976971856, 1000)
(2014.1424101679831, -10)
(2014.425005562354, -10)
(2014.4247135387695, 1000)
(2014.4348370230282, 1000)
(2014.4345449994437, -10)
(2014.4657358994327, -10)
(2014.4654438758482, 1000)
(2014.4755673601069, 1000)
(2014.4752753365221, -10)
(2014.7592752252754, -10)
(2014.7589832016909, 1000)
(2014.7691066859495, 1000)
(2014.768814662365, -10)
(2014.8196684837023, -10)
(2014.8193764601178, 1000)
(2014.8294999443765, 1000)
(2014.830612415174, -10)
(2014.8800617421293, -10)
(2014.881174212927, 1000)
(2014.8912976971856, 1000)
(2014.891005673601, -10)
(2014.9011291578597, -10)
(2014.9008371342752, 1000)
(2014.9109606185339, 1000)
(2014.9106685949494, -10)
(2015.0191066859495, -10)
(2015.011792190455, 1000)
(2015.0219156747137, 1000)
(2015.014601179219, -10)
(2015.0275336522418, -10)
(2015.0275336522418, -10)
} ;

\addplot[ %Sparse
    color=tikzdarkblue,
    pattern=north west lines,
    pattern color=tikzlightblue,
    opacity=0.8,
] 
coordinates {
(2013.9991517410167, -10)
(2013.9991517410167, 1000)
(2014.0092752252754, 1000)
(2014.0303426410057, 1000)
(2014.0300506174212, -10)
(2014.04017410168, -10)
(2014.0412865724775, 1000)
(2014.0514100567361, 1000)
(2014.0511180331516, -10)
(2014.0612415174103, -10)
(2014.0609494938258, 1000)
(2014.0710729780844, 1000)
(2014.0707809545, -10)
(2014.0809044387586, -10)
(2014.080612415174, 1000)
(2014.0907358994327, 1000)
(2014.09184837023, -10)
(2014.1019718544887, -10)
(2014.1016798309045, 1000)
(2014.1118033151631, 1000)
(2014.1115112915786, -10)
(2014.1216347758373, -10)
(2014.1213427522525, 1000)
(2014.1314662365112, 1000)
(2014.131174212927, -10)
(2014.1412976971856, -10)
(2014.1424101679831, 1000)
(2014.1525336522418, 1000)
(2014.152241628657, -10)
(2014.1623651129157, -10)
(2014.1620730893314, 1000)
(2014.17219657359, 1000)
(2014.1719045500056, -10)
(2014.1820280342642, -10)
(2014.1817360106797, 1000)
(2014.1918594949384, 1000)
(2014.192971965736, -10)
(2014.2030954499946, -10)
(2014.2028034264101, 1000)
(2014.2129269106688, 1000)
(2014.2126348870843, -10)
(2014.222758371343, -10)
(2014.2224663477584, 1000)
(2014.232589832017, 1000)
(2014.2322978084323, -10)
(2014.242421292691, -10)
(2014.2435337634886, 1000)
(2014.2536572477472, 1000)
(2014.253365224163, -10)
(2014.28315162977, -10)
(2014.2828596061854, 1000)
(2014.3140505061745, 1000)
(2014.3137584825897, -10)
(2014.3337134275225, -10)
(2014.3334214039382, 1000)
(2014.343544888197, 1000)
(2014.3446573589943, -10)
(2014.4151741016801, -10)
(2014.4148820780954, 1000)
(2014.425005562354, 1000)
(2014.4247135387695, -10)
(2014.4348370230282, -10)
(2014.4345449994437, 1000)
(2014.4446684837023, 1000)
(2014.4457809545, -10)
(2014.4559044387586, -10)
(2014.455612415174, 1000)
(2014.4657358994327, 1000)
(2014.4654438758482, -10)
(2014.4755673601069, -10)
(2014.4752753365221, 1000)
(2014.4853988207808, 1000)
(2014.4851067971965, -10)
(2014.4952302814552, -10)
(2014.4963427522528, 1000)
(2014.5064662365114, 1000)
(2014.5061742129267, -10)
(2014.5162976971853, -10)
(2014.516005673601, 1000)
(2014.5261291578597, 1000)
(2014.5272416286573, -10)
(2014.537365112916, -10)
(2014.5370730893312, 1000)
(2014.5471965735899, 1000)
(2014.5469045500054, -10)
(2014.557028034264, -10)
(2014.5567360106797, 1000)
(2014.5668594949384, 1000)
(2014.566567471354, -10)
(2014.5766909556125, -10)
(2014.57780342641, 1000)
(2014.617421292691, 1000)
(2014.6171292691065, -10)
(2014.6384887084214, -10)
(2014.6381966848369, 1000)
(2014.6581516297697, 1000)
(2014.6578596061852, -10)
(2014.6988819668484, -10)
(2014.6985899432639, 1000)
(2014.7087134275225, 1000)
(2014.708421403938, -10)
(2014.7494437646012, -10)
(2014.7491517410167, 1000)
(2014.7592752252754, 1000)
(2014.7589832016909, -10)
(2014.79017410168, -10)
(2014.7898820780954, 1000)
(2014.800005562354, 1000)
(2014.7997135387695, -10)
(2014.8294999443765, -10)
(2014.830612415174, 1000)
(2014.8407358994327, 1000)
(2014.840443875848, -10)
(2014.9109606185339, -10)
(2014.9106685949494, 1000)
(2014.920792079208, 1000)
(2014.9205000556235, -10)
(2014.9306235398822, -10)
(2014.9317360106797, 1000)
(2014.9418594949384, 1000)
(2014.941567471354, -10)
(2014.9713538769608, -10)
(2014.9710618533763, 1000)
(2014.992421292691, 1000)
(2014.9921292691067, -10)
(2015.0022527533654, -10)
(2015.0019607297809, 1000)
(2015.0191066859495, 1000)
(2015.011792190455, -10)
(2015.0247246634776, -10)
(2015.0174101679831, 1000)
(2015.0275336522418, 1000)
(2015.0275336522418, -10)
} ;


\end{axis}
\end{tikzpicture}

\vspace{-3mm}
\caption{Dense and sparse partitions on ICEWS14. Dense partition marked in red, sparse in blue.}
\label{fig:time_density_icews14}
\end{minipage}
\end{figure}
%\begin{figure}[htb]
\centering
\begin{minipage}{\columnwidthcaption}
\centering

\begin{tikzpicture}
\begin{axis}[
    ymin=0,
    ymax=758.4,
    ylabel={No. of facts},
    xlabel={Time},
    %height=5cm,
    %width=5cm,
    area style,
    scaled y ticks=false
    ]

\addplot[
    fill=tikzgrey,
] coordinates {
(1705.0, 0)
(1706.5, 2.0)
(1710.0, 1.3333333333333333)
(1713.0, 0.0)
(1716.0, 0.0)
(1719.0, 0.0)
(1722.0, 5.333333333333333)
(1725.0, 0.0)
(1728.0, 0.0)
(1731.0, 0.0)
(1734.0, 0.0)
(1737.0, 0.0)
(1740.5, 0.0)
(1744.0, 0.0)
(1747.0, 0.0)
(1750.0, 0.0)
(1753.0, 0.0)
(1756.0, 0.0)
(1759.0, 0.0)
(1762.0, 0.0)
(1765.0, 0.0)
(1768.0, 0.0)
(1771.0, 0.0)
(1774.5, 0.0)
(1778.0, 0.0)
(1781.0, 0.0)
(1784.0, 0.0)
(1787.0, 0.0)
(1790.0, 25.333333333333332)
(1793.0, 0.0)
(1796.0, 4.0)
(1799.0, 1.3333333333333333)
(1802.0, 1.3333333333333333)
(1805.0, 4.0)
(1808.5, 3.0)
(1812.0, 4.0)
(1815.0, 4.0)
(1818.0, 2.6666666666666665)
(1821.0, 1.3333333333333333)
(1824.0, 1.3333333333333333)
(1827.0, 0.0)
(1830.0, 4.0)
(1833.0, 1.3333333333333333)
(1836.0, 4.0)
(1839.0, 0.0)
(1842.5, 0.0)
(1846.0, 0.0)
(1849.0, 1.3333333333333333)
(1852.0, 0.0)
(1855.0, 2.6666666666666665)
(1858.0, 1.3333333333333333)
(1861.0, 2.6666666666666665)
(1864.0, 1.3333333333333333)
(1867.0, 12.0)
(1870.0, 36.0)
(1873.0, 0.0)
(1876.0, 0.0)
(1879.5, 3.0)
(1883.0, 2.6666666666666665)
(1886.0, 6.666666666666667)
(1889.0, 2.6666666666666665)
(1892.0, 5.333333333333333)
(1895.0, 0.0)
(1898.0, 4.0)
(1901.0, 9.333333333333334)
(1904.0, 12.0)
(1907.0, 13.333333333333334)
(1910.0, 6.666666666666667)
(1913.5, 16.0)
(1917.0, 32.0)
(1920.0, 74.66666666666667)
(1923.0, 18.666666666666668)
(1926.0, 20.0)
(1929.0, 28.0)
(1932.0, 16.0)
(1935.0, 29.333333333333332)
(1938.0, 21.333333333333332)
(1941.0, 22.666666666666668)
(1944.0, 44.0)
(1947.5, 38.0)
(1951.0, 34.666666666666664)
(1954.0, 29.333333333333332)
(1957.0, 32.0)
(1960.0, 46.666666666666664)
(1963.0, 57.333333333333336)
(1966.0, 68.0)
(1969.0, 50.666666666666664)
(1972.0, 58.666666666666664)
(1975.0, 77.33333333333333)
(1978.0, 96.0)
(1981.5, 125.0)
(1985.0, 125.33333333333333)
(1988.0, 156.0)
(1991.0, 173.33333333333334)
(1994.0, 248.0)
(1997.0, 305.3333333333333)
(2000.0, 476.0)
(2003.0, 560.0)
(2006.0, 570.6666666666666)
(2009.0, 632.0)
(2012.0, 504.0)
(2014.0, 324)
(2015.0, 312)
(2016.0, 136)
(2017.0, 68)
(2017.0, 0)
} ;

\addplot[ %Dense
    color=tikzdarkred,
    pattern=north east lines,
    pattern color=tikzlightred,
    opacity=0.8,
] 
coordinates {
(1704.9554455445545, -10)
(1704.9554455445545, -10)
(1708.0445544554455, -10)
(2001.5445544554455, -10)
(2001.4554455445545, 1000)
(2010.5445544554455, 1000)
(2010.4554455445545, -10)
(2018.5445544554455, -10)
(2018.5445544554455, -10)
} ;

\addplot[ %Sparse
    color=tikzdarkblue,
    pattern=north west lines,
    pattern color=tikzlightblue,
    opacity=0.8,
] 
coordinates {
(1704.9554455445545, -10)
(1704.9554455445545, 1000)
(1708.0445544554455, 1000)
(1986.5445544554455, 1000)
(1986.4554455445545, -10)
(2017.5445544554455, -10)
(2015.4554455445545, 1000)
(2018.5445544554455, 1000)
(2018.5445544554455, -10)
} ;


\end{axis}
\end{tikzpicture}

\vspace{-3mm}
\caption{Dense and sparse partitions on WikiData12k. Dense partition marked in red, sparse in blue.}
\label{fig:time_density_wikidata12k}
\end{minipage}
\end{figure}
\begin{figure}[htb]
\centering
\begin{minipage}{\columnwidthcaption}
\centering

\begin{tikzpicture}
\begin{axis}[
    ymin=0,
    ymax=251.2,
    ylabel={No. of facts},
    xlabel={Time},
    %height=5cm,
    %width=5cm,
    area style,
    scaled y ticks=false
    ]

\addplot[
    fill=ourgrey,
] coordinates {
(1731.0, 0)
(1732.0, 1.3333333333333333)
(1735.0, 0.0)
(1738.0, 0.0)
(1741.0, 1.3333333333333333)
(1744.0, 0.0)
(1746.5, 0.0)
(1749.0, 0.0)
(1752.0, 0.0)
(1755.0, 1.3333333333333333)
(1758.0, 0.0)
(1761.0, 0.0)
(1763.5, 0.0)
(1766.0, 0.0)
(1769.0, 0.0)
(1772.0, 0.0)
(1775.0, 0.0)
(1778.0, 0.0)
(1780.5, 0.0)
(1783.0, 0.0)
(1786.0, 0.0)
(1789.0, 0.0)
(1792.0, 1.3333333333333333)
(1795.0, 2.6666666666666665)
(1797.5, 0.0)
(1800.0, 1.3333333333333333)
(1803.0, 0.0)
(1806.0, 0.0)
(1809.0, 0.0)
(1812.0, 1.3333333333333333)
(1814.5, 2.0)
(1817.0, 1.3333333333333333)
(1820.0, 1.3333333333333333)
(1823.0, 2.6666666666666665)
(1826.0, 1.3333333333333333)
(1829.0, 8.0)
(1831.5, 2.0)
(1834.0, 4.0)
(1837.0, 4.0)
(1840.0, 2.6666666666666665)
(1843.0, 4.0)
(1846.0, 0.0)
(1848.5, 6.0)
(1851.0, 4.0)
(1854.0, 9.333333333333334)
(1857.0, 1.3333333333333333)
(1860.0, 5.333333333333333)
(1863.0, 2.6666666666666665)
(1865.5, 8.0)
(1868.0, 6.666666666666667)
(1871.0, 2.6666666666666665)
(1874.0, 8.0)
(1877.0, 5.333333333333333)
(1880.0, 2.6666666666666665)
(1882.5, 6.0)
(1885.0, 6.666666666666667)
(1888.0, 6.666666666666667)
(1891.0, 10.666666666666666)
(1894.0, 5.333333333333333)
(1897.0, 6.666666666666667)
(1899.5, 12.0)
(1902.0, 21.333333333333332)
(1905.0, 17.333333333333332)
(1908.0, 17.333333333333332)
(1911.0, 14.666666666666666)
(1914.0, 13.333333333333334)
(1916.5, 48.0)
(1919.0, 34.666666666666664)
(1922.0, 24.0)
(1925.0, 36.0)
(1928.0, 42.666666666666664)
(1931.0, 34.666666666666664)
(1933.5, 26.0)
(1936.0, 28.0)
(1939.0, 36.0)
(1942.0, 36.0)
(1945.0, 41.333333333333336)
(1948.0, 34.666666666666664)
(1950.5, 44.0)
(1953.0, 36.0)
(1956.0, 41.333333333333336)
(1959.0, 48.0)
(1962.0, 61.333333333333336)
(1965.0, 45.333333333333336)
(1967.5, 46.0)
(1970.0, 46.666666666666664)
(1973.0, 60.0)
(1976.0, 57.333333333333336)
(1979.0, 60.0)
(1982.0, 65.33333333333333)
(1984.5, 74.0)
(1987.0, 74.66666666666667)
(1990.0, 84.0)
(1993.0, 84.0)
(1996.0, 112.0)
(1999.0, 142.66666666666666)
(2001.5, 170.0)
(2004.0, 204.0)
(2007.0, 208.0)
(2010.0, 209.33333333333334)
(2013.0, 166.66666666666666)
(2015.0, 120)
(2016.0, 80)
(2017.0, 20)
(2017.0, 0)
} ;

\addplot[ %Dense
    color=ourdarkred,
    pattern=north east lines,
    pattern color=ourlightred,
    opacity=0.8,
] 
coordinates {
(1730.5841584158416, -10)
(1730.5841584158416, -10)
(1733.4158415841584, -10)
(2002.9158415841584, -10)
(2002.9158415841584, 1000)
(2011.5841584158416, 1000)
(2011.5841584158416, -10)
(2018.4158415841584, -10)
(2018.4158415841584, -10)
} ;

\addplot[ %Sparse
    color=ourdarkblue,
    pattern=north west lines,
    pattern color=ourlightblue,
    opacity=0.8,
] 
coordinates {
(1730.5841584158416, -10)
(1730.5841584158416, 1000)
(1733.4158415841584, 1000)
(1915.4158415841584, 1000)
(1915.0841584158416, -10)
(1917.9158415841584, -10)
(1917.5841584158416, 1000)
(1957.4158415841584, 1000)
(1957.5841584158416, -10)
(1963.4158415841584, -10)
(1963.5841584158416, 1000)
(1971.4158415841584, 1000)
(1971.5841584158416, -10)
(2015.5841584158416, -10)
(2015.5841584158416, 1000)
(2018.4158415841584, 1000)
(2018.4158415841584, -10)
} ;


\end{axis}
\end{tikzpicture}

\vspace{-3mm}
\caption{Dense and sparse partitions on YAGO11k. Dense partition marked in red, sparse in blue. Facts from before year 1700 is not included, as they are few in number.}
\label{fig:time_density_yago11k}
\end{minipage}
\end{figure}

\autoref{hyp:time_density} will be examined by creating two non-continuous partitions that contain test facts, a sparse partition where the temporal density is low and a dense partition where the temporal density is high. The two partitions should have approximately the same number of facts in order to make fair comparisons between them.
These partitions are illustrated for YAGO in \autoref{fig:time_density_yago11k}. Intuitively, we expect the \gls{mrr} score to be higher in the dense part of the dataset, as the amount of available data at each point in time is larger.

Additionally, we hypothesize that temporal density greatly impacts time predictions:

\begin{subhypothesis}
\label{hyp:time_density_timestamp}
Prediction quality of time predictions is significantly higher on dense data partitions, than on sparse data partitions.
\end{subhypothesis}

This subhypothesis is created from an intuition that the possibility of making a time prediction that is closer to the correct answer is higher when the data is spread over a shorter period of time.

Construction of the datasets requires the function $\mathit{range}$ and $\mathit{num}$. 
The function $\mathit{range}$ maps a timespan to a set of timestamps $\varT^2 \rightarrow \varP(\varT)$, where $\varP(\varT)$ is the power set of $\varT$. 
It is defined as

\begin{equation}
\mathit{range}(\tau) = \{ i \in \varT \mid \timebegin{\tau} \leq i \wedge i < \timeend{\tau} \}
\end{equation}

\noindent
where $\tau$ is a timespan, $\timebegin{\tau}$ is the beginning timestamp and $\timeend{\tau}$ is the ending timestamp.
The result of $\mathit{range}(\tau)$ is a set of timestamps that contain all timestamps between the beginning timestamp and the end timestamp of $\tau$. The beginning timestamp and all intermediate timestamps are included, but not the end timestamp.

The $\mathit{range}$ function is then used to define the function $\mathit{num}$, which maps a timespan to a natural number $\varT^2 \rightarrow \N$. This function describes how many facts in $\varG$ that are contained in the timespan. The function is defined as

\begin{equation}
\begin{gathered}
\mathit{num}(\tau) = | N |\\
N = \{ f \in \varG \mid \mathit{range}(\tau) \cap \mathit{range}(f_\tau) \neq \emptyset \}
\end{gathered}
\end{equation}

\noindent
where $N$ is the set of all facts where the timestamp overlaps with $\tau$.

For every fact $f \in \varG$ we find the number of facts $\mathit{num}(f_\tau)$ in the timespan $f_\tau$ of $f$, and arrange the values in a sorted sequence of numbers $O$.
The sparse partition $T_P$ of test set $T$ is defined as the subset of $T$ where the number of facts that falls within the timespan of $f_\tau$ is lower than $p$ for all $f \in T$:

\begin{equation}
T_P = \{ t \in T \mid \textit{num}(t_\tau) < p \}
\end{equation}

\noindent
where $p$ is the \nth{25} percentile value of $O$. The dense partition $T_D$ is similarly defined as a subset of $T$ where the number of facts that falls within the timespan of $f_\tau$ is higher than $d$ for all $f \in T$:

\begin{equation}
T_D = \{ t \in T \mid \textit{num}(t_\tau) > d \}
\end{equation}

\noindent
where $d$ is the \nth{75} percentile value of $O$. Note that by these definitions the partitions are not continuous.

The \gls{mrr} score of each method is calculated over the $T_D$ and $T_P$ test sets, on several datasets. For both hypotheses, if the score is significantly higher in $T_D$ than $T_P$, then the hypothesis is deemed true.

\begin{comment}
foreach fact in dataset
    fact_count_i_timespan(fact) >> array
done
sort(array)
median(array)
\end{comment}