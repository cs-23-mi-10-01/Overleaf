\newcommand{\ismarried}[0]{
    node [
            near start, 
            above=5pt,
            color=black!80,
            align=center,
            execute at begin node=\setlength{\baselineskip}{4pt},
        ] 
        {is married\\\footnotesize{2018-2023}}
}

\begin{figure}[htb]
\centering

\begin{tikzpicture}
    \node [draw, circle, outer sep=2pt] (John) at (0,0){John};
    \node [draw, circle, outer sep=2pt] (Jane) at (3,1){Jane};
    \node [draw, circle, densely dotted, scale=3, outer sep=2/3pt] (unknown) at (6,2){};
    \draw [ultra thick, ourdarkgreen, ->] (John) -- (Jane) \ismarried ;
    \draw [ultra thick, ourdarkyellow, ->] (Jane) -- (unknown) \ismarried ;
\end{tikzpicture}
\centering

\begin{minipage}{\columnwidthcaption}
\caption{Illustration of DE-TransE embedding of the relation \textit{is\_married}. John is married to Jane, which is modelled with a translation from John to Jane (green), but when the same translation is applied to Jane the result is unknown (yellow).}
\label{fig:detranse_is_married}
\end{minipage}

\end{figure}