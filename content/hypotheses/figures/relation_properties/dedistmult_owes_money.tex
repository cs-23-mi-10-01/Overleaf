\newcommand{\cellsize}{0.9cm}
\newcommand{\owesmoney}{\textit{owes}\\\textit{money}}
%\newcommand{\hatch}{[pattern=north east lines, pattern color=ourdarkblue]}

\begin{figure}[htb]
\centering
\begin{tikzpicture}
    \tikzstyle{hatch}=[
        pattern=north west lines,
        pattern color=ourgrey,
        ]
    \tikzstyle{every node}=[
        inner sep=0pt,
        %draw=black,
        x=\cellsize, 
        y=\cellsize,
        minimum size=\cellsize,
        align=center,
        execute at begin node=\setlength{\baselineskip}{6pt},
        font=\small,
        ]
    \pgfmatrix{rectangle}{center}{mymatrix}
    {\pgfusepath{}}{\pgfpointorigin}{\let\&=\pgfmatrixnextcell}
    {
        \node(0_0) {\owesmoney};    \& \node(1_0){Jack}; \& \node(2_0) {Peter}; \& \node(3_0) {Kate};\\
        \node(0_1) {Jack};               \& \node(1_1){}; \& \node(2_1) {\scaleto{\textbf{$\times$}}{10pt}}; \& \node(3_1){};\\
        \node(0_2) {Peter};              \& \node(1_2)[hatch] {}; \& \node(2_2) {}; \& \node(3_2){};\\
        \node(0_3){Kate};                \& \node(1_3)[hatch] {}; \& \node(2_3)[hatch]{}; \& \node(3_3){};\\
    }
    \draw (0_0.south west) -- (3_0.south east) (0_1.south west) -- (3_1.south east) (0_2.south west) -- (0_2.south east) (2_2.south west) -- (3_2.south east);
    \draw (0_0.north east) -- (0_3.south east) (1_0.north east) -- (1_2.south east) (2_0.north east) -- (2_3.south east) ;
\end{tikzpicture}
\centering

\begin{minipage}{\columnwidthcaption}
\caption{Illustration of DE-DistMult embedding of the relation \textit{owes money}. The relation exists between Jack and Peter, but it is unknown who owes who money. DistMult cannot model the hatched cells.}
\label{fig:dedistmult_owes_money}
\end{minipage}

\end{figure}

%\def\mystrut(#1,#2){\vrule height #1 depth #2 width 0pt}
% \newcolumntype{C}{%
%    >{\mystrut(3ex,3ex)\centering}%
%    p{3ex}%
%    <{}}  

% \begin{tabular}{r|C|c|c}
% %|*{6}{p{1.5cm}|}}
% \textit{Arrest} & Jack & Peter & Louise\\
% \hline
% Jack    & False             & True              & False \\
% \hline
% Peter   & \cellcolor{gray}  & False             & False \\
% \hline
% Louise  & \cellcolor{gray}  & \cellcolor{gray}  & False \\
% \end{tabular}