\section{Related Work}
\label{sec:related-work}

This paper is a continuation of \cite{P9}, where different embedding methods are analyzed through an exploration of the quality of results when the models make link predictions. The main observation made from that analysis is that the most influential contributing factor that influence the quality of the results is the relations and the structure of the \gls{kg} that surrounds them. This is the reasoning for what this paper concerns, and this paper serves to further examine the structure of the \gls{kg} surrounding the relations, increase reliability of the findings from the previous paper, and find other contributing factors that can influence the quality of the results.

As this paper is a direct continuation of \cite{P9}, the related work of that paper is also applicable here.
It covers the different categories of \gls{tkg} embedding methods, namely transformation, tensor decomposition, and neural network. 
Transformational methods use geometric functions to score facts in the embedding.
Tensor decomposition methods use tensor and eigenvector products to decompose tensors into low-dimentional representations that is used to score the facts.
Neural network methods use a number of learned layers to score facts based on the numerical representation of the elements of the facts.
We have been unable identify an influential, recent, and temporal neural network method, which we could replicate the results of and therefore neural network methods are not included in this study.
Influential transformational methods include RotatE \cite{sun2019rotate}, TeRo \cite{xu2020tero}, and ChronoR \cite{sadeighan2021chronor}. Influential tensor decomposition methods include TNTComplEx \cite{lacroix2020tcomplex}, TimePlex \cite{jain2020timeplex}, and ATiSE \cite{xu19atise}. 
Influential neural network methods include TFLEX \cite{lin22tflex} and Re-Net \cite{jin2019renet}.
A temporal model agnostic method takes an existing non-temporal embedding method and modifies it, such that temporal information is added to the embeddings.
Influential model agnostic methods include diachronic embeddings \cite{goel19diachronicemb} and time aware representations \cite{garcia-duran2018ta}.

Temporal accuracy metrics has previously been defined for scoring functions using a measure that compares absolute distances between elements of intervals \cite{surdeanu2013overviewtac}, and using bounding boxes to evaluate the intersections and hulls between intervals \cite{jain2020timeplex}. These metrics are mostly made for scoring functions and not statistical evaluation of time predictions, as they represent the scores as a measure that cannot be evaluated directly by humans, and it is difficult to interpret if the results are usable in a \gls{qa} context from these metrics alone.

Relation properties such as symmetry, antisymmetry, and inversion have been defined in non-temporal contexts \cite{schmidt10relationalmathematics}.
These properties define the structure of the \gls{kg} surrounding the relation, and some methods are incompatible with some of these properties \cite{chami2020atth, gregucci23sepa}.

The ensemble method ORB-E is based on a straightforward voting ensemble method \cite{MOHAMMED2023757}.
An ensemble method with similar construction to ORB-E is presented in \cite{otte2022towards}. It also has the domain of temporal \gls{kge}, but it does not consider query characteristics or dynamic weight distribution. 
%tilføj ensemble related work. måske christian og christian?
