\section{Datasets}
\label{sec:datasets}
When selecting datasets for model evaluation we require the data to be at least partly temporal as that is the focus of the project. Prevalence in related work is also valued highly, as it enables us to compare our implementation with the original implementation when both are evaluated over the same data.

The selected graphs are ICEWS \cite{boschee2015ICEWS}, WikiData \cite{vrandecic2014wikidata}, and YAGO \cite{mahdisoltani2015YAGO3, tahon2020YAGO4}. 
ICEWS is a set of fully temporal, event-based \glspl{kg} specific to the domain of crisis alerts, where each graph is constrained within one year.
ICEWS is particularly noted for its consistency and even temporal distribution of data. 
WikiData and YAGO are partly temporal, general knowledge \glspl{kg} and are not constrained to any certain time span.
WikiData is particularly noted for its size and YAGO for its focus on temporal data and ability to represent uncertainty in timestamps.

\begin{table}[htb]
\centering
\begin{minipage}{\columnwidthcaption}
\centering
\caption{Statistics of datasets}
\label{tab:dataset_stats}
\end{minipage}

\vspace{-3mm}
\begin{tabular}{r|M{0.22}M{0.22}M{0.21}} \hline
Dataset & \mbox{ICEWS14} & WikiData12k & YAGO11k \\
\hline
\# Facts & 96730 & 40621 & 20507\\
\# Entities & 7128 & 12554 & 10623\\
\# Relations & 230 & 24 & 10\\
Time Period & 2014 & 19--2020 & -431--2844\\
\hline
\end{tabular}
\end{table}

Specifically the datasets ICEWS14-7K\footnote{ICEWS14-7K is typically referred to as simply ICEWS14. However, so is ICEWS14-12K \cite{trivedi2017knowevolve}. We append the entity count to the name in order to distinguish between the two datasets.} \cite{garcia-duran2018ta}, WikiData12K \cite{dasgupta2018hyte}, and YAGO11K \cite{dasgupta2018hyte} are used. Statistics of the datasets can be found in \autoref{tab:dataset_stats}. Date and month information has been removed from WikiData12K and YAGO11K for the sake of uniformity.