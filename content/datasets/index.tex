\section{Datasets}
\label{sec:datasets}
When selecting datasets for model evaluation we require the data to be at least partly temporal. Prevalence in related work is also valued highly, as it enables us to compare our implementation with the original implementation when both are evaluated over the same data.

\begin{table}[htb]
\centering
\begin{minipage}{\columnwidthcaption}
\centering
\caption{Statistics of datasets}
\label{tab:dataset_stats}
\end{minipage}

\vspace{-3mm}
\begin{tabular}{r|M{0.22}M{0.22}M{0.21}} \hline
Dataset & \mbox{ICEWS14} & WikiData12k & YAGO11k \\
\hline
\# Facts & 96730 & 40621 & 20507\\
\# Entities & 7128 & 12554 & 10623\\
\# Relations & 230 & 24 & 10\\
Timespan & 2014 & 19--2020 &-431--2844\\
\hline
\end{tabular}
\end{table}

The selected graphs are ICEWS \cite{boschee2015ICEWS}, WikiData \cite{vrandecic2014wikidata}, and YAGO \cite{mahdisoltani2015YAGO3, tahon2020YAGO4}. 
ICEWS is a number of fully temporal, event-based \glspl{kg} specific to the domain of crisis alerts, where each graph contains facts from a single year.
The granularity of time steps in ICEWS is one day, and the dataset contains one timestamp per fact.
These timestamps are treated as timespans with the same beginning and end time.
ICEWS is particularly noted for its consistency and uniform temporal distribution of data. 
WikiData and YAGO are partly temporal, general knowledge \glspl{kg} and are not constrained to any certain time span.
WikiData has a time step granularity of one year and YAGO has a granularity of one day.
Both represent time as a timespan with a beginning and an end.
WikiData is particularly noted for its size and YAGO for its focus on temporal data and ability to represent uncertainty in timestamps.

%\footnote{\mbox{ICEWS14-7k} is typically referred to as simply ICEWS14. However, so is \mbox{ICEWS14-12K} \cite{trivedi2017knowevolve}. We append the entity count to the name in order to distinguish between the two datasets.} 
Specifically we use the datasets \mbox{ICEWS14}\footnote{ICEWS14 exists in at least two versions: One with 7128 entities \cite{garcia-duran2018ta} and one with 12498 \cite{trivedi2017knowevolve}. We use the first version.}
, WikiData12K \cite{dasgupta2018hyte}, and YAGO11K \cite{dasgupta2018hyte}, henceforth referred to simply as ICEWS, WikiData and YAGO. Statistics of the datasets can be found in \autoref{tab:dataset_stats}. Date and month information has been removed from YAGO for the sake of uniformity, making the granularity of time steps one year for both WikiData and YAGO.