\section*{Resume}
%intro til kgs? 
Denne artikel udforsker hvorvidt spørgsmåls forskellige karakterisikaer har en påvirkelse på præcisionen af link prediction på temporal knowlegde grafs. De temporal knowlegde grafs der bliver undersøgt i denne artikel er ICEWS, WikiData og YAGO, og metoder brugt er DE-TransE, DE-SimplE, DE-Dismult, AtisE, TERO og TimePlex. Dette bliver gjordt ved først at opsette tre overordnede hypoteser, som undersøger disse karakterisikaer med et antal underhypoteser. Hypoteserne bliver herefter enten bevist eller modbevist baseret på resultater. Disse resultater bliver derefter brugt til at lave en ensemble voting metode, som bliver sammenlignet med en naiv ensemble voting metode som ikke bliver er baseret på resultaterne. 

Den første hypotese undersøger hvorvidt densitet af spørgsmål i tidsenheder har en påvirkning på de forskellige modellers ydeevne. Både samlet ydeevne og ydevne for tids forudsigelse bliver undersøgt her, og vi går ud fra at der er bedre ydevne når der er mange spørgsmål per tidsenhed, altså at der er høj densitet. 
Evalueringen af denne hyppotese begynder med at opdele vores dataset i tre dele baseret på deres densitet af spørgsmål i tidsenheder. De top 25\% er høj densitets sættet, De laveste 25\% er lav densitets sættet og de miderste 50\% bliver ikke brugt. Dette sikre os i at få stor forskel i densitet imellem høj og lav densitets sættet. Modellerne er herefter kørt på begge disse datasæt og forskellen i resultater mellem dem er udregnet. Det viste sig at på ICEWS og WikiData er der bedre ydevne på de lav densitets datasæt, men på YAGO er det en stor forbedring på det høj densitets sæt. Dette betyder at hyppotesen ikke er sand, men temporal densitet betyder mere på datasæt med stort tids interval. Den samme undersøgelse blev også gjordt for tids forudsigelse specifikt. Her kan man se en stor forbedring i ydeevne i høj densitets sættet i forhold til lav densitetes sættet. Dette leder til at under hyppotesen omkring hvorvidt tids forudsigelser er bedre på dataset med høj densitet på tid til at være sand. 

Den næste hypotese undersøger hvorvidt gennemsnits forudsigelsen for tid er bedre end hver model's egen forudsigelse. Dette kan gøres da tid er en kontinuerlig værdi og et gennemsnit kan faktisk findes, i modsætning til enheder eller relationer. Måden gennemsnits tiden bliver udregner er the mean average af tiderne modellerne forudsiger. På ICEWS har nogenlunde samme mean average error fra deres forudsagt tid til den rigtige tid. Her fik den gennemsnitlige tid et bedre resultat end alle andre metoder. På de to andre datasæt tre af metoderne har rigtig dårlige forudsigelser, så den gennemsnitlige forudsigelse bliver forværret pga. dem. Dette betyder at denne hyppotese er sand hvis alle modeller har circa samme ydeevne. 

Den sidste hypotese undersøger relationer egenskaber og hvordan det påvirker ydeeven af modeller. Egenskaberne undersøgt er symetriske relationer, som for eksempel "søskende", antisymetriske relationer, som for eksempel "mor til" og inverse relationer, som for eksempel paret "modtage gave" og "give gave". 
Den første underhyppotese DE-TRansE 

%ensemble learning

%ablation studies

%For at forbedre præcisionen af linkforudsigelse på Temporale Vidensgrafer (TKGs) er der blevet udviklet mange Vidensgrafindlejring (KGE) metoder, der gradvist forbedrer den overordnede kvalitet af forudsigelserne. Tidsstempelforudsigelser er især blevet undersøgt i højere detaljer i nyere forskning. I denne artikel undersøges styrkerne og svaghederne ved flere temporale KGE-metoder i forhold til de temporale egenskaber ved relationstyper og den temporale datatæthed, som bestemmes af strukturen af Vidensgrafen (KG). Dette arbejde bygger på tidligere arbejde, der analyserede kvaliteten af forudsigelser baseret på forudsigelsesmålet. Denne analyse fører til en ensemblelæringsmetode, hvor vægtene beregnes ud fra egenskaberne ved forespørgslen og præstationen af de involverede modeller. De resulterende regler, der bestemmer vægtene for hver model, illustrerer derefter, hvilke forespørgselsegenskaber der har størst betydning for forskellige metoder og generelt set. Derudover analyseres den relativt mindre udforskede domæne for tidsforudsigelser for deres samlede nøjagtighed og hvor velegnede de nuværende metoder er til brug i spørgsmål-svar (QA) systemer, og der præsenteres en strategi for tidsforudsigelser, der udnytter tidsoplysningernes kontinuerlige natur.