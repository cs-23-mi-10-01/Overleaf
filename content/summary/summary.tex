\thispagestyle{empty}
\begin{center}\Huge\bfseries\sffamily
Resumé
\end{center}
\begin{multicols*}{2}
%intro til kgs? 
Denne artikel omhandler \textbf{embedding metoder for temporal knowledge grafer og undersøger hvorvidt forskellige karakteristika af et query har en påvirkning på præcisionen} af link prediction. De undersøgte grafer er ICEWS, WikiData og YAGO3 og de undersøgte metoder er DE-TransE, DE-SimplE, DE-Dismult, ATiSE, TeRo og TimePlex. Dette bliver gjordt ved først at opstille tre overordnede hypoteser som omhandler de udvalgte karakterstika. Hypoteserne bliver herefter enten be- eller afkræftet baseret på vores resultater. Disse resultater bliver derefter brugt til at lave en ensemble voting model der favoriserer specifikke modeller afhængigt af de karakteristika der findes i det pågældende query. Til sammenligning laves også en naiv ensemble voting model som ikke favoriserer nogen modeller. 

%hyp1
Den første hypotese undersøger hvorvidt \textbf{koncentrationen af de kendte faktum igennem tidsperioder} har en påvirkning på ydeevnen af de forskellige modeller. Både samlet ydeevne og ydeevne for tidsforudsigelse bliver undersøgt her, og vi forventer at der er bedre ydeevne når der er en høj koncentration af kendte faktum i den tidsperiode som et givent query omhandler.
Evalueringen af denne hypotese begynder med at opdele vores dataset i tre partitioner, en partition for tidsperioder med højst koncentration af kendte faktum, en med lav koncentration af kendte faktum, og en med mellem koncentration, som ikke bliver brugt. Dette sikrer os en stor forskel i koncentrationen af data i mellem det høje og det lave koncentrationssæt. Metoderne evalueres over disse to datasæt. Det viste sig at på ICEWS og WikiData er der bedre ydevne på datasættet med den lave koncentration, men på YAGO er der en stor forbedring på datasættet med den høje koncentration. Siden vores forventede resultat kun kunne observeres i én af de tre dataset betyder det at at hypotesen er afkræftet. Den samme undersøgelse blev også gjort for tidsforudsigelser specifikt. Her kan man se en stor forbedring i ydeevne på datasættet med høj koncentration i forhold til datasættet med lav koncentration, hvilket betyder at underhypotesen om tidsforudsigelser er bekræftet.

%hyp2
Den anden hypotese undersøger hvorvidt en samlet \textbf{gennemsnitsforudsigelse for tidsforudsigelser mellem flere modeller} generelt giver bedre resultater end tidsforudsigelser lavet af hver model individuelt. Dette kan lade sig gøre da tid er en kontinuerlig værdi og det derfor er muligt at finde et gennemsnit, i modsætning til andre elementer af grafen. Tidsforudsigelser af hver individuel model på ICEWS har nogenlunde samme middelafvigelse fra den forudsagte tid til den rigtige tid. Her fik den gennemsnitslige tid et bedre resultat end hver af de andre metoder. På de to andre datasæt har tre af modellerne betydelig værre middelafvigelse i forudsigelserne, så den gennemsnitlige forudsigelse får værre resultater af at inkludere dem i gennemsnittet. Dette betyder at denne hypotese er sand i nogle situationer, men at forudsigelserne er stærkt afhængeige af middelafvigelsen i forudsigelserne af de individuelle modeller. 

%hyp3
Den tredje hypotese undersøger relationer og den omkringliggende struktur i grafen for at \textbf{kategorisere relationerne som symmetriske, antisymmetriske og inverse, og hvordan det påvirker ydeeven} af modeller som teoretisk ikke burde være i stand til at modellere de relationer, sammenlignet med modeller som kan.
Til dette formål laves der seks nye datasæt, to for symmetriske og ikke-symmetriske, to for antisymmetriske og ikke-antisymmetriske, og to for inverse og ikke-inverse. Modellerne bliver evalueret på disse datasæt, og sammenlignet for deres ydeevne.
Hvad vi fandt frem til var at DE-TransE, som ikke burde kunne håndtere symmetri, har betydeligt værre ydeevne på symmetriske relationer end de sammenlignede modeller. Udover det fandt vi at DE-DistMult, som hverken burdte kunne modellere antisymmetriske eller inverse relationer, kan modellere antisymmetriske og inverse relationer omtrent lige så godt som DE-SimplE.
Generelt kan vi se at der er en forbindelse mellem kategorien af relationer, egenskaberne af modellerne og deres resultater med datasæt af forskellige relationskategorier. Dog har vi også resultater der viser at modeller kan få gode resultater på trods af deres teoretiske begrænsninger. 

%ensemble learning
Resultaterne fra hypoteserne bliver brugt til at lave ORB-E som er vores \textbf{regelbaserede ensemblemodel}. I ensemblemodellen har metoder en vægt baseret på karakteristika af et query og deres ydeevne for det karakteriska. Vægten bestemmer hvor stor en påvirkning den metode har for ensemblemodellens evaluering af det pågældende query. Til at sammenligne med ORB-E laver vi også en naiv ensemblemodel, som giver hver model den samme vægt, så alle modeller har samme indflydelse for alle queries. Resultaterne viser at ORB-E er bedre end alle de andre metoder på ICEWS og YAGO, og på WikiData deler ORB-E førstepladsen med TeRo. Den naive model har en ydeevne som er på cirka samme niveau som de andre individuelle modeller. Dette viser at de forskellige karakteristika af et query kan bruges til at forbedre ensemble metoder ved dynamisk tildeling af vægte. 

%ablation studies
Til sidst lavede vi også nogle \textbf{ablationsstudier} for at nærmere undersøge effekten af de forskellige karakteristika på ORB-E. De 12 forskellige studier er delt op i tre dele. Syv studier hvor et enkelt karakteristika er fjernet, fire hvor alle karakteristika er fjernet undtagen et og et enkelt hvor vægten af et enkelt karakteristika er blevet justeret. Forskellen i disse ablationsstudier er lille, men generelt kan vi se at forudsigelsesmålet som karakteristika har den bedste positive indflydelse på ORB-E. Udover det kan man se at det er på YAGO der er størst forskel på resultaterne af ablationsstudierne, hvilket antyder at her er de forskellige query-karakteristika vigtigere.


%For at forbedre præcisionen af linkforudsigelse på Temporale Vidensgrafer (TKGs) er der blevet udviklet mange Vidensgrafindlejring (KGE) metoder, der gradvist forbedrer den overordnede kvalitet af forudsigelserne. Tidsstempelforudsigelser er især blevet undersøgt i højere detaljer i nyere forskning. I denne artikel undersøges styrkerne og svaghederne ved flere temporale KGE-metoder i forhold til de temporale egenskaber ved relationstyper og den temporale datatæthed, som bestemmes af strukturen af Vidensgrafen (KG). Dette arbejde bygger på tidligere arbejde, der analyserede kvaliteten af forudsigelser baseret på forudsigelsesmålet. Denne analyse fører til en ensemblelæringsmetode, hvor vægtene beregnes ud fra egenskaberne ved forespørgslen og præstationen af de involverede modeller. De resulterende regler, der bestemmer vægtene for hver model, illustrerer derefter, hvilke forespørgselsegenskaber der har størst betydning for forskellige metoder og generelt set. Derudover analyseres den relativt mindre udforskede domæne for tidsforudsigelser for deres samlede nøjagtighed og hvor velegnede de nuværende metoder er til brug i spørgsmål-svar (QA) systemer, og der præsenteres en strategi for tidsforudsigelser, der udnytter tidsoplysningernes kontinuerlige natur.
\end{multicols*}