\subsection{Relation Properties}
\label{sec:relation_properties_experiment}
%\begin{table*}[htb]
\centering
\begin{minipage}{\fullwidthcaption}
\centering
\caption{\gls{mrr} scores on symmetric ($T_S$) and non-symmetric ($T_S'$) partitions of testsets.}
\label{tab:time_density_comparison}
\end{minipage}

\vspace{-3mm}

\begin{tabular}{r|SSSS|SSSS}\hline
 & \multicolumn{4}{c|}{\mbox{ICEWS}} & \multicolumn{4}{c}{YAGO} \\
Method & {$T_S'$} & {$T_S$} & {Diff} & {Cmp. DE-T} & {$T_S'$} & {$T_S$} & {Diff} & {Cmp. DE-T} \\ \hline
\rowcolor{gray!15}
DE-T &
0.24  & 0.26  & \better{0.02} & +0.00 &
0.06  & 0.32  & \better{0.26} & +0.00 \\
DE-D &
0.33  & 0.48  & \better{0.15} & \sigbetter{0.13} &
0.03  & 0.63  & \better{0.60} & \sigbetter{0.34} \\
DE-S & 
0.34  & 0.48  & \better{0.14} & \sigbetter{0.12} &
0.04  & 0.62  & \better{0.59} & \sigbetter{0.33} \\
\hline
\end{tabular}
\end{table*}


\begin{table*}[htb]
\centering
\begin{minipage}{\fullwidthcaption}
\centering
\caption{\gls{mrr} scores on partitions of test sets with ($T_S$, $T_A$, $T_I$) and without ($T_S'$, $T_A'$, $T_I'$) certain properties. Significant results marked with $\blacktriangle$ or $\blacktriangledown$.}
\vspace{-3mm}

\begin{tabular}{cr|SSS|SSS|SSS}
\hline
&
& \multicolumn{3}{c|}{Symmetry} 
& \multicolumn{3}{c|}{Antisymmetry}
& \multicolumn{3}{c}{Inversion} 
\\
& Method 
& {$T_S'$} & {$T_S$} & {Diff.}
& {$T_A'$} & {$T_A$} & {Diff.} 
& {$T_I'$} & {$T_I$} & {Diff.}
\\
\hline
\multirow{3}{*}{\rotatebox[origin=c]{90}{ICEWS}} &
DE-TransE &
0.24  & 0.26  & \better{0.02}   &
0.23  & 0.32  & \sigbetter{0.09}   &
0.24  & 0.34  & \sigbetter{0.11} 
\\
&
DE-DistMult &
0.33    & 0.48  & \sigbetter{0.15}    &
0.35    & 0.38  & \better{0.03}    &
0.35   & 0.44  & \sigbetter{0.10}
\\
&
DE-SimplE & 
0.34    & 0.48  & \sigbetter{0.14}    &
0.36    & 0.39  & \better{0.03}    &
0.36    & 0.49  & \sigbetter{0.13} 
\\
\hline
\multirow{3}{*}{\rotatebox[origin=c]{90}{WD}} &
DE-TransE &
{--}    & {--}  & {--}    &
0.11  & 0.14  & \better{0.03}   &
{--}    & {--}  & {--} 
\\
&
DE-DistMult &
{--}    & {--}  & {--}    &
0.12    & 0.14  & \better{0.02}    &
{--}    & {--}  & {--} 
\\
&
DE-SimplE & 
{--}    & {--}  & {--}    &
0.12    & 0.14  & \better{0.03}    &
{--}    & {--}  & {--} 
\\
\hline
\multirow{3}{*}{\rotatebox[origin=c]{90}{YAGO}} &
DE-TransE &
0.06  & 0.32  & \sigbetter{0.26} &
0.32  & 0.06  & \sigworse{0.26} &
{--}  & {--} & {--} 
\\
&
DE-DistMult &
0.03    & 0.63  & \sigbetter{0.60}&
0.63    & 0.03  & \sigworse{0.60} &
{--}    & {--}  & {--}  
\\
&
DE-SimplE &
0.04    & 0.62  & \sigbetter{0.59}&
0.62    & 0.04  & \sigworse{0.58} &
{--}    & {--}  & {--} 
\\
\hline
\end{tabular}

\label{tab:relation_diff}
\end{minipage}
\end{table*}



Hypothesis \autoref{hyp:relation_properties} concerns relations with certain properties and their connection to model performance.

To evaluate this, the methods have been evaluated on test sets divided into a number of different relation properties, each test set containing no relation predictions.

In \autoref{app:test_set_statistics} the number of facts in each test set is detailed, as well as the number of types of relations.
For an overview of what each test set contains, see \autoref{app:test_sets_overview}.

ICEWS is the dataset best suited for analysis of this hypothesis, as there is a higher number of relations, with more varied relation properties.
All three sub-hypotheses of \autoref{hyp:relation_properties} refer to specific methods, and what they can and cannot represent. They are all diachronic embedding methods, and they will be compared to the other diachronic embedding methods, as they share the most characteristics with each other, but differentiate in the relations they can and cannot model.

To evaluate each subhypothesis we have analyzed and compared the performance of DE-TransE, DE-DistMult, and DE-SimplE on the different test sets, as can be seen in \autoref{tab:relation_diff}. Some datasets have no relations with some properties and those results are therefore blank.

\autoref{hyp:relation_property_sym} concerns symmetric relations. As DE-TransE cannot model symmetry but DE-DistMult and DE-SimplE can, we expect DE-TransE to achieve worse results on symmetric than non-symmetric relations, in relation to the results for DE-DistMult and DE-SimplE on those relations.
On WikiData no relations met the requirements of the soft label and therefore there is no data for that dataset.
On both ICEWS and YAGO, the symmetric relations seem to be simpler than the non-symmetric relations for the embedding methods to model -- They all achieve a better performance on the symmetric dataset than the non-symmetric dataset. What can also be observed is that DE-TransE has less of an improvement between the symmetric test sets and the non-symmetric test sets compared to DE-DistMult and DE-SimplE, as was theorised in \autoref{hyp:relation_property_sym}. This is illustrated for ICEWS in \autoref{fig:relation_difference_icews_symmetry}.
On ICEWS the performance of DE-TransE is similar across the symmetric test set and the non-symmetric, whereas the performance of the compared models is significantly better on the symmetric test set. On YAGO DE-TransE performs significantly better on symmetric than non-symmetric relations, but the performance of the compared models is improved by a larger margin.
The results are not as strong as we initially anticipated, but they still indicate that DE-TransE performs worse on symmetric relations than other models do, compared to their performance on non-symmetric relations. This indicates that \textbf{\autoref{hyp:relation_property_sym} is true}.

\begin{figure*}[htb]
\centering
\begin{minipage}{1.0\textwidth}
\centering
% SYMMETRY
\begin{minipage}[c]{0.33\textwidth}
\centering
\begin{subfigure}[c]{\textwidth}
\begin{tikzpicture}
\begin{axis}[
    ybar,
    bar width=8pt,
    xmin=-0.3,
    xmax=2.3,
    xtick={0,1,2},
    xticklabels={DE-T,DE-D,DE-S},
    ymin=0.00,
    ymax=0.20,
    yticklabels={y,0.00,0.05,0.10,0.15,0.20,0.25,0.30,0.35},
    ylabel={Diff. of MRR scores $T_S'$ and $T_S$},
    height=5cm,
    width=6cm,
]
\addplot[color=ourdarkgreen, fill=ourlightgreen] coordinates {
(0, 0.02) %DE_TransE
(1, 0.15) %DE_DistMult
(2, 0.14) %DE_SimplE
} ;
\addplot[ourblack,sharp plot,update limits=false,dashed] coordinates {
(-0.5, 0.02)
(2.5, 0.02)
} ;
\draw (0.6,0.02) -- (0.8,0.02) (0.7,0.02) -- (0.7,0.15) (0.6,0.15) -- (0.8,0.15);
\node at (0.4,0.085){\small{+0.13}}; %DE_DistMult
\draw (1.6,0.02) -- (1.8,0.02) (1.7,0.02) -- (1.7,0.14) (1.6,0.14) -- (1.8,0.14);
\node at (1.4,0.08){\small{+0.12}}; %DE_SimplE
\end{axis}
\end{tikzpicture}

\caption{Symmetry}
\label{fig:relation_difference_icews_symmetry}
\end{subfigure}
\end{minipage}
% ANTI-SYMMETRY
\begin{minipage}[c]{0.33\textwidth}
\centering
\begin{subfigure}[c]{\textwidth}
\begin{tikzpicture}
\begin{axis}[
    ybar,
    bar width=8pt,
    xmin=-0.9,
    xmax=2.3,
    xtick={0,1,2},
    xticklabels={DE-T,DE-D,DE-S},
    ymin=0.00,
    ymax=0.20,
    yticklabels={y,0.00,0.05,0.10,0.15,0.20,0.25,0.30,0.35},
    ylabel={Diff. of MRR scores $T_A'$ and $T_A$},
    height=5cm,
    width=6cm,
]
\addplot[color=ourdarkgreen, fill=ourlightgreen] coordinates {
(0, 0.09) %DE_TransE
(1, 0.03) %DE_DistMult
(2, 0.03) %DE_SimplE
} ;
\addplot[ourblack,sharp plot,update limits=false,dashed] coordinates {
(-1.5, 0.03)
(3.5, 0.03)
} ;
\draw (-0.4,0.09) -- (-0.2,0.09) (-0.3,0.09) -- (-0.3,0.03) (-0.4,0.03) -- (-0.2,0.03);
\node at (-0.6,0.06){\small{+0.06}}; %DE_TransE
\end{axis}
\end{tikzpicture}

\caption{Antisymmetry}
\label{fig:relation_difference_icews_antisymmetry}
\end{subfigure}
\end{minipage}
% INVERSION
\begin{minipage}[c]{0.33\textwidth}
\centering
\begin{subfigure}[c]{\textwidth}
\begin{tikzpicture}
\begin{axis}[
    ybar,
    bar width=8pt,
    xmin=-0.9,
    xmax=2.3,
    xtick={0,1,2},
    xticklabels={DE-T,DE-D,DE-S},
    ymin=0.00,
    ymax=0.20,
    yticklabels={y,0.00,0.05,0.10,0.15,0.20,0.25,0.30,0.35},
    ylabel={Diff. of MRR scores $T_I'$ and $T_I$},
    height=5cm,
    width=6cm,
]
\addplot[color=ourdarkgreen, fill=ourlightgreen] coordinates {
(0, 0.11) %DE_TransE
(1, 0.10) %DE_DistMult
(2, 0.13) %DE_SimplE
} ;
\addplot[ourblack,sharp plot,update limits=false,dashed] coordinates {
(-1.5, 0.10)
(3.5, 0.10)
} ;
\draw (-0.4,0.11) -- (-0.2,0.11) (-0.3,0.11) -- (-0.3,0.10) (-0.4,0.10) -- (-0.2,0.10);
\node at (-0.6,0.105){\small{+0.01}}; %DE_TransE
\draw (1.6,0.13) -- (1.8,0.13) (1.7,0.13) -- (1.7,0.10) (1.6,0.10) -- (1.8,0.10);
\node at (1.4,0.115){\small{+0.03}}; %DE_SimplE
\end{axis}
\end{tikzpicture}

\caption{Inversion}
\label{fig:relation_difference_icews_inversion}
\end{subfigure}
\end{minipage}
\end{minipage}
\begin{minipage}{\fullwidthcaption}
\caption{Comparison of differences in performance of models between test sets that contain a given relation property, and test set that do not contain that property in ICEWS.}
\label{fig:relation_difference_icews}
\end{minipage}
\end{figure*}


\autoref{hyp:relation_property_antisym} concerns antisymmetric relations. As DE-DistMult cannot model antisymmetry but DE-TransE and DE-SimplE can, we expect DE-DistMult to achieve worse results on antisymmetric than non-antisymmetric relations, in relation to the results for DE-TransE and DE-SimplE on those relations.
On ICEWS only DE-TransE has significantly better performance on the antisymmetric relations than non-antisymmetric relations.
On WikiData all models perform similarly and on YAGO all models perform worse on the antisymmetric relations.
The differences in improvement for antisymmetry on ICEWS are illustrated in \autoref{fig:relation_difference_icews_antisymmetry}.
As the figure shows, only DE-TransE performs significantly better than DE-DistMult, and the performance of DE-DistMult is almost the same as DE-SimplE.
We expected both DE-TransE and DE-SimplE to perform better across all datasets, but this is evidently not the case.
The same pattern is evident on YAGO, but there are no differences in improvement on WikiData.
Therefore, the results indicate that DE-DistMult perfoms similarly on antisymmetric relations and non-antisymmetric relations as other methods despite the theoretical disadvantage that DE-DistMult has, which indicates that \textbf{\autoref{hyp:relation_property_antisym} is false}.

\autoref{hyp:relation_property_inv} concerns inverse relations. As DE-DistMult cannot model inversion but DE-TransE and DE-SimplE can, we expect DE-DistMult to achieve worse results on inverse than non-inverse relations, in relation to the results for DE-TransE and DE-SimplE on those relations.
On WikiData and YAGO no relations met the requirements of the soft label and therefore there is no data for those datasets. 
The differences in improvement for ICEWS are illustrated in \autoref{fig:relation_difference_icews_inversion}. 
The comparison shows that DE-DistMult has higher performance on the inverse test set than the non-inverse test set, 
but that the other models improve by a larger margin, which is what we expected to see.
The difference is however not very pronouced, and the performance is overall similar to the performance of DE-SimplE. These results are not remarkable enough to confirm the hypothesis, and therefore the results indicate that \textbf{\autoref{hyp:relation_property_inv} is false}, however this result is only based on a single dataset, and therefore not very well supported.

Overall these results indicate that \textbf{\autoref{hyp:relation_properties} is true} for some methods and relation properties. This indicates that the performance of a model can in some cases be predicted by the theoretical limitations of the methods, but the model can also achieve good results despite them.