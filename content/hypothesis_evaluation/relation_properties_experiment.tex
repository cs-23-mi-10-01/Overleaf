\subsection{Relation Properties}
\label{sec:relation_properties_experiment}

\begin{table}[htb]
\centering
\begin{minipage}{0.95\columnwidth}
\centering
\caption{Test sets, and what they contain}
\vspace{-3mm}

\begin{tabular}{r|l}\hline
Test set & Contains \\ \hline
$T_D$ & Dense partiton timestamps \\ 
$T_P$ & Sparse partition timestamps \\
$T_S$ & Symmetrical relations \\ 
$T_S'$ & Non-symmetrical relations \\ 
$T_A$ & Anti-symmetrical relations \\ 
$T_A'$ & Non-antisymmetrical relations \\ 
$T_I$ & Inverse relations \\ 
$T_I'$ & Non-inverse relations \\ \hline
%$T_R$ & Reflexive relations \\ 
%$T_R'$ & Non-reflexive relations \\
\end{tabular}

\label{tab:test_set_explanations}
\end{minipage}
\end{table}


\begin{table}[htb]
\centering
\begin{minipage}{\columnwidthcaption}
\centering
\caption{Number of facts in each relation property test set}
\end{minipage}
\vspace{-3mm}

\resizebox{\columnwidth}{!}{
\begin{tabular}{lc|cc|cc|cc|cc}\hline
Dataset & $|\varR|$ & $|T_S'|$ & $|T_S|$ & $|T_A'|$ & $|T_A|$ & $|T_I'|$ & $|T_I|$ & $|T_R'|$ & $|T_R|$ \\ \hline
ICEWS14 & 220 & 22902 & 3987 & 23253 & 3636 & 25581 & 1308 & 26889 & 0 \\
WIKIDATA & 24 & 16248 & 0 & 580 & 15668 & 16248 & 0 & 16248 & 0 \\
YAGO & 10 & 7500 & 704 & 704 & 7500 & 8204 & 0 & 8204 & 0 \\
 \hline
\end{tabular}
}

\label{tab:relation_property_test_sets}
\end{table}


%\input{content/hypothesis_evaluation/tables/relation_properties_folder/icews14_original_timestamps}
%\input{content/hypothesis_evaluation/tables/relation_properties_folder/wikidata12k_original_timestamps}
%\input{content/hypothesis_evaluation/tables/relation_properties_folder/yago11k_original_timestamps}
\begin{table}[htb]
\centering
\begin{minipage}{\columnwidthcaption}
\centering
\caption{Hypothesis \autoref{hyp:relation_property_sym} comparison of DE-TransE with DE-DistMult and DE-Simple on symmetric relation testsets $T_S$ and non-symmeric relation testsets $T_S'$.}
\vspace{-3mm}

\begin{tabular}{r|ccc}\hline
ICEWS14 & DE-T & DE-D & DE-S \\ \hline
$T_S'$ & 0.24 & 0.33 & 0.34 \\
$T_S$ & 0.26 & 0.48 & 0.48 \\ \hline
Difference & +0.02 & +0.15 & +0.14 \\ \hline\hline
YAGO & DE-T & DE-D & DE-S \\ \hline
$T_S'$ & 0.06 & 0.03 & 0.04 \\
$T_S$ & 0.32 & 0.63 & 0.62 \\ \hline
Difference & +0.26 & +0.60 & +0.59 \\
 \hline
\end{tabular}

\label{tab:hypothesis_3_a_comparison}
\end{minipage}
\end{table}


\begin{table}[htb]
\centering
\begin{minipage}{0.95\columnwidth}
\centering
\caption{Hypothesis \autoref{hyp:relation_property_antisym} comparison of DE-DistMult with DE-TransE and DE-Simple on anti-symmetric relation testsets $T_A$ and non-antisymmeric relation testsets $T_A'$.}
\vspace{-3mm}

\begin{tabular}{r|ccc}\hline
ICEWS14 & DE-T & DE-D & DE-S \\ \hline
$T_A'$ & 0.23 & 0.35 & 0.36 \\
$T_A$ & 0.32 & 0.38 & 0.39 \\ \hline
Difference & +0.09 & +0.03 & +0.03 \\ \hline\hline
WIKIDATA & DE-T & DE-D & DE-S \\ \hline
$T_A'$ & 0.11 & 0.12 & 0.12 \\
$T_A$ & 0.14 & 0.14 & 0.14 \\ \hline
Difference & +0.03 & +0.02 & +0.03 \\ \hline\hline
YAGO & DE-T & DE-D & DE-S \\ \hline
$T_A'$ & 0.32 & 0.63 & 0.62 \\
$T_A$ & 0.06 & 0.03 & 0.04 \\ \hline
Difference & -0.26 & -0.60 & -0.58 \\ \hline
\end{tabular}

\label{tab:hypothesis_3_b_comparison}
\end{minipage}
\end{table}


\begin{table}[htb]
\centering
\begin{minipage}{\columnwidthcaption}
\centering
\caption{Hypothesis \autoref{hyp:relation_property_inv} comparison of DE-DistMult with DE-TransE and DE-Simple on inverse relation testsets $T_I$ and non-inverse relation testsets $T_I'$.}
\vspace{-3mm}

\begin{tabular}{r|ccc}\hline
\mbox{ICEWS14-7k} & DE-T & DE-D & DE-S \\ \hline
$T_I'$ & 0.24 & 0.35 & 0.36 \\
$T_I$ & 0.34 & 0.44 & 0.49 \\ \hline
Difference & +0.11 & +0.10 & +0.13 \\
 \hline
\end{tabular}

\label{tab:hypothesis_3_c_comparison}
\end{minipage}
\end{table}



To evaluate hypothesis \autoref{hyp:relation_properties} the methods have been evaluated on testsets divided into a number of different relation properties. The testsets of each dataset all contain a relation in the query, and predicts on head, tail or timestamp.

In \autoref{tab:relation_property_test_sets} the number of facts in each test set is detailed, as well as the number of types of relations.
For an overview of what each test set contains, see \autoref{tab:test_set_explanations}. There are no relations with the reflexive property in any dataset, and as such test set $T_R$ is empty for all datasets.
%$T_S$ contain all facts that has symmetrical relations, $T_S'$ contain all facts with non-symmetrical relations, $T_A$ contain all facts with anti-symmetrical relations, $T_A'$ contain all facts with non-antisymmetrical relations, $T_I$ contain all facts with inverse relations, $T_I'$ contain all facts with non-inverse relations. $T_R$ is the set of all facts with reflexive relations, but no relations in the datasets has this property, and therefore the set $T_R'$, which contain all facts with non-reflexive relations, contain all facts with known relation. 
The models will only be compared on testsets where there are facts that have a given property and facts that do not have that given property. ICEWS14 is the dataset best suited for analysis of this hypothesis, as there is a higher and more varied number of relation types, with more varied relation properties.

\autoref{tab:relation_properties_icews14_comparison} contains our findings for the dataset ICEWS14. The expected behaviour for DE-TransE is to have a worse score of symmetrical relations as this model cannot model symmetry, as is outlined in \autoref{tab:overview_of_models}. The results indicate that this is not the case -- DE-TransE has a similar overall score on test set $T_S$ as $T_S'$. Most of the models have a better score on $T_S$ however, which indicates that the symmetrical relations are simply easier to predict on for the models than the non-symmetrical ones.

DE-DistMult is expected to have a worse score for anti-symmetry and inversion. This is also not the case, as the score for anti-symmetry is similar to the score for non-antisymmetrical relations on DE-DistMult on dataset ICEWS14. The score for inverse relations are significantly higher than the score for non-inverse relation.

DE-SimplE is expected to combine the best parts of DE-TransE and DE-DistMult, and achieve similar or better results than the two other methods across relation types. This seems to be the case, as DE-Simple has a higher or similar MRR score than the other DE models on every testset.

TeRo seems to have a significantly higher MRR score on symmetrical and inverse relations, and a similar score on anti-symmetrical ranks. ATiSE follows the same pattern. \missing[Write something about TeRo and ATiSE]

TimePlex has a significantly lower score on symmetrical and inverse relations, and a significantly higher score on anti-symmetrical relations. This is the most unique distribution of best results, and might be related to the category of the model, as TimePlex is the only neural network model tested.

\autoref{tab:relation_properties_wikidata12k_comparison} contains our findings for quality of predictions on WIKIDATA. This dataset only has relations that have anti-symmetrical properties among the properties that we inspect, and therefore the figure contains only those findings. The models overall have similar performance across anti-symmetrical relations, which is only surprising on the DE-DistMult method, as this method is not supposed to be able to model anti-symmetry.

\autoref{tab:relation_properties_yago11k_comparison} contains our findings for the predictions of models on testsets in YAGO. The relation types are not very varied in YAGO, and each relation that is not symmetrical is anti-symmetrical and vice versa. The results illustrate that the symmetrical relations are overall easier for the models to make predictions on than the set of non-symmetrical relations.

Overall, from the findings in these tests it is possible to conclude that hypothesis \autoref{hyp:relation_properties} is false, and there seems to be no correlation between what each model is theoretically able to depict and what the models are able to depict in reality.


