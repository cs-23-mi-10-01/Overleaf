\subsection{Relation Properties}
\label{sec:relation_properties_experiment}

\begin{table}[htb]
\centering
\begin{minipage}{0.95\columnwidth}
\centering
\caption{Test sets, and what they contain}
\vspace{-3mm}

\begin{tabular}{r|l}\hline
Test set & Contains \\ \hline
$T_D$ & Dense partiton timestamps \\ 
$T_P$ & Sparse partition timestamps \\
$T_S$ & Symmetrical relations \\ 
$T_S'$ & Non-symmetrical relations \\ 
$T_A$ & Anti-symmetrical relations \\ 
$T_A'$ & Non-antisymmetrical relations \\ 
$T_I$ & Inverse relations \\ 
$T_I'$ & Non-inverse relations \\ \hline
%$T_R$ & Reflexive relations \\ 
%$T_R'$ & Non-reflexive relations \\
\end{tabular}

\label{tab:test_set_explanations}
\end{minipage}
\end{table}


\begin{table}[htb]
\centering
\begin{minipage}{\columnwidthcaption}
\centering
\caption{Number of facts in each relation property test set}
\end{minipage}
\vspace{-3mm}

\resizebox{\columnwidth}{!}{
\begin{tabular}{lc|cc|cc|cc|cc}\hline
Dataset & $|\varR|$ & $|T_S'|$ & $|T_S|$ & $|T_A'|$ & $|T_A|$ & $|T_I'|$ & $|T_I|$ & $|T_R'|$ & $|T_R|$ \\ \hline
ICEWS14 & 220 & 22902 & 3987 & 23253 & 3636 & 25581 & 1308 & 26889 & 0 \\
WIKIDATA & 24 & 16248 & 0 & 580 & 15668 & 16248 & 0 & 16248 & 0 \\
YAGO & 10 & 7500 & 704 & 704 & 7500 & 8204 & 0 & 8204 & 0 \\
 \hline
\end{tabular}
}

\label{tab:relation_property_test_sets}
\end{table}


%\input{content/hypothesis_evaluation/tables/relation_properties_folder/icews14_original_timestamps}
%\input{content/hypothesis_evaluation/tables/relation_properties_folder/wikidata12k_original_timestamps}
%\input{content/hypothesis_evaluation/tables/relation_properties_folder/yago11k_original_timestamps}
\begin{table}[htb]
\centering
\begin{minipage}{\columnwidthcaption}
\centering
\caption{Hypothesis \autoref{hyp:relation_property_sym} comparison of DE-TransE with DE-DistMult and DE-Simple on symmetric relation testsets $T_S$ and non-symmeric relation testsets $T_S'$.}
\vspace{-3mm}

\begin{tabular}{r|ccc}\hline
ICEWS14 & DE-T & DE-D & DE-S \\ \hline
$T_S'$ & 0.24 & 0.33 & 0.34 \\
$T_S$ & 0.26 & 0.48 & 0.48 \\ \hline
Difference & +0.02 & +0.15 & +0.14 \\ \hline\hline
YAGO & DE-T & DE-D & DE-S \\ \hline
$T_S'$ & 0.06 & 0.03 & 0.04 \\
$T_S$ & 0.32 & 0.63 & 0.62 \\ \hline
Difference & +0.26 & +0.60 & +0.59 \\
 \hline
\end{tabular}

\label{tab:hypothesis_3_a_comparison}
\end{minipage}
\end{table}


\begin{table}[htb]
\centering
\begin{minipage}{0.95\columnwidth}
\centering
\caption{Hypothesis \autoref{hyp:relation_property_antisym} comparison of DE-DistMult with DE-TransE and DE-Simple on anti-symmetric relation testsets $T_A$ and non-antisymmeric relation testsets $T_A'$.}
\vspace{-3mm}

\begin{tabular}{r|ccc}\hline
ICEWS14 & DE-T & DE-D & DE-S \\ \hline
$T_A'$ & 0.23 & 0.35 & 0.36 \\
$T_A$ & 0.32 & 0.38 & 0.39 \\ \hline
Difference & +0.09 & +0.03 & +0.03 \\ \hline\hline
WIKIDATA & DE-T & DE-D & DE-S \\ \hline
$T_A'$ & 0.11 & 0.12 & 0.12 \\
$T_A$ & 0.14 & 0.14 & 0.14 \\ \hline
Difference & +0.03 & +0.02 & +0.03 \\ \hline\hline
YAGO & DE-T & DE-D & DE-S \\ \hline
$T_A'$ & 0.32 & 0.63 & 0.62 \\
$T_A$ & 0.06 & 0.03 & 0.04 \\ \hline
Difference & -0.26 & -0.60 & -0.58 \\ \hline
\end{tabular}

\label{tab:hypothesis_3_b_comparison}
\end{minipage}
\end{table}


\begin{table}[htb]
\centering
\begin{minipage}{\columnwidthcaption}
\centering
\caption{Hypothesis \autoref{hyp:relation_property_inv} comparison of DE-DistMult with DE-TransE and DE-Simple on inverse relation testsets $T_I$ and non-inverse relation testsets $T_I'$.}
\vspace{-3mm}

\begin{tabular}{r|ccc}\hline
\mbox{ICEWS14-7k} & DE-T & DE-D & DE-S \\ \hline
$T_I'$ & 0.24 & 0.35 & 0.36 \\
$T_I$ & 0.34 & 0.44 & 0.49 \\ \hline
Difference & +0.11 & +0.10 & +0.13 \\
 \hline
\end{tabular}

\label{tab:hypothesis_3_c_comparison}
\end{minipage}
\end{table}



To evaluate hypothesis \autoref{hyp:relation_properties} the methods have been evaluated on testsets divided into a number of different relation properties. The testsets of each dataset all contain a relation in the query, and predicts on head, tail or timestamp.

In \autoref{tab:relation_property_test_sets} the number of facts in each test set is detailed, as well as the number of types of relations.
For an overview of what each test set contains, see \autoref{tab:test_set_explanations}. There are no relations with the reflexive property in any dataset, and as such test set $T_R$ is empty for all datasets.
The models will only be compared on testsets where there are facts that have a given property and facts that do not have that given property. ICEWS14 is the dataset best suited for analysis of this hypothesis, as there is a higher number of relation types, with more varied relation properties.

All three sub-hypotheses of \autoref{hyp:relation_properties} all refer to specific methods, and what they can and cannot represent. They are all diacronic embedding methods, and they will be compared to the other diachronic embedding methods, as they share the most characteristics with eachother.

To evaluate on \autoref{hyp:relation_property_sym}, we have analyzed and compared the performance of DE-TransE, DE-DistMult, and DE-SimplE, as can be seen in \autoref{tab:hypothesis_3_a_comparison}.
On both dataset ICEWS14 and YAGO, the symmetrical relations seem to be significantly more simple for the embedding methods to represent -- They all achieve a higher performance on the symmetrical dataset than the non-symmetrical dataset. What can also be observed is that DE-TransE has a lower difference in improvement between the symmetrical testsets and the non-symmetrical testsets. In ICEWS14 the performance of DE-TransE is similar across the symmetrical testset as the non-symmetrical. The results are not as strong as we initally anticipated, but it still indicates that DE-TransE performs worse on symmetrical relations than other models do, compared to their performance on non-symmetrical relations, and this indicates that \autoref{hyp:relation_property_sym} is true.

For our analysis of \autoref{hyp:relation_property_antisym}, we have compared the performance of DE-DistMult with DE-TransE and DE-SimplE, as can be seen in \autoref{tab:hypothesis_3_b_comparison}.
As the table shows, there is little that indicates that DE-DistMult is worse at anti-symmetric relations than the other DE models, and the performance of DE-DistMult is almost the same as DE-SimplE. This indicates that DE-DistMult perfoms similarly on anti-symmetrical relations as non-antisymmetrical relations, which indicates that \autoref{hyp:relation_property_antisym} is false, despite the theoretical disadvantage that DE-DistMult has.

Finally, to evaluate \autoref{hyp:relation_property_inv}, we have compared the performance of DE-DistMult with the performance of DE-TransE and DE-SimplE, on an inverse relation testset and a non-inverse relation testset in \autoref{tab:hypothesis_3_c_comparison}. Only ICEWS14 has inverse relations.
The comparison shows that DE-DistMult has higher performance on the inverse testset than the non-inverse testset, but the difference between these two performance results are lower then on other models. The difference is not very significant however, and the performance is overall similar to the performance of DE-SimplE. These results are not sigificant enough to confirm the hypothesis, and therefore the results indicate that \autoref{hyp:relation_property_inv} is false, however this result is only based on a single dataset, and therefore is not very well supported.

Overall these results seem to indicate that \autoref{hyp:relation_properties} is true under some circumstances, which indicate that the real performance of each model can in some cases be predicted by the theoretical limitations of the methods, but the methods will also achieve good results despite it.
