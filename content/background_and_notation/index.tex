\section{Background and Notation}
\label{sec:background-and-notation}

Let $\varE$, $\varR$ and $\varT$ represent the set of all \textbf{entities}, \textbf{relations} and \textbf{timestamps} respectively. In addition, the \textit{missing timestamp} $\varmiss$ is also contained in $\varT$. \textbf{Facts} are represented as $(h, r, t, \tau)$, where $h, t \in \varE$, $r \in \varR$ and $\tau \in \varT^2$. $\tau$ is a \textbf{timespan} consisting of two timestamps, where $\timebegin{\tau}$ refers to the first timestamp, and $\timeend{\tau}$ refers to the last. Let $\zeta$ represent all true facts in a world, and $\zeta'$ represent all false facts. Let $\varK$ be a \textbf{temporal knowledge graph}, which is a subset of $\zeta$. Let $\eta_r \subset \varK$ represent all facts in the knowledge graph where $r$ is the relation in the fact.

We define $(h, [r_1, \dots, r_n], t, \tau)$ to be a chain of relations from $h$ to $t$, such that $(h, r_1, e_1, \tau_1), (e_1, r_2, e_2, \tau_2), \dots, (e_{n-1}, r_n, t, \tau_n)$ where $\timebegin{\tau_1} \leq \timebegin{\tau_2} \leq \dots \leq \timebegin{\tau_n}$, and $\timebegin{\tau} = \timebegin{\tau_1}, \timeend{\tau} = \timeend{\tau_n}$.

A relation $r$ is \textbf{symmetric} if $(e_1, r, e_2, \tau) \in \zeta \Leftrightarrow (e_2, r, e_1, \tau) \in \zeta$ for some entities $e_1, e_2 \in \varE$ and timespans $\tau \in \varT^2$.
A relation $r$ is \textbf{anti-symmetric} if $(e_1, r, e_2, \tau) \in \zeta \Rightarrow (e_2, r, e_1, \tau) \in \zeta'$ for some entities $e_1, e_2 \in \varE$ and timespans $\tau \in \varT^2$.
A relation $r^{-1}$ is the \textbf{inversion} of relation $r$ if $(e_1, r, e_2, \tau) \in \zeta \Leftrightarrow (e_2, r^{-1}, e_1, \tau) \in \zeta$ for some entities $e_1, e_2 \in \varE$ and timespans $\tau \in \varT^2$.
A relation $r$ is the \textbf{composition} of relations $r_1, \dots r_n$ if $(e_1, [r_1, \dots, r_n], e_2, \tau) \in \zeta \Rightarrow (e_1, r, e_2, \tau) \in \zeta$ for some entities $e_1, e_2 \in \varE$ and timespans $\tau \in \varT^2$.
A relation $r$ is \textbf{reflexive} if $(e_1, r, e_2, \tau) \in \zeta \Rightarrow e_1 = e_2$ for some entities $e_1, e_2 \in \varE$ and timespans $\tau \in \varT^2$.

Each knowledge graph is split into a number of \textbf{train} sets $\train_i$, \textbf{validation} sets $\valid_i$ and \textbf{test} sets $\test_i$. No facts are shared among the sets within the same split $\train_i \cap \valid_i = \emptyset$, $\valid_i \cap \test_i = \emptyset$, $\train_i \cap \test_i = \emptyset$, but all facts are contained in the sets $\train_i \cup \valid_i \cup \test_i = \varK$.

For the statistical analysis of the report, we define the MRR score $a$ to be significantly higher than the MRR score $b$ if $a > b + (1-b) \times 0.1$. Conversely, we define the MRR score $a$ to be significantly lower than the MRR score $b$ if $a < b \times 0.9$. We define significant results for Mean Reciprocal Precision (MRP) similarly, and MRP is defined in section \missing[X].

