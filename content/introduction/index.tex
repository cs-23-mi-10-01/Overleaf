\section{Introduction}
\label{sec:introduction}

Questions can be divided into three types: Simple, multi-hop and complex.
Simple questions can be answered with a single triple \cite{bordes2015simple}, multi-hop questions require a path of triples \cite{zhang2017multihop}, and complex questions require logical reasoning over multiple triples \cite{talmorberant2018decomposition}.
We focus on simple questions as complex and multi-hop questions can be deconstructed to multiple simple questions. \cite{yani2021}

Most \gls{qa} systems consist of two parts: (1) Translating a question in natural language to a query (2) Resolving the query.
We do not consider the first problem as the aim of the survey is to investigate embedding methods and their qualities.
As simple questions can be expressed as a tuple, the task of simple \gls{qa} here corresponds to the task of link prediction, which enables us to select datasets developed for that task as well as datasets developed for \gls{qa}.

This paper is a continuation of \cite{P9}, where we found that the quality of the are highly dependant on the target of the prediction. Highest performance was observed on tail predictions, followed by head, relation and time predictions. In this paper, this observation will be more thoroughly examined and tested with more datasets and train/test splits of the data.

In addition, it was observed that the best indicator for prediction quality seemed to be the relation type. In this paper, this concept will be expanded, by looking at the relations in the knowledge graph, categorizing them and evaluating the performance of the models over different categories of relations. The relations are categorized into the structural features symmetry, anti-symmetry, inversion, composition and hierarchy, which will be defined in this paper in relation to temporal information. Their temporal nature will also be examined, such as the average duration of events and relations.

Ultimately, the findings will contribute to an ensemble learning model with dynamic weights across different models, depending on the features of the query.