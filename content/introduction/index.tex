\section{Introduction}
\label{sec:introduction}

% explain kg, fact, tkg, embedding, link prediction
A \gls{kg} is a data structure that stores multi-relational data, typically in the form of a directed, edge-labelled graph, where nodes represent entities and edges represent relations.
Information is stored as facts of format $(h,r,t)$, where $h$ and $t$ are entities and $r$ is a relation.
A \gls{kg} can be extended with temporal information in the form of timestamps, $\tau$. This is called a \gls{tkg}.
A \gls{kge} is a low-dimensional, numerical representation of a \gls{kg}, to enable data to be transformed and analyzed mathematically.
Embeddings are typically evaluated over the link prediction task, which aims to find a missing fact element in a query.

% explain background (last semester)
This work is a continuation of our previous work \cite{P9}, where we investigated characteristics of selected \gls{kge} methods and their influence on link prediction results. For this paper we extend our work with another \gls{kge} method, additional datasets, new hypotheses as well as use our findings to construct an ensemble learning method.

% explain preliminary experiments + relation observations
Previously, we found indications that the quality of the predictions are highly dependent on the target of the prediction, which we examine more thoroughly by testing on multiple datasets and splits. In addition, we observed that the relation type in a query is also a good indicator for prediction quality, and in this paper this observation will be further expanded by categorizing relations based on the structure of the surrounding graph, including the temporal information, and evaluating performance of models across these categories. Additionally, we examine the degree of precision exhibited by trained models when predicting on timestamps.
% explain hypotheses
Here we formulate hypotheses on the role of temporal data density in link prediction results, potential of combining model timestamp predictions and methods' ability to model relations with different properties.
% explain ensemble learning
Finally, the findings are utilized in the construction of an ensemble learning model with dynamic weights across different models, depending on the features of the query.

% contributions
Our contributions are \missing[hopefully brilliant].

% structure of paper
The notation used is documented in \autoref{sec:background-and-notation} and the related work is examined in \autoref{sec:related-work}.
In \autoref{sec:hypotheses} is an in-depth explanation of our hypotheses, the intuition behind them and how we intend to examine them with illustrated examples.
Our selection of methods and datasets are explained in \autoref{sec:methods} and \autoref{sec:datasets} respectively.
The experiments not conducted as part of a hypothesis are covered in \autoref{sec:preliminary_experiments} and the experiments conducted to evaluate hypotheses are covered in \autoref{sec:hypothesis_evaluation}.
Details about the ensemble learning model and how it uses the results are explained in \autoref{sec:ensemble_model}.
The evaluation, conclusion and future prospects of our work are detailed in \autoref{sec:evaluation}, \autoref{sec:conclusion}, and \autoref{sec:future-work} respectively.


\begin{comment}
Questions can be divided into three types: Simple, multi-hop and complex.
Simple questions can be answered with a single triple \cite{bordes2015simple}, multi-hop questions require a path of triples \cite{zhang2017multihop}, and complex questions require logical reasoning over multiple triples \cite{talmorberant2018decomposition}.
We focus on simple questions as complex and multi-hop questions can be deconstructed to multiple simple questions. \cite{yani2021}

Most \gls{qa} systems consist of two parts: (1) Translating a question in natural language to a query (2) Resolving the query.
We do not consider the first problem as the aim of the survey is to investigate embedding methods and their qualities.
As simple questions can be expressed as a tuple, the task of simple \gls{qa} here corresponds to the task of link prediction, which enables us to select datasets developed for that task as well as datasets developed for \gls{qa}.

This paper is a continuation of \cite{P9}, where we found that the quality of the predictions are highly dependant on the target of the prediction. Highest performance was observed on tail predictions, followed by head, relation and time predictions. In this paper, this observation will be more thoroughly examined and tested with more datasets and train/test splits of the data.

In addition, we observed that the best indicator for prediction quality seemed to be the relation type. In this paper, this concept will be expanded, by looking at the relations in the knowledge graph, categorizing them and evaluating the performance of the models over different categories of relations. The relations are categorized into the structural features symmetry, anti-symmetry, inversion, composition and hierarchy, which will be defined in this paper in relation to temporal information. Their temporal nature will also be examined, such as the average duration of events and relations.

Ultimately, the findings will contribute to an ensemble learning model with dynamic weights across different models, depending on the features of the query.
\end{comment}