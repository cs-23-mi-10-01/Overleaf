\section{Introduction}
\label{sec:introduction}

% explain kg, fact, tkg, embedding, link prediction
A \gls{kg} is a data structure that stores multi-relational data, typically in the form of a directed, edge-labelled graph, where nodes represent entities and edges represent relations.
Information is stored as facts of format $(h,r,t)$, where elements $h$ and $t$ are entities and element $r$ is a relation.
A \gls{tkg} is a \gls{kg} where the facts have been extended with a time information element $\tau$, either as a singular timestamp or as a timespan.
A query is a fact with a missing element.
A \gls{kge} is a low-dimensional, numerical representation of a \gls{kg}, and facilitates link prediction, which aims to find the missing fact element of a query.
The result of link prediction on a query is a ranked list of most probable answers and the quality of an embedding can be determined by the average rank of the correct answers \cite{knowledge_graphs_survey}.

% goal of the paper
The goal of this paper is to analyze the characteristics of link prediction queries to find which characteristics suit which types of models, enabling the construction of an ensemble method where models are prioritized according to the analysis findings and characteristics in the query.
These characteristics depend on the information available in the elements of the query, as well as the structure of the \gls{kg} and the available temporal information. 

% explain preliminary experiments + relation observations
This work is a continuation of our previous work \cite{P9}, where we investigated characteristics of selected \gls{kge} methods and their influence on link prediction.
In the previous work, we found indications that the quality of the predictions is highly dependent on the target of the prediction, which we in this work examine more thoroughly as a preliminary experiment by testing on multiple datasets and splits of those datasets.

% explain hypotheses
We formulate hypotheses on the role of temporal data density in link prediction results, potential of combining time predictions of models and methods' ability to model relations with different properties.
Temporal density refers to the amount of facts within a timespan, and datasets are split into sparse and dense partitions, enabling us to examine model performance depending on the temporal density.
The possible relation properties are symmetric, antisymmetric and inverse, and each relation can have more than one assigned property, based on the structure and temporal information of the surrounding graph.
% explain ensemble learning
Finally, the findings are utilized in the construction of an ensemble model ORB-E with dynamic weights across different models, depending on the characteristics of the query.
It is evaluated against a naive version that assigns the same weight to each included model.
Additionally, we make use of the continuous nature of timestamps to examine the degree of precision in time predictions, and present an approach of how to improve precision using joint selection of most probable timestamps over several models.


% contributions
Our contributions are an in-depth analysis of query characteristics, and how they affect the performance of link prediction for each \gls{kge} method.
We perform link prediction on both entities, relations and timestamps, of which only entities are standard practice.
Time and relation predictions are both a focus in this work, which is not widely studied in other articles.
We evaluate accuracy of time predictions, which is also not widely studied.
Furthermore, the results gained from this analysis are utilized to create a new ensemble method called ORB-E, which yield better results than all individual models across all datasets.

% structure of paper
The notation used is documented in \autoref{sec:background-and-notation} and the related work is examined in \autoref{sec:related-work}.
\Autoref{sec:hypotheses} contains an in-depth explanation of our hypotheses, the intuition behind them and how we intend to examine them.
Our selection of methods and datasets are explained in \autoref{sec:methods} and \autoref{sec:datasets} respectively.
The experiments not conducted as part of a hypothesis are covered in \autoref{sec:preliminary_experiments} and the experiments conducted to evaluate hypotheses are covered in \autoref{sec:hypothesis_evaluation}.
Details about the ensemble method and how it uses the results are explained in \autoref{sec:ensemble_model}.
The discussion, conclusion and future prospects of our work are detailed in \autoref{sec:evaluation}, \autoref{sec:conclusion}, and \autoref{sec:future-work} respectively.