% FOR REVIEW
%\documentclass[manuscript,screen,timestamp,review,nonacm=true]{acm_files/acmart}
% FOR SUBMISSION
\documentclass[sigconf,screen,nonacm=true]{acm_files/acmart}
%\pagenumbering{roman}

\citestyle{acmauthoryear}

\usepackage[acronym, nonumberlist, nopostdot]{glossaries}
\usepackage{xargs}
\usepackage{ifthen}
\usepackage{hyperref}
\usepackage{enumitem}
\usepackage{pgfplots}
\usepackage{pgf}
\usepackage{collcell}
\usepackage{booktabs}
\usepackage[super]{nth}
\usepackage{array}
\usepackage{makecell}
\usepackage{titlesec}
\usepackage{soul}
\usepackage{amsfonts}
\usepackage{amsmath}
%\usepackage{amssymb}
\usepackage{scalerel}
\usepackage{stackengine,wasysym}

%%%%%%%%%%%% MATH SYMBOLS %%%%%%%%%%%%

% Ascii symbols: https://www.ascii-code.com/

% Cursive
\DeclareSymbolFont{mathcal}{OML}{txmi}{m}{it}
\DeclareMathSymbol{\varamathcal}{\mathord}{mathcal}{97} % <- Ascii code for "a"
\DeclareMathSymbol{\varbmathcal}{\mathord}{mathcal}{98}
\DeclareMathSymbol{\varcmathcal}{\mathord}{mathcal}{99}
\DeclareMathSymbol{\vardmathcal}{\mathord}{mathcal}{100}
\DeclareMathSymbol{\varemathcal}{\mathord}{mathcal}{101}
\DeclareMathSymbol{\varfmathcal}{\mathord}{mathcal}{102}
\DeclareMathSymbol{\vargmathcal}{\mathord}{mathcal}{103}
\DeclareMathSymbol{\varhmathcal}{\mathord}{mathcal}{104}
\DeclareMathSymbol{\varimathcal}{\mathord}{mathcal}{105}
\DeclareMathSymbol{\varjmathcal}{\mathord}{mathcal}{106}
\DeclareMathSymbol{\varkmathcal}{\mathord}{mathcal}{107}
\DeclareMathSymbol{\varlmathcal}{\mathord}{mathcal}{108}
\DeclareMathSymbol{\varmmathcal}{\mathord}{mathcal}{109}
\DeclareMathSymbol{\varnmathcal}{\mathord}{mathcal}{110}
\DeclareMathSymbol{\varomathcal}{\mathord}{mathcal}{111}
\DeclareMathSymbol{\varpmathcal}{\mathord}{mathcal}{112}
\DeclareMathSymbol{\varqmathcal}{\mathord}{mathcal}{113}
\DeclareMathSymbol{\varrmathcal}{\mathord}{mathcal}{114}
\DeclareMathSymbol{\varsmathcal}{\mathord}{mathcal}{115}
\DeclareMathSymbol{\vartmathcal}{\mathord}{mathcal}{116}
\DeclareMathSymbol{\varumathcal}{\mathord}{mathcal}{117}
\DeclareMathSymbol{\varvmathcal}{\mathord}{mathcal}{118} 
\DeclareMathSymbol{\varwmathcal}{\mathord}{mathcal}{119}
\DeclareMathSymbol{\varxmathcal}{\mathord}{mathcal}{120}
\DeclareMathSymbol{\varymathcal}{\mathord}{mathcal}{121}
\DeclareMathSymbol{\varzmathcal}{\mathord}{mathcal}{122}

% Double struck
\newcommand{\R}{\mathbb{R}}
\newcommand{\C}{\mathbb{C}}

% Sum-style symbols
\DeclareMathOperator*{\doublepipe}{||}
\DeclareMathOperator*{\maxop}{\mathit{max}}

% Special "hat" symbols
\newcommand\reallywidetilde[1]{\ThisStyle{%
  \setbox0=\hbox{$\SavedStyle#1$}%
  \stackengine{-.1\LMpt}{$\SavedStyle#1$}{%
    \stretchto{\scaleto{\SavedStyle\mkern.2mu\AC}{.5150\wd0}}{.3\ht0}%
  }{O}{c}{F}{T}{S}%
}}

\newcommand\reallywidebar[1]{\ThisStyle{%
  \setbox0=\hbox{$\SavedStyle#1$}%
  \stackengine{-.1\LMpt}{$\SavedStyle#1$}{%
    \stretchto{\scaleto{\SavedStyle\mkern.2mu-}{.5150\wd0}}{.3\ht0}%
  }{O}{c}{F}{T}{S}%
}}

\newcommand\reallywidehat[1]{\ThisStyle{%
  \setbox0=\hbox{$\SavedStyle#1$}%
  \stackengine{-.1\LMpt}{$\SavedStyle#1$}{%
    \stretchto{\scaleto{\SavedStyle\mkern.2mu\wedge}{.5150\wd0}}{.3\ht0}%
  }{O}{c}{F}{T}{S}%
}}

\newcommand\reallywidehatwithnum[2]{\ThisStyle{%
  \setbox0=\hbox{$\SavedStyle#2$}%
  \stackengine{-.1\LMpt}{$\SavedStyle#2$}{%
    \stretchto{\scaleto{\SavedStyle\mkern.2mu\wedge}{.5150\wd0}}{.3\ht0}\footnotesize{$#1$}%
  }{O}{c}{F}{T}{S}%
}}

\newcommand\timebegin[1]{\ThisStyle{%
  \setbox0=\hbox{$\SavedStyle#1$}%
  \stackengine{-.1\LMpt}{$\SavedStyle#1$}{%
    \stretchto{\scaleto{\SavedStyle\mkern.2mu\vdash}{1\wd0}}{.55\ht0}%
  }{O}{c}{F}{T}{S}%
}}

\newcommand\timeend[1]{\ThisStyle{%
  \setbox0=\hbox{$\SavedStyle#1$}%
  \stackengine{-.1\LMpt}{$\SavedStyle#1$}{%
    \stretchto{\scaleto{\SavedStyle\mkern.2mu\dashv}{1\wd0}}{.55\ht0}%
  }{O}{c}{F}{T}{S}%
}}


%%%%%%%%%%%%%%% COMMANDS %%%%%%%%%%%%%%%
\newcommand{\vara}{\varamathcal}
\newcommand{\varb}{\varbmathcal}
\newcommand{\varc}{\varcmathcal}
\newcommand{\vard}{\vardmathcal}
\newcommand{\vare}{\varemathcal}
\newcommand{\varf}{\varfmathcal}
%\newcommand{\varg}{\vargmathcal}
\newcommand{\varh}{\varhmathcal}
\newcommand{\vari}{\varimathcal}
\newcommand{\varj}{\varjmathcal}
\newcommand{\vark}{\varkmathcal}
\newcommand{\varl}{\varlmathcal}
\newcommand{\varm}{\varmmathcal}
\newcommand{\varn}{\varnmathcal}
\newcommand{\varo}{\varomathcal}
\newcommand{\varp}{\varpmathcal}
\newcommand{\varq}{\varqmathcal}
\newcommand{\varr}{\varrmathcal}
\newcommand{\vars}{\varsmathcal}
\newcommand{\vart}{\vartmathcal}
\newcommand{\varu}{\varumathcal}
%\newcommand{\varv}{\varvmathcal}
%\newcommand{\varw}{\varwmathcal}
\newcommand{\varx}{\varxmathcal}
%\newcommand{\vary}{\varymathcal}
\newcommand{\varz}{\varzmathcal}
\newcommand{\varA}{\mathcal{A}}
\newcommand{\varB}{\mathcal{B}}
\newcommand{\varC}{\mathcal{C}}
\newcommand{\varD}{\mathcal{D}}
\newcommand{\varE}{\mathcal{E}}
\newcommand{\varF}{\mathcal{F}}
\newcommand{\varG}{\mathcal{G}}
\newcommand{\varH}{\mathcal{H}}
\newcommand{\varI}{\mathcal{I}}
\newcommand{\varJ}{\mathcal{J}}
\newcommand{\varK}{\mathcal{K}}
\newcommand{\varL}{\mathcal{L}}
\newcommand{\varM}{\mathcal{M}}
\newcommand{\varN}{\mathcal{N}}
\newcommand{\varO}{\mathcal{O}}
\newcommand{\varP}{\mathcal{P}}
\newcommand{\varQ}{\mathcal{Q}}
\newcommand{\varR}{\mathcal{R}}
\newcommand{\varS}{\mathcal{S}}
\newcommand{\varT}{\mathcal{T}}
\newcommand{\varU}{\mathcal{U}}
\newcommand{\varV}{\mathcal{V}}
\newcommand{\varW}{\mathcal{W}}
\newcommand{\varX}{\mathcal{X}}
\newcommand{\varY}{\mathcal{Y}}
\newcommand{\varZ}{\mathcal{Z}}

\newcommand{\varconcat}{\doublepipe\limits}
\newcommand{\varsum}{\sum\limits}
\newcommand{\varmax}{\maxop\limits}







\PassOptionsToPackage{dvipsnames}{xcolor}
\pgfplotsset{compat=1.18}
\setlength{\footskip}{14pt}

\makeglossaries
\newcommand{\citethis}{\textsuperscript{[cite this]}}

\newcommand{\missing}[1][MISSING]{\hl{#1}}

\newcommand{\tabletext}{\footnotesize}

\makeatletter
\newcommand{\customlabel}[2]{%
   \protected@write \@auxout {}{\string \newlabel {#1}{{#2}{\thepage}{#2}{#1}{}} }%
   \hypertarget{#1}{#2}
}
\makeatother

% Heatmap table commands
\definecolor{lightred}{RGB}{255,125,125}
\definecolor{darkred}{RGB}{75,20,20}
\definecolor{customblue}{RGB}{100,120,200}
\definecolor{customred}{RGB}{200,120,100}

\newcommand*{\MinColor}{lightred}
\newcommand*{\NeutralColor}{white}
\newcommand*{\MaxColor}{darkred}
\newcommand*{\MinNumber}{-0.5}
\newcommand*{\NeutralNumber}{0}
\newcommand*{\MaxNumber}{0.83}
\newcommand{\ApplyMonoGradient}[1]{%
  \pgfmathsetmacro{\Percent}{min(100.0,max(0,100.0*(#1-\MinNumber)/(\MaxNumber-\MinNumber)))}%
  \edef\x{\noexpand\cellcolor{\MaxColor!\Percent!\MinColor}}\x\textcolor{white}{#1}%
}
\newcommand{\ApplyDuoGradient}[1]{%
  \pgfmathsetmacro{\PercentMin}{min(100.0,max(0,100.0*(#1-\MinNumber)/(\NeutralNumber-\MinNumber)))}%
  \pgfmathsetmacro{\PercentMax}{min(100.0,max(0,100.0*(#1-\NeutralNumber)/(\MaxNumber-\NeutralNumber)))}%
  \edef\x{\noexpand\cellcolor{\MaxColor!\PercentMax!\NeutralColor!\PercentMin!\MinColor}}\x\textcolor{black}{#1}%
}
\newcolumntype{R}{>{\collectcell\ApplyMonoGradient}{c}<{\endcollectcell}}
\newcolumntype{S}{>{\collectcell\ApplyDuoGradient}{c}<{\endcollectcell}}
%\newcolumntype{T}{>{\footnotesize\raggedleft\let\newline\\\arraybackslash\hspace{0pt}}m{.2\linewidth}}
% Heatmap table commands end

%wrap text in cells
\newcolumntype{T}[1]{>{%
\raggedright\let\newline\\\arraybackslash\hspace{0pt}%
}p{{#1}\linewidth}}%

\newcolumntype{M}{>{
\centering\let\newline\\\arraybackslash\hspace{0pt}
}m{.15\linewidth}}
\renewcommand{\theadfont}{\bfseries\normalsize}

\newcommand{\weighttable}[5]{%
\addvspace{2mm}%
\noindent%
\begin{tabular}{p{.25\linewidth} T{.75}}
    $\mathbf{{#1}}$ & \vspace{1mm}\\
    Description: & {#2} \\
    Value: & {#3} \\
    Reasoning: & {#4}\\
    Notes: & {#5}
\end{tabular}}

\graphicspath{ {./media/} }

\AtBeginDocument{%
  \providecommand\BibTeX{{%
    \normalfont B\kern-0.5em{\scshape i\kern-0.25em b}\kern-0.8em\TeX}}}

%% Rights management information.  This information is sent to you
%% when you complete the rights form.  These commands have SAMPLE
%% values in them; it is your responsibility as an author to replace
%% the commands and values with those provided to you when you
%% complete the rights form.
%\setcopyright{cc}
\copyrightyear{2022}

\begin{document}

\fancyfoot[C]{\thepage/\pageref*{TotPages}}
\title{Temporally constrained properties of relations and their impact on link prediction}
\renewcommand{\footskip}{16mm}

\author{Astrid Ipsen}
\email{aipsen18@student.aau.dk}
\affiliation{%
  \institution{Aalborg University}
  \streetaddress{Fredrik Bajers Vej 7K}
  \city{Aalborg Øst}
  \country{Denmark}
  \postcode{9220}
}

\author{Jeppe W. Lindberg}
\email{jwli21@student.aau.dk}
\affiliation{%
  \institution{Aalborg University}
  \streetaddress{Fredrik Bajers Vej 7K}
  \city{Aalborg Øst}
  \country{Denmark}
  \postcode{9220}
}

\author{Jonas C. Lindberg}
\email{jlindb18@student.aau.dk}
\affiliation{%
  \institution{Aalborg University}
  \streetaddress{Fredrik Bajers Vej 7K}
  \city{Aalborg Øst}
  \country{Denmark}
  \postcode{9220}
}
%\section*{Resume}
%intro til kgs? 
Denne artikel udforsker hvorvidt spørgsmåls forskellige karakterisikaer har en påvirkelse på præcisionen af link prediction på temporal knowlegde grafs. De temporal knowlegde grafs der bliver undersøgt i denne artikel er ICEWS, WikiData og YAGO, og metoder brugt er DE-TransE, DE-SimplE, DE-Dismult, AtisE, TERO og TimePlex. Dette bliver gjordt ved først at opsette tre overordnede hypoteser, som undersøger disse karakterisikaer med et antal underhypoteser. Hypoteserne bliver herefter enten bevist eller modbevist baseret på resultater. Disse resultater bliver derefter brugt til at lave en ensemble voting metode, som bliver sammenlignet med en naiv ensemble voting metode som ikke bliver er baseret på resultaterne. 

Den første hypotese undersøger hvorvidt densitet af spørgsmål i tidsenheder har en påvirkning på de forskellige modellers ydeevne. Både samlet ydeevne og ydevne for tids forudsigelse bliver undersøgt her, og vi går ud fra at der er bedre ydevne når der er mange spørgsmål per tidsenhed, altså at der er høj densitet. 
Evalueringen af denne hyppotese begynder med at opdele vores dataset i tre dele baseret på deres densitet af spørgsmål i tidsenheder. De top 25\% er høj densitets sættet, De laveste 25\% er lav densitets sættet og de miderste 50\% bliver ikke brugt. Dette sikre os i at få stor forskel i densitet imellem høj og lav densitets sættet. Modellerne er herefter kørt på begge disse datasæt og forskellen i resultater mellem dem er udregnet. Det viste sig at på ICEWS og WikiData er der bedre ydevne på de lav densitets datasæt, men på YAGO er det en stor forbedring på det høj densitets sæt. Dette betyder at hyppotesen ikke er sand, men temporal densitet betyder mere på datasæt med stort tids interval. Den samme undersøgelse blev også gjordt for tids forudsigelse specifikt. Her kan man se en stor forbedring i ydeevne i høj densitets sættet i forhold til lav densitetes sættet. Dette leder til at under hyppotesen omkring hvorvidt tids forudsigelser er bedre på dataset med høj densitet på tid til at være sand. 

Den næste hypotese undersøger hvorvidt gennemsnits forudsigelsen for tid er bedre end hver model's egen forudsigelse. Dette kan gøres da tid er en kontinuerlig værdi og et gennemsnit kan faktisk findes, i modsætning til enheder eller relationer. Måden gennemsnits tiden bliver udregner er the mean average af tiderne modellerne forudsiger. På ICEWS har nogenlunde samme mean average error fra deres forudsagt tid til den rigtige tid. Her fik den gennemsnitlige tid et bedre resultat end alle andre metoder. På de to andre datasæt tre af metoderne har rigtig dårlige forudsigelser, så den gennemsnitlige forudsigelse bliver forværret pga. dem. Dette betyder at denne hyppotese er sand hvis alle modeller har circa samme ydeevne. 

Den sidste hypotese undersøger relationer egenskaber og hvordan det påvirker ydeeven af modeller. Egenskaberne undersøgt er symetriske relationer, som for eksempel "søskende", antisymetriske relationer, som for eksempel "mor til" og inverse relationer, som for eksempel paret "modtage gave" og "give gave". 
Den første underhyppotese DE-TRansE 

%ensemble learning

%ablation studies

%For at forbedre præcisionen af linkforudsigelse på Temporale Vidensgrafer (TKGs) er der blevet udviklet mange Vidensgrafindlejring (KGE) metoder, der gradvist forbedrer den overordnede kvalitet af forudsigelserne. Tidsstempelforudsigelser er især blevet undersøgt i højere detaljer i nyere forskning. I denne artikel undersøges styrkerne og svaghederne ved flere temporale KGE-metoder i forhold til de temporale egenskaber ved relationstyper og den temporale datatæthed, som bestemmes af strukturen af Vidensgrafen (KG). Dette arbejde bygger på tidligere arbejde, der analyserede kvaliteten af forudsigelser baseret på forudsigelsesmålet. Denne analyse fører til en ensemblelæringsmetode, hvor vægtene beregnes ud fra egenskaberne ved forespørgslen og præstationen af de involverede modeller. De resulterende regler, der bestemmer vægtene for hver model, illustrerer derefter, hvilke forespørgselsegenskaber der har størst betydning for forskellige metoder og generelt set. Derudover analyseres den relativt mindre udforskede domæne for tidsforudsigelser for deres samlede nøjagtighed og hvor velegnede de nuværende metoder er til brug i spørgsmål-svar (QA) systemer, og der præsenteres en strategi for tidsforudsigelser, der udnytter tidsoplysningernes kontinuerlige natur.
\begin{abstract}
%\glspl{kg} are graph representations of entites and their relations to eachother. Temporal information expands the information contained in a \gls{kg} with information about events and temporally contrained relations. \gls{qa} is a task that finds an answer to a query that \glspl{kg} are especially well suited for, but the performance of \gls{kg} \gls{qa} is limited, as all available \glspl{kg} are incomplete. To account for this, \glspl{kge} are used to infer missing relations in a \gls{kg} and increase the number of questions that can be answered with the \gls{kg}. 

To improve the accuracy of link prediction on \glspl{tkg}, many \gls{kge} methods have been developed, steadily improving the overall quality of predictions. Timestamp predictions in particular have been studied in higher detail in recent research. In this paper, the strengths and weaknesses of several temporal \gls{kge} methods are examined, in relation to the temporal properties of relation types and temporal data density as determined by the structure of the \gls{kg}. This work builds on previous work that analysed the quality of predictions based on the prediction target. This analysis leads to an ensemble learning method with weights calculated from the properties of the query and the performance of the involved models. The resulting rules that determine the weights of each model then illustrates which query properties have the highest importance for different methods, and overall. Additionally, the relatively less-explored domain of time predictions are analyzed for their overall accuracy and how suited current methods are for use in \gls{qa} systems, and a strategy for time predictions that makes use of the continuous nature of time information is presented.
\end{abstract}

% Keyword generator:
%\input{acm_files/acm_code.tex}

\maketitle
\newacronym{kg}{KG}{Knowledge Graph}%
\newacronym{tkg}{TKG}{Temporal Knowledge Graph}%
\newacronym{kge}{KGE}{Knowledge Graph Embedding}%
\newacronym{tkge}{TKGE}{Temporal Knowledge Graph Embedding}%
\newacronym{qa}{QA}{Question Answering}%
\newacronym{del}{DEL Graph}{Directed Edge-Labelled Graph}%
\newacronym{gnn}{GNN}{Graph Neural Network}%
\newacronym{rec-gnn}{Rec-GNN}{Recurrent Neural Network}%
\newacronym{mrr}{MRR}{Mean Reciprocal Rank}%
%\newacronym{void}{VoID}{Vocabulary of Interlinked Databases}%
%\newacronym{http}{HTTP}{HyperText Transfer Protocol}%
%\newacronym{rdf}{RDF}{Resource Description Framework}%
%\newacronym{uri}{URI}{Uniform Resource Identifier}%
%\newacronym{sparql}{SPARQL}{SPARQL Protocol and RDF Query Language}%
%\newacronym{html}{HTML}{HyperTExt Markup Language}%

\textbf{Code:} \url{https://github.com/cs-23-mi-10-01}
\\
\textbf{Date:} 1 June, 2023



% CONTENT
\section{Introduction}
\label{sec:introduction}

Questions can be divided into three types: Simple, multi-hop and complex.
Simple questions can be answered with a single triple \cite{bordes2015simple}, multi-hop questions require a path of triples \cite{zhang2017multihop}, and complex questions require logical reasoning over multiple triples \cite{talmorberant2018decomposition}.
We focus on simple questions as complex and multi-hop questions can be deconstructed to multiple simple questions. \cite{yani2021}

Most \gls{qa} systems consist of two parts: (1) Translating a question in natural language to a query (2) Resolving the query.
We do not consider the first problem as the aim of the survey is to investigate embedding methods and their qualities.
As simple questions can be expressed as a tuple, the task of simple \gls{qa} here corresponds to the task of link prediction, which enables us to select datasets developed for that task as well as datasets developed for \gls{qa}.

This paper is a continuation of [semester 9], where we found that the quality of the predictions highly depended on the target of the prediction. Highest performance was observed on tail predictions, followed by head, relation and time predictions. In this paper, this observation will be more thoroughly examined and tested with more datasets and train/test splits of the data.

In addition, it was observed that the best indicator for prediction quality seemed to be the relation type. In this paper, this concept will be expanded, by looking at the relations in the knowledge graph, categorizing them and evaluating the performance of the models over different categories of relations. The relations are categorized into structural features, which as symmetry, anti-symmetry, inversion, composition and heirarchy. Their temporal nature will also be examined, such as the average duration of events and relations.

Ultimately, the findings will contribute to an ensemble learning model with dynamic weights across different models, depending on the features of the query.
\section{Background and Notation}
\label{sec:background-and-notation}

Let $\varE$, $\varR$ and $\varT$ represent the set of all \textbf{entities}, \textbf{relations} and \textbf{timestamps} respectively. \textbf{Facts} are represented as $(h, r, t, \tau)$, where $h, t \in \varE$, $r \in \varR$ and $\tau \in \varT^2$. $\tau$ is a \textbf{timespan} consisting of two timestamps, where $\timebegin{\tau}$ refers to the first timestamp, and $\timeend{\tau}$ refers to the last. Let $\zeta$ represent all true facts in a world, and $\zeta'$ represent all false facts. Let $\varK$ be a \textbf{temporal knowledge graph}, which is a subset of $\zeta$. Let $\eta_r \subset \varK$ represent all facts in the knowledge graph where $r$ is the relation in the fact.

We define $(h, [r_1, \dots, r_n], t, \tau)$ to be a chain of relations from $h$ to $t$, such that $(h, r_1, e_1, \tau_1), (e_1, r_2, e_2, \tau_2), \dots, (e_{n-1}, r_n, t, \tau_n)$ where $\timebegin{\tau_1} \leq \timebegin{\tau_2} \leq \dots \leq \timebegin{\tau_n}$, and $\timebegin{\tau} = \timebegin{\tau_1}, \timeend{\tau} = \timeend{\tau_n}$.

A relation $r$ is \textbf{symmetric} if $(e_1, r, e_2, \tau) \in \zeta \Leftrightarrow (e_2, r, e_1, \tau) \in \zeta$ for some entities $e_1, e_2 \in \varE$ and timespans $\tau \in \varT^2$.
A relation $r$ is \textbf{anti-symmetric} if $(e_1, r, e_2, \tau) \in \zeta \Rightarrow (e_2, r, e_1, \tau) \in \zeta'$ for some entities $e_1, e_2 \in \varE$ and timespans $\tau \in \varT^2$.
A relation $r^{-1}$ is the \textbf{inversion} of relation $r$ if $(e_1, r, e_2, \tau) \in \zeta \Leftrightarrow (e_2, r^{-1}, e_1, \tau) \in \zeta$ for some entities $e_1, e_2 \in \varE$ and timespans $\tau \in \varT^2$.
A relation $r$ is the \textbf{composition} of relations $r_1, \dots r_n$ if $(e_1, [r_1, \dots, r_n], e_2, \tau) \in \zeta \Rightarrow (e_1, r, e_2, \tau) \in \zeta$ for some entities $e_1, e_2 \in \varE$ and timespans $\tau \in \varT^2$.
A relation $r$ is \textbf{reflexive} if $(e, r, e, \tau) \in \zeta$ for some entities $e \in \varE$ and timespans $\tau \in \varT^2$.

\section{Related Work}
\label{sec:related-work}

This paper is a continuation of \cite{P9}, where different embedding methods are analyzed through an exploration of the quality of results when the models make link predictions. The main observation made from that analysis is that the most influential contributing factor that influence the quality of the results is the relations and the structure of the \gls{kg} that surrounds them. This is the reasoning for what this paper concerns, and this paper serves to further examine the structure of the \gls{kg} surrounding the relations, increase reliability of the findings from the previous paper, and find other contributing factors that can influence the quality of the results.

As this paper is a direct continuation of \cite{P9}, the related work of that paper is also applicable here.
It covers the different categories of \gls{tkg} embedding methods, namely transformation, tensor decomposition, and neural network. 
Transformational methods use geometric functions to score facts in the embedding.
Tensor decomposition methods use tensor and eigenvector products to decompose tensors into low-dimentional representations that is used to score the facts.
Neural network methods use a number of learned layers to score facts based on the numerical representation of the elements of the facts.
We have been unable identify an influential, recent, and temporal neural network method, which we could replicate the results of and therefore neural network methods are not included in this study.
Influential transformational methods include RotatE \cite{sun2019rotate}, TeRo \cite{xu2020tero}, and ChronoR \cite{sadeighan2021chronor}. Influential tensor decomposition methods include TNTComplEx \cite{lacroix2020tcomplex}, TimePlex \cite{jain2020timeplex}, and ATiSE \cite{xu19atise}. 
Influential neural network methods include TFLEX \cite{lin22tflex} and Re-Net \cite{jin2019renet}.
A temporal model agnostic method takes an existing non-temporal embedding method and modifies it, such that temporal information is added to the embeddings.
Influential model agnostic methods include diachronic embeddings \cite{goel19diachronicemb} and time aware representations \cite{garcia-duran2018ta}.

Temporal accuracy metrics has previously been defined for scoring functions using a measure that compares absolute distances between elements of intervals \cite{surdeanu2013overviewtac}, and using bounding boxes to evaluate the intersections and hulls between intervals \cite{jain2020timeplex}. These metrics are mostly made for scoring functions and not statistical evaluation of time predictions, as they represent the scores as a measure that cannot be evaluated directly by humans, and it is difficult to interpret if the results are usable in a \gls{qa} context from these metrics alone.

Relation properties such as symmetry, antisymmetry, and inversion have been defined in non-temporal contexts \cite{schmidt10relationalmathematics}.
These properties define the structure of the \gls{kg} surrounding the relation, and some methods are incompatible with some of these properties \cite{chami2020atth, gregucci23sepa}.

The ensemble method ORB-E is based on a straightforward voting ensemble method \cite{MOHAMMED2023757}.
An ensemble method with similar construction to ORB-E is presented in \cite{otte2022towards}. It also has the domain of temporal \gls{kge}, but it does not consider query characteristics or dynamic weight distribution. 
%tilføj ensemble related work. måske christian og christian?

\section{Hypotheses}
\label{sec:hypotheses}

\newcounter{hcounter}
\addtocounter{hcounter}{1}
\subsection{Hypothesis \thehcounter} %stochastic variance of trainind data
\label{sec:hypothesis_\thehcounter}

\begin{quote}
\customlabel{hypothesis\thehcounter}{H\thehcounter}:
Prediction quality varies between models of same embedding methods trained on different train-test-validation splits of the same datasets.
\end{quote}

The purpose of this hypothesis is to evaluate the importance of stochastic variance in the training data. \missing[Daniele had an article about the topic, may be nice to ref here].

To test this we need multiple splits of a dataset and to ensure the data is not coincidental we need multiple datasets and multiple methods.
As training models is time consuming, it is best to limit the experiment to as few models as possible while fulfilling these requirements.
%Therefore we use three training splits on all three datasets and the methods \missing.
The methods examined in this hypothesis should be selected such that they are the least time consuming subset of all selected methods.
\addtocounter{hcounter}{1}
\subsection{Hypothesis \thehcounter} %type of prediction target
\label{sec:hypothesis_\thehcounter}

\begin{quote}
\customlabel{hypothesis\thehcounter}{H\thehcounter}: 
%The quality of a prediction is influenced by which fact element is being predicted, and varies between different datasets and splits of datasets.
Prediction quality of queries where the prediction target is one type of fact element varies between models of same embedding methods trained on different datasets.
\end{quote}

The purpose of this hypothesis is to examine a suggestion from our earlier research, that it holds for all methods that prediction quality is ordered $t>h>r>\tau$ when the prediction target is a fact element of that type.

To test this we need multiple models of an embedding method trained on different datasets, preferably for multiple embedding methods.
As it is trivial to examine prediction quality dependant on prediction target type once a model has trained this hypothesis is evaluated across all models trained for this research.
\addtocounter{hcounter}{1}
\subsection{Hypothesis \thehcounter} %time granularity
\label{sec:hypothesis_\thehcounter}

\begin{quote}
\customlabel{hypothesis\thehcounter}{H\thehcounter}:
Prediction quality of queries where the prediction target $\tau$ is improved by considering precision granularity of predictions.
\end{quote}

The purpose of this hypothesis is to examine how far wrong predictions on $\tau$ are from the right results. 
In contrast to the remaining fact elements, $\tau$ is a continuous rather than discrete value and as such it is possible to measure degrees of precision.

To test this we need to measure the distance between predicted and correct values.
\missing[This is done both by considering the amount of correct predictions on different levels of time granularity and by calculating the distance between predicted and correct answers.]
\addtocounter{hcounter}{1}
\subsection{Hypothesis \thehcounter} %embedding method category
\label{sec:hypothesis_\thehcounter}

\begin{quote}
\customlabel{hypothesis\thehcounter}{H\thehcounter}:
Prediction quality varies between models of different embedding methods with specific categories.
\end{quote}

The purpose of this hypothesis is to examine importance and characteristics of different embedding method categories.

To test this we need multiple methods of each category.

\newcounter{hsubcounter}\addtocounter{hsubcounter}{1}

\begin{quote}
\customlabel{hypothesis\thehcounter\alph{hsubcounter}}
{H\thehcounter (\alph{hsubcounter})}:
Prediction quality of queries where the prediction target is $r$ or $\tau$ is better for models of embedding methods with the transformation category.
\end{quote}

The purpose of this sub-hypothesis is to examine a suggestion from our earlier research.
\addtocounter{hcounter}{1}
\subsection{Hypothesis \thehcounter} %time representation
\label{sec:hypothesis_\thehcounter}

\begin{quote}
\customlabel{hypothesis\thehcounter}{H\thehcounter}:
Prediction quality varies between models of different embedding methods with independent timestamp embeddings and time-dependent augmentations of existing embeddings.
\end{quote}
\addtocounter{hcounter}{1}
\subsection{Hypothesis \thehcounter} %relation properties
\label{sec:hypothesis_\thehcounter}

\begin{quote}
\customlabel{hypothesis\thehcounter}{H\thehcounter}:
%Prediction quality varies between models over different types of relations that are involved in the prediction task.
Prediction quality of queries where the relation has specific properties varies between models of different embedding methods.
\newline\customlabel{hypothesis5a}{H5 (a)}: Prediction quality of queries where the relation has the symmetry or anti-symmetry properties varies between models of different embedding methods with joint and separate entity embeddings.
\newline\customlabel{hypothesis5b}{H5 (b)}: When the relation has the inversion property.
\newline\customlabel{hypothesis5c}{H5 (c)}: When the relation has the composition property.
\newline\customlabel{hypothesis5d}{H5 (d)}: Prediction quality of queries where the relation has the hierarchy property is better for models of embeddings methods within hyperbolic space.
\end{quote}

To incorporate this in the ensemble learning method, each relation gets classified with a soft label for each relation type, using a function that maps each relation to a real number that describes how many relations fulfill the requirements for that relation type. The reasoning for this is that knowledge graphs are not complete, and some relations might be missing from the graph.

The soft label for symmetry for relation $r$ is

\begin{equation}
\begin{gathered}
\mathit{sym}(r) = \frac{|S|}{|\eta_r|}\\
S = \{ (e_1, r, e_2, \tau) \mid (e_1, r, e_2, \tau) \in \eta_r \wedge (e_2, r, e_1, \tau) \in \eta_r \}
\end{gathered}
\end{equation}

\noindent
The soft label for anti-symmetry for relation $r$ is

\begin{equation}
\begin{gathered}
\mathit{asym}(r) = \frac{|A|}{|\eta_r|}\\
A = \{ (e_1, r, e_2, \tau) \mid (e_1, r, e_2, \tau) \in \eta_r \wedge (e_2, r, e_1, \tau) \notin \eta_r \}
\end{gathered}
\end{equation}

\noindent
The soft label for inversion for relation $r$ is

\begin{equation}
\begin{gathered}
\mathit{inv}(r) = \varmax_{r^i \in \varR \setminus \{r\} } \frac{|I_{r^i}|}{|\eta_r|}\\
I_{r^i} = \{ (e_2, r^i, e_1, \tau) \mid (e_1, r, e_2, \tau) \in \eta_r \wedge (e_2, r^i, e_1, \tau) \in \eta_{r^i} \}
\end{gathered}
\end{equation}

\noindent
The soft label for reflexitivity for relation $r$ is

\begin{equation}
\begin{gathered}
\mathit{ref}(r) = \frac{|R|}{|\eta_r|}\\
R = \{ (e, r, e, \tau) \mid (e, r, e, \tau) \in \eta_r \}
\end{gathered}
\end{equation}

The average score for each model is then calculated for all relations that align to each of these labels, and used to calculate the most probable ranking of predictions over those relation types. The average rank of symmetrical relations on model $m$ is

%\begin{equation}
%\mathit{avg\_sym}(m) = \frac{ \varsum_{r \in \varR} \left( |\eta_r| * \mathit{sym}(r) * \frac{\varsum_{q \in \eta_r} \mathit{rank}(q , m)}{|\eta_r|} \right) }{ \varsum_{r \in \varR} |\eta_r| * \mathit{sym}(r) }
%\end{equation}

\begin{equation}
\mathit{avg\_sym}(m) = 
\frac{ \varsum_{r \in \varR} \left( \mathit{sym}(r) * \varsum_{q \in \eta_r} \frac{1}{\mathit{rank}(q , m)} \right) }
{ \varsum_{r \in \varR} |\eta_r| * \mathit{sym}(r) }
\end{equation}

\noindent
Similarly, $\mathit{avg\_asym}(m)$, $\mathit{avg\_inv}(m)$ and $\mathit{avg\_ref}(m)$ is defined, replacing $\mathit{sym}(r)$ with $\mathit{asym}(r)$, $\mathit{inv}(r)$ and $\mathit{ref}(r)$ respectively.

When a query $(h, r, t, \tau)$ is passed to the ensemble model, a weight vector $v_r \in \R^{|M|}$ is calculated for each model $m_i \in M$ from the relation $r$. This vector is defined with

\begin{equation}
v_r = \varconcat_{i = 1}^{|M|} \, \left(
\begin{aligned}
&\mathit{sym}(r) * \mathit{avg\_sym}(m_i) \, + \\
&\mathit{asym}(r) * \mathit{avg\_asym}(m_i) \, + \\
&\mathit{inv}(r) * \mathit{avg\_inv}(m_i) \, + \\
&\mathit{ref}(r) * \mathit{avg\_re}f(m_i)
\end{aligned} \right)
\end{equation}

\noindent
where $\doublepipe$ denotes concatenation between numbers, to create a vector.
\addtocounter{hcounter}{1}
\subsection{Hypothesis \thehcounter} %time density
\label{sec:hypothesis_\thehcounter}

\begin{quote}
\customlabel{hypothesis\thehcounter}{H\thehcounter}:
Prediction quality of queries where $\tau$ is in a temporally dense subset of the training dataset is better.
\end{quote}

\begin{comment}
\addtocounter{hcounter}{1}
\subsection{Hypothesis \thehcounter}
\label{sec:hypothesis_\thehcounter}

\begin{quote}
\customlabel{hypothesis\thehcounter}{H\thehcounter}:
Prediction quality [of queries where xxx] varies between models of same/different embedding methods [with SomeCharacteristic SomeValue] [trained on same/different datasets].
\end{quote}
\end{comment}
\section{Methods}
\label{sec:methods}

The selection of methods has been expanded from \cite{P9} and consists of DE-TransE, DE-DistMult, DE-Simple, ATiSE, TeRo, and TimePlex. 
In \autoref{tab:overview_of_models}, an overview of what each method is designed for is provided. Methods have been selected such that they are diverse in the relation properties that they are theoretically capable of modelling and overall approach.
In this section, a general outline of the method approaches are given. 

\begin{table}[htb]
\centering
\begin{minipage}{\columnwidthcaption}
\centering
\caption{Overview of method and characteristics. T: Transformation, TD: Tensor decomposition.}
\label{tab:overview_of_models}
\end{minipage}

\vspace{-3mm}
%DE-TransE, DE-DistMult, DE-SimplE: \cite{goel19diachronicemb} ATiSE: \cite{xu19atise} TeRo: \cite{xu2020tero} TimePlex: \cite{jain2020timeplex}
\resizebox{\columnwidth}{!}{
\begin{tabular}{r|cccc}
\hline
Method & Category & {Symmetry} & {Anti-Symmetry} & {Inversion} \\ \hline
DE-TransE & T & $\times$ & \checkmark & \checkmark \\
DE-DistMult & TD & \checkmark & $\times$ & $\times$ \\
DE-SimplE & TD & \checkmark & \checkmark & \checkmark \\
ATiSE & TD & \checkmark & \checkmark & \checkmark \\
TeRo & T & \checkmark & \checkmark & \checkmark \\
TimePlex & TD & \checkmark & \checkmark & \checkmark \\ \hline
\end{tabular}
}
% \begin{tblr}{colspec={r|cccc},rowspec={Q[b]QQQQQ}}
% \hline
% Model & Category & {Symmetry} & {Anti-\\Symmetry} & {Inversion} \\ \hline
% DE-TransE & T & $\times$ & \checkmark & \checkmark \\
% DE-DistMult & TD & \checkmark & $\times$ & $\times$ \\
% DE-SimplE & TD & \checkmark & \checkmark & \checkmark \\
% ATiSE & TD & \checkmark & \checkmark & \checkmark \\
% TeRo & T & \checkmark & \checkmark & \checkmark \\
% TimePlex & TD & \checkmark & \checkmark & \checkmark \\ \hline
% \end{tblr}

\end{table}
DE is a model-agnostic method, meaning it adapts existing non-temporal methods to the use of temporal information. Here we use it on the methods TransE, DistMult and SimplE.

DE-TransE is based on the TransE \cite{bordes2013transe} method, a method used as a benchmark for many embedding approaches. 
TransE embeds entities and relations as vectors in Euclidian space, 
models relations as a translation between head and tail entity and scores the facts by Euclidian distance calculation.
TransE is unable to model symmetry as that would require embedding both entities with the same vector and the symmetric relation to be a 0 distance vector. This would make it impossible to differentiate between the entities in all other relations.

DE-DistMult is based on the DistMult \cite{yang2015distmult} method, a method that like TransE uses a single vector to embed each relation and entity. Unlike TransE it scores facts as summations of element wise products of the embedding vectors.
As the ordering of elements in the element-wise product does not affect the result, DistMult cannot model edge direction, which in turn means that it cannot model anti-symmetry or inversion.

DE-SimplE is based on SimplE \cite{kazemi2018simple} which in turn is based on Canonical Polyadic \cite{hitchcock1927cp} embedding. 
SimplE embeds relations and entities using two vectors each. Relation embeddings have a vector $r$ for forward relations and a vector $r^{-1}$ for reverse relations. Entity embeddings have a vector $e_h$ for head entities, and a vector $e_t$ for tail entities.
The score function is then defined as the average of the element wise product of the embeddings of the elements in two facts, namely $(e_h, r, e_t)$ and $(e_t, r^{-1}, e_h)$.

Methods modified with DE \cite{goel19diachronicemb} are extended such that the base method includes temporal information by making the entity embeddings dependent on time.
Each model has a part of their entity embedding vector values that are impacted by timestamp.
DE adds two learnable embedding vectors for each entity, one for dependency on time, and one for a bias.
The resulting entity embedding values are found by multiplying the time-dependent vector values with the timestamp, adding them to a bias value, using an activation function on them, and then multiplying them with the non time-dependent embedding. 
The part of embedding vector values that are not dependent on the timestamp are simply the unaltered original embedding vector values.

ATiSE \cite{xu19atise} is a method that uses additive time series decomposition to model changes and uncertainty in the data over time. The additive time series is a combination of a linear function, a sine function and a random function, to describe trends, seasonal changes and noise respectively. Each entity and relation has a seperate time series to represent how those elements change over time, following the patterns that those elements demonstrate. Entities and relations are embedded as vectors, and they follow Gaussian probability distributions. The score function considers the difference between the probability distribution of entities in head and tail positions of a fact, and their similarity to the probability distribution of the relation of the fact at a certain time. To compare the probability distributions, ATiSE uses Kullback–Leibler divergence.

TeRo \cite{xu2020tero} embeds entities and relations as vectors in complex space and uses element-wise rotations in complex space to model entities at specific timestamps.
The scoring function is defined as the distance between the time-specific head entity plus the relation, and the conjugate of the time-specific tail entity.

TimePlex \cite{jain2020timeplex} is a model that embeds entities and timestamps in complex vector space, and each relation with three vectors in complex space. The first relation embedding vector represents a relation that is true for a head entity at a specific timestamp, the second relation embedding represents a relation that is true for a head entity and a tail entity, the third relation represents a relation that is true for a tail entity and a specific timestamp. The score function uses element-wise products between the head, relations, tail and timestamps, the latter embeddings in the products being the conjugate of those vectors. In addition, the score function also includes two additional scores, one of them consider recurrences of events, the other considers time contraints between pairs of relations.
\section{Datasets}
\label{sec:datasets}
When selecting datasets for model evaluation we require the data to be at least partly temporal. Prevalence in related work is also valued highly, as it enables us to compare our implementation with the original implementation when both are evaluated over the same data.

\begin{table}[htb]
\centering
\begin{minipage}{\columnwidthcaption}
\centering
\caption{Statistics of datasets}
\label{tab:dataset_stats}
\end{minipage}

\vspace{-3mm}
\begin{tabular}{r|M{0.22}M{0.22}M{0.21}} \hline
Dataset & \mbox{ICEWS14} & WikiData12k & YAGO11k \\
\hline
\# Facts & 96730 & 40621 & 20507\\
\# Entities & 7128 & 12554 & 10623\\
\# Relations & 230 & 24 & 10\\
Time Period & 2014 & 19--2020 & -431--2844\\
\hline
\end{tabular}
\end{table}

The selected graphs are ICEWS \cite{boschee2015ICEWS}, WikiData \cite{vrandecic2014wikidata}, and YAGO \cite{mahdisoltani2015YAGO3, tahon2020YAGO4}. 
ICEWS is a number of fully temporal, event-based \glspl{kg} specific to the domain of crisis alerts, where each graph contains facts from a single year.
Time is represented in the form of a singular timestamp and the granularity of time steps is one day.
ICEWS is particularly noted for its consistency and uniform temporal distribution of data. 
WikiData and YAGO are partly temporal, general knowledge \glspl{kg} and are not constrained to any certain time span.
Both represent time as a timespan with a beginning and an end and their granularity of time steps is one year.
WikiData is particularly noted for its size and YAGO for its focus on temporal data and ability to represent uncertainty in timestamps.

Specifically we use the datasets \mbox{ICEWS14}
%\footnote{\mbox{ICEWS14-7k} is typically referred to as simply ICEWS14. However, so is \mbox{ICEWS14-12K} \cite{trivedi2017knowevolve}. We append the entity count to the name in order to distinguish between the two datasets.} 
\footnote{ICEWS14 exists in at least two versions: One with 7128 entities \cite{garcia-duran2018ta} and one with 12498 \cite{trivedi2017knowevolve}. We use the first version.}
, WikiData12K \cite{dasgupta2018hyte}, and YAGO11K \cite{dasgupta2018hyte}, henceforth referred to simply as ICEWS, WikiData and YAGO. Statistics of the datasets can be found in \autoref{tab:dataset_stats}. Date and month information has been removed from WikiData and YAGO for the sake of uniformity.
\section{Preliminary Experiments}
\label{sec:preliminary_experiments}

As a preliminary step, the overall \gls{mrr} scores of each model on every dataset has been evaluated, to determine the general performace of the models. These results are presented in \autoref{tab:overall_results}. 

\begin{table}[htb]
\centering
\begin{minipage}{\columnwidthcaption}
\centering
\caption{Overall \gls{mrr} results of the methods on the given datasets, with the original dataset splits.}
\vspace{-3mm}

\begin{tabular}{r|SSS}
\hline
Method & {ICEWS} & {WikiData} & {YAGO}\\
\hline
DE-TransE   & 0.23 & 0.34 & 0.29 \\
DE-DistMult & 0.30 & 0.31 & 0.23 \\
DE-SimplE   & 0.33 & 0.32 & 0.25 \\
TeRo        & 0.38 & 0.43 & 0.30 \\
ATiSE       & 0.35 & 0.40 & 0.31 \\
TimePlex    & 0.24 & 0.24 & 0.15 \\
\hline

\end{tabular}

\label{tab:overall_results}
\end{minipage}
\end{table}


Additionally, some of the models have been evaluated using different datasets and different splits of the dataset, to support the reliability of the observations from our previous work \cite{P9}. We discovered that for models trained on ICEWS, the best results are achieved when the prediction target is the tail, followed by the head, relation, and then timestamp. To investigate whether this is a coincidence, we created three new splits of ICEWS, trained DE-TransE, DE-SimplE, ATiSE and TeRo on those splits, and checked the quality of predictions of these models on the new splits. The same pattern emerged, confirming that the predictions on ICEWS follows this expected result on prediction targets.

The same experiment was conducted using the datasets WikiData and YAGO. These datasets and new splits all exhibited a different pattern from ICEWS for prediction targets. In both new datasets and all their splits, the relation prediction is most accurate, followed by the tail, head and then time predictions.
This is likely because Wikidata and YAGO have considerably fewer relation types than ICEWS, and it is therefore less likely to be incorrect. The results indicate that the \textbf{performance is generally best when making tail predictions, followed by head, relation and then time}, however if some fact element has a considerably lower amount of possible answers, that might increase the expected performance for predictions on that fact element.

The overall performance, as well as the performance when predicting on specific fact elements have been illustrated for each model and each dataset in \autoref{app:dataset_split_comparisons}.

%\begin{table}[htb]
\centering
\begin{minipage}{\columnwidthcaption}
\centering
\caption{Overall \gls{mrr} results of evaluations on the given methods, on the testsets of alternative splits $T_1$, $T_2$ and $T_3$.}
\label{tab:split_results}
\end{minipage}
\vspace{-3mm}

\begin{tabular}{r|ccc|ccc|ccc}
\hline
 & \multicolumn{3}{c|}{ICEWS14-7k} & \multicolumn{3}{c|}{WikiData12k} & \multicolumn{3}{c}{YAGO11k}\\
Method & $T_1$ & $T_2$ & $T_3$ & $T_1$ & $T_2$ & $T_3$ & $T_1$ & $T_2$ & $T_3$ \\
\hline
DE-TransE  & 0.23 & 0.23 & 0.23 & 0.34 & 0.34 & 0.34 & 0.27 & 0.27 & 0.27 \\
DE-SimplE  & 0.34 & 0.34 & 0.34 & 0.30 & 0.30 & 0.30 & 0.21 & 0.21 & 0.21 \\
TeRo  & 0.40 & 0.39 & 0.39 & 0.72 & 0.71 & 0.41 & 0.28 & 0.24 & 0.28 \\
ATiSE & 0.30 & 0.30 & 0.30 & 0.64 & 0.64 & 0.39 & 0.29 & 0.28 & 0.28 \\
\hline

\end{tabular}
\end{table}



\section{Hypothesis Evaluation}
\label{sec:hypothesis_evaluation}

This section contains the evaluation of hypotheses presented in \autoref{sec:hypotheses}.

\subsection{Time Density}
\label{sec:time_density_experiment}

\begin{table}[htb]
\centering
\begin{minipage}{0.95\columnwidth}
\centering
\caption{Time density comparison}
\vspace{-3mm}

\begin{tabular}{l|cc}\hline
Dataset  & $|T_D|$ & $|T_P|$ \\ \hline
ICEWS14  & 7422    & 6687    \\
WIKIDATA & 4248    & 3924    \\
YAGO     & 2068    & 2000    \\\hline
\end{tabular}

\label{tab:time_density_testset_stats}
\end{minipage}
\end{table}


\begin{table}[htb]
\centering
\begin{minipage}{0.95\columnwidth}
\centering
\caption{Time density comparison}
\vspace{-3mm}

\begin{tabular}{l|cc|cc|cc}\hline
 & \multicolumn{2}{c|}{ICEWS14} & \multicolumn{2}{c|}{WIKIDATA}& \multicolumn{2}{c}{YAGO} \\
Model & $T_D$ & $T_P$ & $T_D$ & $T_P$ & $T_D$ & $T_P$ \\ \hline
DE-T   & 
0.27    & 0.28   &
0.43    & 0.47   &
0.39    & 0.39   \\
DE-D &
0.37    & 0.38   &
0.40    & 0.43   &
0.34    & 0.28   \\
DE-S   &
0.41    & 0.43   &
0.41    & 0.45   &
0.36    & 0.29   \\
TeRo        & 
0.44    & 0.45   &
0.46    & 0.53   &
0.34    & 0.28   \\
ATiSE       & 
0.42    & 0.43   &
0.45    & 0.52   &
0.41    & 0.31   \\
TimePlex    &
0.35    & 0.38   &
0.22    & 0.29   &
0.15    & 0.17   \\ \hline
\end{tabular}

\label{fig:time_density_comparison}
\end{minipage}
\end{table}


\begin{table}[htb]
\centering
\begin{minipage}{\columnwidthcaption}
\centering
\caption{Difference in MRR scores of models when predicting timestamp compared to other targets, on dense testset $T_D$ and sparse testset $T_P$ on ICEWS14.}
\label{tab:time_density_timestamp_icews14}
\end{minipage}
\vspace{-3mm}

\begin{tabular}{cc|cccccc} \hline
\multicolumn{2}{c|}{ICEWS14} & & & & & & \\
Test set & Target & DE-T & DE-D & DE-S & TeRo & ATiSE & TimePlex \\ \hline 
$T_D$ & $h$, $r$, $t$ & 
0.27 & 0.37 & 0.41 & 0.44 & 0.42 & 0.35 \\
$T_D$ & $\tau$        & 
0.09 & 0.08 & 0.08 & 0.18 & 0.16 & 0.02 \\ \hline
\multicolumn{2}{c|}{Difference} & 
-0.18 & -0.29 & -0.33 & -0.26 & -0.26 & -0.33 \\ \hline\hline
$T_P$ & $h$, $r$, $t$ & 
0.28 & 0.40 & 0.44 & 0.46 & 0.44 & 0.38 \\
$T_P$ & $\tau$        & 
0.10 & 0.09 & 0.09 & 0.17 & 0.15 & 0.02 \\ \hline 
\multicolumn{2}{c|}{Difference} & 
-0.18 & -0.31 & -0.35 & -0.29 & -0.29 & -0.36 \\ \hline
\end{tabular}

\end{table}


\begin{table}[htb]
\centering
\begin{minipage}{\columnwidthcaption}
\centering
\caption{Difference in MRR scores of models when predicting timestamp compared to other targets, on dense testset $T_D$ and sparse testset $T_P$ on WIKIDATA.}
\label{tab:time_density_timestamp_wikidata}
\end{minipage}
\vspace{-3mm}

\begin{tabular}{cc|cccccc} \hline
\multicolumn{2}{c|}{WIKIDATA} & & & & & & \\
Test set & Target & 
DE-T & DE-D & DE-S & TeRo & ATiSE & TimePlex \\ \hline 
$T_D$ & $h$, $r$, $t$ & 
0.43 & 0.40 & 0.42 & 0.46 & 0.45 & 0.21 \\
$T_D$ & $\tau$        & 
0.00 & 0.00 & 0.00 & 0.29 & 0.24 & 0.25 \\ \hline
\multicolumn{2}{c|}{Difference} & 
-0.43 & -0.40 & -0.42 & -0.14 & -0.21 & +0.04 \\ \hline\hline
$T_P$ & $h$, $r$, $t$ & 
0.50 & 0.47 & 0.48 & 0.53 & 0.52 & 0.36 \\
$T_P$ & $\tau$        & 
0.01 & 0.00 & 0.00 & 0.26 & 0.19 & 0.17 \\ \hline 
\multicolumn{2}{c|}{Difference} & 
-0.49 & -0.47 & -0.48 & -0.27 & -0.33 & -0.19 \\ \hline
\end{tabular}

\end{table}


\begin{table}[htb]
\centering
\begin{minipage}{\columnwidthcaption}
\centering
\caption{Difference in MRR scores of models when predicting timestamp compared to other targets, on dense testset $T_D$ and sparse testset $T_P$ on YAGO.}
\label{tab:time_density_timestamp_yago}
\end{minipage}
\vspace{-3mm}

\begin{tabular}{cc|cccccc} \hline
\multicolumn{2}{c|}{YAGO} & & & & & & \\
Test set & Target & DE-T & DE-D & DE-S & TeRo & ATiSE & TimePlex \\ \hline 
$T_D$ & $h$, $r$, $t$ & 
0.39 & 0.34 & 0.36 & 0.34 & 0.41 & 0.17 \\
$T_D$ & $\tau$        & 
0.00 & 0.00 & 0.00 & 0.25 & 0.16 & 0.12 \\ \hline
\multicolumn{2}{c|}{Difference} & 
-0.39 & -0.34 & -0.36 & -0.09 & -0.25 & -0.05 \\ \hline\hline
$T_P$ & $h$, $r$, $t$ & 
0.38 & 0.26 & 0.26 & 0.28 & 0.31 & 0.23 \\
$T_P$ & $\tau$        & 
0.01 & 0.00 & 0.01 & 0.18 & 0.08 & 0.05 \\ \hline 
\multicolumn{2}{c|}{Difference} & 
-0.37 & -0.26 & -0.25 & -0.10 & -0.23 & -0.18 \\ \hline
\end{tabular}

\end{table}



The evaluation of hypothesis \autoref{hyp:time_density} includes evaluating the quality of prediction of the models, with testsets split into a dense part $T_D$ and a sparse part $T_S$. These two partitions are approximately same size as can be seen in \autoref{tab:time_density_testset_stats}, but the sparse dataset is spread over a larger timespan than the dense partition. The MRR score results can be seen in \autoref{tab:time_density_comparison}. According to the hypothesis, we expect the results to be more accurate in the dense partition than in the sparse partition.

The results from ICEWS14 and WIKIDATA is that the results are better on the sparse partitions than on the dense partitions, while the results from YAGO generally yield a higher score on the dense dataset.

The evaulation of hypothesis \autoref{hyp:time_density_timestamp_dense} and \autoref{hyp:time_density_timestamp_sparse} involve finding the general trend among time predictions versus all other kinds of predictions. The results are presented in \autoref{tab:time_density_timestamp_icews14}, \autoref{tab:time_density_timestamp_wikidata}, and \autoref{tab:time_density_timestamp_yago}.
As the results indicate, timestamps are generally more challenging for the models to predict, and the overall results are lower in time predictions than predictions over other elements.
The mean average difference in the dense test set and the sparse test set in ICEWS14 is $-0.27$ and $-0.30$ respectively, indicating that the dense dataset performs better on timestamp predictions, but not by a significant amount.
In WIKIDATA this mean average difference is $-0.26$ for the dense test set and $-0.37$ for the sparse, indicating that the dense test set performs significantly better on time predictions than the sparse test set.
In YAGO, the mean average differences are $-0.25$ and $-0.23$, indicating that the sparse test set performs better than the dense test set for time predictions in this dataset, but significantly.

While this result implies some connection between the density of the data and the prediction quality, it mostly shows that prediction quality is not primarily affected by the density of the data. Instead, other parameters might influence the results such as the quality and completeness of the data in the sparse and dense parts of the datasets. If the sparse parts are less complete, it means that only the most influential and important relations are present in those parts of the dataset, and this might be easier for the models to make link predictions on. Alternatively, as different relation types become more represented over time while others become less represented, the periods might be easier or harder for the models to model, and some relations are easier than others to model.

Overall, with this result, we cannot draw any definitive conclusions on hypothesis \autoref{hyp:time_density}.

\subsection{Joint Timestamp Selection}
\label{seubsec:timestamp_voting_experiment}

Hypothesis \autoref{hyp:timestamp_voting} concerns making time predictions by using the average of the best prediction of several models.

\begin{table}[htb]
\centering
\begin{minipage}{\columnwidthcaption}
\centering
\caption{Average precision of top timestamp predictions of each model, and the precision of the timestamp result of the models selecting a result. As TeRo and ATiSE find a range of possible timestamps for WikiData and YAGO, their closest and furthest timestamps in their ranges are noted.}
\vspace{-3mm}

\begin{tabular}{r|ccc}\hline
 & ICEWS & WikiData & YAGO \\
Model & (days) & (years) & (years) \\ \hline
DE-TransE & 90.51 & 1003.50 & 863.69 \\ 
DE-DistMult & 87.90 & 978.24 & 791.51 \\ 
DE-SimplE & 93.66 & 966.97 & 814.93 \\ 
TeRo & 129.75 & 76.93–102.53 & 358.14–642.29 \\
ATiSE & 137.39 & 85.39–153.25 & 94.57–579.43 \\ 
TimePlex & 122.87 & \textbf{21.90} & \textbf{148.27}\\
\hline
Selection & \textbf{84.71} & 513.83 & 389.76 \\ \hline 

\end{tabular}

\label{tab:timestamp_voting_table}
\end{minipage}
\end{table}



To evaulate this, the models are compared using the \gls{mae} score achieved when making time predictions. The \gls{mae} is in days for ICEWS, and years for WikiData and YAGO. The scores are also compared to the \gls{mae} score of predictions achieved when all models jointly select a timestamp answer. The results of this evaluation can be seen in \autoref{tab:timestamp_voting_table}. Error distributions of the individual methods can be found in \autoref{time_prediction_error_distribution}.

When TeRo and ATiSE make time predictions, they attempt to predict both the beginning and end timestamp of the query, resulting in a time interval prediction. When contributing to the joint timestamp selection, a timestamp in the middle of the time interval is used instead.

The results show that the \gls{mae} score is lower in predictions on ICEWS when jointly selecting the answer timestamp.
An average error of 90+ days is very high in ICEWS, as this dataset only spans a year. This means the average error range is 6 months, which is half of the dataset.
For TeRo, ATiSE and TimePlex, the result indicates that the predicted timestamp is random, which is supported by their error distribution in \autoref{time_prediction_error_distribution}. Joint timestamp selection seems to average out wrong answers in both directions for each method, and thereby achieves a higher score, supporting \autoref{hyp:timestamp_voting}. 

On WikiData TimePlex has the highest precision, and the difference between the \gls{mae} of the most precise model and the least precise model DE-TransE is very high at \textasciitilde1000 years. The results on WikiData indicate that this dataset is difficult for the DE models to make time predictions on, and the \gls{mae} scores of these three models seem to indicate that the predicted answer is random, which is again supported by the error distribution in \autoref{time_prediction_error_distribution}. Joint selection achieves a score that is in the middle between the two extremes, and as it is decided using the mean average of the time predictions of all contributing models, the DE models make the results significantly worse. 

For YAGO the same pattern as WikiData emerges. However TimePlex is significantly worse on this dataset and ATiSE has the best score when comparing the closest timestamp in its time span. The joint time prediction is once again impaired by the worst models. 

The results show that the overall error of time predictions is high. The best \gls{mae} result on \mbox{ICEWS} is $84.71$ days, which is an unacceptable average error in a \gls{qa} context, as the purpose of queries on ICEWS is to make early predictions on critical events like military operations or civilian unrest. These need to be highly accurate to be useful. Similarly, having an average error of $94+$ years on YAGO is unacceptable, as queries like birth dates and war periods need to be somewhat precise to be useful. On the other hand, TimePlex achieves an \gls{mae} score of $21.90$, which is accurate enough to be useful in some contexts that do not require high accuracy, e.g. when asking about when technological ages like the industrial age began and ended. TimePlex is the only examined model that attempts to optimize time predictions, and as such it is encouraging that it achieves the only useful result, but it still only achieves it on one of the three examined datasets.

Overall, this indicates that \textbf{\autoref{hyp:timestamp_voting} is true} if the base models have approximately equal precision.
A more complex voting mechanism when jointly selecting timestamps might yield better results, such as giving less weight to less accurate models, or using more than just the top scoring prediction for each model.
\subsection{Relation Properties}
\label{sec:relation_properties_experiment}

\begin{table}[htb]
\centering
\begin{minipage}{0.95\columnwidth}
\centering
\caption{Test sets, and what they contain}
\vspace{-3mm}

\begin{tabular}{r|l}\hline
Test set & Contains \\ \hline
$T_D$ & Dense partiton timestamps \\ 
$T_P$ & Sparse partition timestamps \\
$T_S$ & Symmetrical relations \\ 
$T_S'$ & Non-symmetrical relations \\ 
$T_A$ & Anti-symmetrical relations \\ 
$T_A'$ & Non-antisymmetrical relations \\ 
$T_I$ & Inverse relations \\ 
$T_I'$ & Non-inverse relations \\ \hline
%$T_R$ & Reflexive relations \\ 
%$T_R'$ & Non-reflexive relations \\
\end{tabular}

\label{tab:test_set_explanations}
\end{minipage}
\end{table}


\begin{table}[htb]
\centering
\begin{minipage}{\columnwidthcaption}
\centering
\caption{Number of facts in each relation property test set}
\end{minipage}
\vspace{-3mm}

\resizebox{\columnwidth}{!}{
\begin{tabular}{lc|cc|cc|cc|cc}\hline
Dataset & $|\varR|$ & $|T_S'|$ & $|T_S|$ & $|T_A'|$ & $|T_A|$ & $|T_I'|$ & $|T_I|$ & $|T_R'|$ & $|T_R|$ \\ \hline
ICEWS14 & 220 & 22902 & 3987 & 23253 & 3636 & 25581 & 1308 & 26889 & 0 \\
WIKIDATA & 24 & 16248 & 0 & 580 & 15668 & 16248 & 0 & 16248 & 0 \\
YAGO & 10 & 7500 & 704 & 704 & 7500 & 8204 & 0 & 8204 & 0 \\
 \hline
\end{tabular}
}

\label{tab:relation_property_test_sets}
\end{table}


%\input{content/hypothesis_evaluation/tables/relation_properties_folder/icews14_original_timestamps}
%\input{content/hypothesis_evaluation/tables/relation_properties_folder/wikidata12k_original_timestamps}
%\input{content/hypothesis_evaluation/tables/relation_properties_folder/yago11k_original_timestamps}
\begin{table}[htb]
\centering
\begin{minipage}{\columnwidthcaption}
\centering
\caption{Hypothesis \autoref{hyp:relation_property_sym} comparison of DE-TransE with DE-DistMult and DE-Simple on symmetric relation testsets $T_S$ and non-symmeric relation testsets $T_S'$.}
\vspace{-3mm}

\begin{tabular}{r|ccc}\hline
ICEWS14 & DE-T & DE-D & DE-S \\ \hline
$T_S'$ & 0.24 & 0.33 & 0.34 \\
$T_S$ & 0.26 & 0.48 & 0.48 \\ \hline
Difference & +0.02 & +0.15 & +0.14 \\ \hline\hline
YAGO & DE-T & DE-D & DE-S \\ \hline
$T_S'$ & 0.06 & 0.03 & 0.04 \\
$T_S$ & 0.32 & 0.63 & 0.62 \\ \hline
Difference & +0.26 & +0.60 & +0.59 \\
 \hline
\end{tabular}

\label{tab:hypothesis_3_a_comparison}
\end{minipage}
\end{table}


\begin{table}[htb]
\centering
\begin{minipage}{0.95\columnwidth}
\centering
\caption{Hypothesis \autoref{hyp:relation_property_antisym} comparison of DE-DistMult with DE-TransE and DE-Simple on anti-symmetric relation testsets $T_A$ and non-antisymmeric relation testsets $T_A'$.}
\vspace{-3mm}

\begin{tabular}{r|ccc}\hline
ICEWS14 & DE-T & DE-D & DE-S \\ \hline
$T_A'$ & 0.23 & 0.35 & 0.36 \\
$T_A$ & 0.32 & 0.38 & 0.39 \\ \hline
Difference & +0.09 & +0.03 & +0.03 \\ \hline\hline
WIKIDATA & DE-T & DE-D & DE-S \\ \hline
$T_A'$ & 0.11 & 0.12 & 0.12 \\
$T_A$ & 0.14 & 0.14 & 0.14 \\ \hline
Difference & +0.03 & +0.02 & +0.03 \\ \hline\hline
YAGO & DE-T & DE-D & DE-S \\ \hline
$T_A'$ & 0.32 & 0.63 & 0.62 \\
$T_A$ & 0.06 & 0.03 & 0.04 \\ \hline
Difference & -0.26 & -0.60 & -0.58 \\ \hline
\end{tabular}

\label{tab:hypothesis_3_b_comparison}
\end{minipage}
\end{table}


\begin{table}[htb]
\centering
\begin{minipage}{\columnwidthcaption}
\centering
\caption{Hypothesis \autoref{hyp:relation_property_inv} comparison of DE-DistMult with DE-TransE and DE-Simple on inverse relation testsets $T_I$ and non-inverse relation testsets $T_I'$.}
\vspace{-3mm}

\begin{tabular}{r|ccc}\hline
\mbox{ICEWS14-7k} & DE-T & DE-D & DE-S \\ \hline
$T_I'$ & 0.24 & 0.35 & 0.36 \\
$T_I$ & 0.34 & 0.44 & 0.49 \\ \hline
Difference & +0.11 & +0.10 & +0.13 \\
 \hline
\end{tabular}

\label{tab:hypothesis_3_c_comparison}
\end{minipage}
\end{table}



To evaluate hypothesis \autoref{hyp:relation_properties} the methods have been evaluated on testsets divided into a number of different relation properties. The testsets of each dataset all contain a relation in the query, and predicts on head, tail or timestamp.

In \autoref{tab:relation_property_test_sets} the number of facts in each test set is detailed, as well as the number of types of relations.
For an overview of what each test set contains, see \autoref{tab:test_set_explanations}. There are no relations with the reflexive property in any dataset, and as such test set $T_R$ is empty for all datasets.
The models will only be compared on testsets where there are facts that have a given property and facts that do not have that given property. ICEWS14 is the dataset best suited for analysis of this hypothesis, as there is a higher number of relation types, with more varied relation properties.

All three sub-hypotheses of \autoref{hyp:relation_properties} all refer to specific methods, and what they can and cannot represent. They are all diacronic embedding methods, and they will be compared to the other diachronic embedding methods, as they share the most characteristics with eachother.

To evaluate on \autoref{hyp:relation_property_sym}, we have analyzed and compared the performance of DE-TransE, DE-DistMult, and DE-SimplE, as can be seen in \autoref{tab:hypothesis_3_a_comparison}.
On both dataset ICEWS14 and YAGO, the symmetrical relations seem to be significantly more simple for the embedding methods to represent -- They all achieve a higher performance on the symmetrical dataset than the non-symmetrical dataset. What can also be observed is that DE-TransE has a lower difference in improvement between the symmetrical testsets and the non-symmetrical testsets. In ICEWS14 the performance of DE-TransE is similar across the symmetrical testset as the non-symmetrical. The results are not as strong as we initally anticipated, but it still indicates that DE-TransE performs worse on symmetrical relations than other models do, compared to their performance on non-symmetrical relations, and this indicates that \autoref{hyp:relation_property_sym} is true.

For our analysis of \autoref{hyp:relation_property_antisym}, we have compared the performance of DE-DistMult with DE-TransE and DE-SimplE, as can be seen in \autoref{tab:hypothesis_3_b_comparison}.
As the table shows, there is little that indicates that DE-DistMult is worse at anti-symmetric relations than the other DE models, and the performance of DE-DistMult is almost the same as DE-SimplE. This indicates that DE-DistMult perfoms similarly on anti-symmetrical relations as non-antisymmetrical relations, which indicates that \autoref{hyp:relation_property_antisym} is false, despite the theoretical disadvantage that DE-DistMult has.

Finally, to evaluate \autoref{hyp:relation_property_inv}, we have compared the performance of DE-DistMult with the performance of DE-TransE and DE-SimplE, on an inverse relation testset and a non-inverse relation testset in \autoref{tab:hypothesis_3_c_comparison}. Only ICEWS14 has inverse relations.
The comparison shows that DE-DistMult has higher performance on the inverse testset than the non-inverse testset, but the difference between these two performance results are lower then on other models. The difference is not very significant however, and the performance is overall similar to the performance of DE-SimplE. These results are not sigificant enough to confirm the hypothesis, and therefore the results indicate that \autoref{hyp:relation_property_inv} is false, however this result is only based on a single dataset, and therefore is not very well supported.

Overall these results seem to indicate that \autoref{hyp:relation_properties} is true under some circumstances, which indicate that the real performance of each model can in some cases be predicted by the theoretical limitations of the methods, but the methods will also achieve good results despite it.

\section{Ensemble Model}
\label{sec:ensemble_model}
The purpose of the ensemble experiement is to find the impact of our findings in the hypothesis. This is done by creating two ensemble voting mathods. The first method is the naive ensemble voting where each method have the same weight when deciding the answer to a query. the second is the decision three ensemble voting, where based on our findings and what query that is to be answered, the methods have different weights, and therefore different levels of importance when answering the query. 

\subsection{naive voting ensemble}

\subsection{decision tree voting ensemble}
\section{Evaluation}
\label{sec:evaluation}

In this section, some considerations made during the project is presented. Choices made and alternatives will be explained.

\subsection{Alternative Approaches to Relation Properties}
\label{sec:alt_approaches_to_relation_properties}
Relation properties are determined using a soft label approach. This section details alternative approaches and their advantages and disadvantages.

Ideally, we would extract the properties of relations directly from metadata about the source graphs, as this would accurately depict the properties of the relations. However, most \glspl{kg} do not contain this metadata, and none of the \glspl{kg} used in this paper has complete metadata information, making this approach unfeasible.

Another alternative approach is manual assignment of properties based on notions about what each relation models. This approach risks misinterpreting the purpose of the relations, resulting in property assignment that do not reflect the data, and involves additional work to include each dataset.

We chose to do soft label assignment, as it depicts the relevant data, does not require a complete \gls{kg} unlike hard label assignment, and makes it easy to add new datasets.

Based on the labels assigned to each relation, we considered using data materialization, to create more complete \glspl{kg}. This would include modifying the dataset with new facts that can be inferred from the relation properties e.g. adding $(e_1, r, e_2, \tau)$ if $(e_2, r, e_1, \tau)$ exists and $r$ has the symmetric property.
We chose not to do this, as we still could not guarantee a complete \gls{kg} after data materialization, and it would make it impossible to compare our results to other works that was learned on the original dataset.

The soft label thresholds were selected empirically. We identified some relations that we expect to have certain properties, and the thresholds were selected such that those relations were assigned the expected relation properties.
\subsection{Errors in WikiData Dataset}
During this project we discovered that the WikiData dataset contains multiple errors.

Timestamps in WikiData are formatted as yyyy-mm-dd 
and uses \# in place of numbers where values are unknown.
However, there are 54 instances where the timestamps contain only one or two characters in the year value. The data appears to be correct in most of these cases, but we have identified at least two instances of factual errors in these timestamps.

% E.g. line 12849 of the train set is:

% \begin{lstlisting}
% 5725, 17, 6634, 19-##-##, 19-##-##
% \end{lstlisting}

% which evaluates to:

% \begin{lstlisting}
% Wendy Hiller, award received,
% Dame Commander of the Order of the British
% Empire, year 19, year 19
% \end{lstlisting}

% However, the timestamp should be 1975 not 19 \cite{WendyHiller}.

As the scale of the errors appears to be negligible and because identifying and correcting the errors would be both time consuming and prevent accurate comparison between the results of this and other papers, we elected to make no changes.


% Disadvantages of the rule-based learner (fx is it extensible? is it accurate?)




%Måske skriv omkring overvejelser omkring "significant results"

% Skriv om at DE metoderne ikke klarer time_to

% tero og atise bruger time spans så mrr er skewed til deres fordel for wikidata og yago fordi de ikke kan gøre det lige så forkert som de andre metoder. de har dog ikke fordelen på icews og de er stadig lidt bedre

% de methods virker til at være totalt random i deres predictions og bedre score på yago kan være pga mere spredt dataset. overvej det blank timestamp som de træner på




\section{Conclusion}
\label{sec:conclusion}

We have analyzed the \gls{tkg} embedding methods DE-TransE, De-DistMult, DE-SimplE, ATiSE, TeRo and TimePlex to evaluate their performance depending on the charactersitics of queries.
%hypoteser
The inspected characteristics are the temporal density of data, the relation properties symmetry, antisymmetry, and inversion, and prediction target which has been included and expanded upon from a previous set of findings.

%hypoteser
We find that the overall score of predictions is not particularly impacted by the temporal density of the data, however time predictions specifically score higher in temporally dense partitions.
Our findings also support that DE-TransE, which theoretically cannot model symmetry, is indeed worse at symmetric relations than other models, while DE-DistMult, which theoretically cannot model antisymmetry and inversion, is not worse at antisymmetric or inverse relations than other models.

%ensemble
The findings are used in a new ensemble model ORB-E where the results are used to calculate query-specific weights for each model. ORB-E achieves better performance than each individual model, as well as an ensemble model that assignes the same static weight to each model.
This suggests that it is possible to achieve a better performance by taking a model's advantages and disadvantages into account.

%time accuracy
Finally, we use the continuous nature of timestamps to examine the accuracy of time predictions and improve them.
We find that the difference between the predicted and correct timestamp is generally too large to be useful in most \gls{qa} systems.
However, if all models are equally accurate in their top timestamp answer, the average of all predictions is more accurate than each indvidual model.
\section{Future Work}
\label{sec:future-work}

%Metalearner
In this paper, an ensemble learning method based on a calculated decision tree has been presented. Another possible approach is an ensemble learning method that learns the weights to assign to each model using the properties of the prediction task, and the involved relation. Alternatively, a learned ensemble method could learn the weights directly from the results of each prediction, and not use the decision tree at all. If these learned ensemble methods perform better than the naive ensemble voting and the decision tree ensemble voting methods, it could be possible to inspect the learned weights to find which entities, relations, and timestamp each model is most well suited for.

%Error difference based predictions
As illustrated in the results of this paper, the \gls{mae} of time predictions are higher than what is acceptable in a question answering context. Having an average error of 500 years in some datasets means that the answer to some asked question is not very useful. To mitigate this, it might be useful to create a method that optimize this facet of time predictions, and learn an embedding that allows time predictions to be within acceptable margins, by using the error difference as a part of the scoring function. This could result in a method that can more accurately answer temporal questions in a question answering system.

%temporal properties of relations IE. person a kills person b. person b can't die again.
More detailed and accurate data could enable additional temporal signals in the data, as temporal events can permanently change an entity, and therefore change the behavour that entity will have in all following relations. For example, once a person gets a master's degree, that person will never go back to not having a master's degree. Presumably, persons that have a higher education act different than those that does not have a higher education. Different models might be more suited to model people with a higher education than those without. Additionally, it could be possible to create a method that accounts for the state of the entity at a given point in time, depending on which relations that entity has been a part of earlier.



\begin{acks}
    We would like to thank Daniele Dell'Aglio and Huan Li for help and guidance.
\end{acks}

% BIBLIOGRAPHY
\bibliographystyle{ACM-Reference-Format}
\bibliography{references}

% APPENDIX
%\onecolumn
\pagebreak
\appendix
\section{Glossary}
\printglossary[type=\acronymtype, title= ]
\section{Overview of Test Sets}
\label{app:test_sets_overview}

\begin{table}[htb]
\centering
\begin{minipage}{0.95\columnwidth}
\centering
\caption{Test sets, and what they contain}
\vspace{-3mm}

\begin{tabular}{r|l}\hline
Test set & Contains \\ \hline
$T_D$ & Dense partiton timestamps \\ 
$T_P$ & Sparse partition timestamps \\
$T_S$ & Symmetrical relations \\ 
$T_S'$ & Non-symmetrical relations \\ 
$T_A$ & Anti-symmetrical relations \\ 
$T_A'$ & Non-antisymmetrical relations \\ 
$T_I$ & Inverse relations \\ 
$T_I'$ & Non-inverse relations \\ \hline
%$T_R$ & Reflexive relations \\ 
%$T_R'$ & Non-reflexive relations \\
\end{tabular}

\label{tab:test_set_explanations}
\end{minipage}
\end{table}



\section{Test Set Statistics}
\label{app:test_set_statistics}

\begin{table}[htb]
\centering
\begin{minipage}{\columnwidthcaption}
\centering
\caption{Number of facts in each test set}
\label{tab:test_set_stats}
\end{minipage}
\vspace{-3mm}

%\resizebox{\columnwidth}{!}{
\begin{tabular}{r|c|cc|cc|cc|cc}\hline
Dataset & $|\varR|$ & $|T_D|$ & $|T_P|$  & $|T_S'|$ & $|T_S|$ & $|T_A'|$ & $|T_A|$ & $|T_I'|$ & $|T_I|$
%& $|T_R'|$ & $|T_R|$ 
\\ \hline
ICEWS & 220 & 7422 & 6687 & 22902 & 3987 & 23253 & 3636 & 25581 & 1308
%& 26889 & 0 
\\
WikiData & 24 & 4248 & 3924 & 16248 & 0 & 580 & 15668 & 16248 & 0 
%& 16248 & 0 
\\
YAGO & 10 & 2068 & 2000 & 7500 & 704 & 704 & 7500 & 8204 & 0 
%& 8204 & 0
\\
 \hline
\end{tabular}
%}

\end{table}


%\section{KG Selection Framework}
\label{app:kg_selection_framework}
\begingroup
\setlength\tabcolsep{0pt}
\titleformat{\subsubsection}[block]{\large\bfseries}{\thesubsubsection}{3mm}{}

This section contains a detailed description of use of the \gls{kg} selection framework introduced in \cite{farber2017dataquality}.
The analyzed \glspl{kg} are DBPEDIA, FreeBase, OpenCyc, WikiData \cite{vrandecic2014wikidata}, and YAGO \cite{mahdisoltani2015YAGO3} which were included in the original framework.
%The \glspl{kg} WordNet, ICEWS, and GDELT have been added as they are prevalent in the related work.
The result of the analysis is available in \autoref{tab:selection_framework}.

\begin{table*}[ht]
\centering
\begin{minipage}{0.95\textwidth}
\centering
\small
\includegraphics[scale=0.5]{content/appendix/figures/image.png}
\caption{Values for DBPEDIA, FreeBase, OpenCyc, WikiData and YAGO are taken from \cite{farber2017dataquality} as a rule. Exceptions are marked with '*' and explained in \autoref{app:kg_selection_framework}.
%Values for WordNet, ICEWS and GDELT are added \missing
}
\renewcommand{\arraystretch}{1.2}

\begin{comment}
\begin{tabular}{|m{2.5cm}|M|c|c|c|c|M|c|}
\hline
\thead{Method} & %method
\thead{Category} & %category
\thead{Temporal\\Information} & %temporal
\thead{Symmetric\\Relations} & %symmetry
\thead{Anti-\\Symmetric\\Relations} & %anti-symmetry
\thead{Entity\\Embedding} & %entity embedding
%\thead{Relation Embedding} & %relation embedding
\thead{Time\\Dimension} & %time dimension
\thead{Space} %space
\\
\hline
%\rowcolor{blue!10}
TransE \newline\cite{transe} & %method
Transformation \newline (translation) & %category
 & %temporal
 & %symmetry
\checkmark & %anti-symmetry
Separate & %entity embedding
%Single Vector & %relation embedding
\hrulefill & %time dimension
$\R$%space
\\
\hline
\end{tabular}
\end{comment}

\label{tab:selection_framework}
\end{minipage}
\end{table*}

Weights are assigned based on the perceived importance of a criterion relative to the project. They are determined as follows:
\begin{enumerate}
    \setcounter{enumi}{-1}
    \item Irrelevant
    \item Interesting, but most likely irrelevant
    \item Important
    \item Central
\end{enumerate}

In the following is given: a short description of each criterion; the assigned weight; reasoning behind the weight; notes where values have been changed from the original framework.

\subsection{Intrinsic Category}
The intrinsic category describes the data quality independent of the context.

\subsubsection{Accuracy}
The accuracy dimension describes whether the data is without errors.


\weighttable
{m_{synRDF(g)}}
{The syntactic validity of RDF documents.}
{3}
{Syntactic validity is necessary for machine interpretation and machine interpretation is necessary to train a model. As it is unrealistic to fix errors manually it is central that the documents are syntactically valid.}
{}

\weighttable
{m_{synLit}(g)}
{The syntactic validity of literals dependent on data types or pre-defined regular expressions.}
{2}
{The syntactic validity of literals is important as inconsistencies presumably affects the models negatively and we aim to analyze the models at their optimal performance. However it is possible to train the models and obtain a result even with syntactic invalidity in literals and as such it is not as important as $m_synRDF(g)$, the syntactic validity of documents.}
{The value for $m_{synLit}(YAGO)$ was changed from 0.62 to 0.99. The low value was due to YAGO allowing wildcards in the date datatype. As this is a feature in our context, the value was dismissed and an estimate was made based on the remaining factors.}

\weighttable
{m_{semTriple}(g)}
{The degree to which facts hold true.}
{0}
{The purpose of an embedding model is to represent available data and the performance of the model is evaluated against that same data. As such it is irrelevant whether the data holds true or not outside of the \gls{kg}.}
{}

\subsubsection{Trustworthiness}
The Trustworthiness dimension describes whether the data is credible.

\weighttable
{m_{graph}(h_g)}
{Trustworthiness on \gls{kg} level measured by manner of insertion and curation.}
{1}
{This criterion prioritizes manual over automatic work and experts over community. 
While we believe that this approach values the more reliable data higher and we acknowledge that the quality of training data highly impacts the quality of the resulting models, it is also noted that automatic means tend to yield bigger datasets, which also highly impacts the quality of the resulting models. Based on the assumption that significant problems will be revealed through other criteria we consider this criterion interesting, but likely irrelevant.}
{}

\weighttable
{m_{fact}(g)}
{Triples or documents are connected to sources}
{1}
{It lends credibility to the graph and is likely useful in a production setting, but while it is interesting, it is probably irrelevant as it is not the focus of the project and thus likely won't be of use.}
{}

\weighttable
{m_{NoVal}(g)}
{Indication of unknown and empty values}
{1}
{Could be used for negative sampling, as samples that are close to the truth are more valuable than those that are not \cite{yang2020negative-sampling, hyunh2019fact-checking-benchmark}. While this could help identification of those samples and negative sampling could improve model performance, it is not the focus of the project, probably won't be used and is thus interesting, but likely irrelevant.}
{The value for $m_{NoVal}(YAGO)$ was changed from 0 to 0.5 as YAGO does have some capacity for storing this information, through relations and use of wildcards to represent uncertainty.}

\subsubsection{Consistency}
The consistency dimension describes whether the data conflicts with itself.

\weighttable
{m_{checkRestr}(h_g)}
{Whether schema restrictions are checked at insertion time.}
{2}
{In general it is important that the data is consistent in order to enable models to properly represent the information.}
{}

\weighttable
{m_{conClass}(g)}
{Whether class restrictions are upheld.}
{2}
{It is important that class restrictions are upheld in order to eliminate possible sources of error, but as no hypothesis directly concerns class restrictions it is not central.}
{The values for $m_{conClass}(FreeBase)$ and $m_{conClass}(WikiData)$ are dismissed as the inspected constraints are not available in those graphs.}

\weighttable
{m_{conRelat}(g)}
{Whether relation restrictions are upheld.}
{3}
{As hypothesis \missing concerns model ability to represent various relations it is central that the relation constraints are upheld in the data.}
{The value for $m_{conClass}(WikiData)$ is dismissed as the inspected constraint is not available in that graph.}

\subsection{Contextual}
The contextual category describes the data quality in relation to the context.

\subsubsection{Relevancy}
The relevancy dimension describes whether the data is relevant to the purpose.

\weighttable
{m_{Ranking}(g)}
{Ranking of fact importance relative to one another}
{1}
{Could be used to extract more meaningful positive samples or when training models with an attention mechanism. However neither of those are the focus of this project and the information is thus interesting, but likely irrelevant.}
{The source is inconsistent in reports of $m_{Ranking}(FreeBase)$ and reports both 0 and 1. The 0 value is selected as it is consistent with the textual analysis and the values reported for remaining \glspl{kg}.}

\subsubsection{Completeness}
The completeness dimension describes whether the data is missing information relevant to the purpose.

\weighttable
{m_{cSchema}(g)}
{Quantifies missing classes and relations}
{3}
{As certain classes and relation are the subject of several analyses it is central that those classes and relations are present in the data. This is elaborated in \missing.}
{}

\weighttable
{m_{cCol}(g)}
{Quantifies missing relations on instances of classes}
{0}
{It is both an accepted reality and the point of the link prediction task that some information will be missing. While the total number of instances per relation is important, the degree to which entities are annotated with all entities is irrelevant.}
{}

\weighttable
{m_{cPop}(g)}
{Quantifies missing instances}
{0}
{This is irrelevant as the analysis concerns models' ability to embed data with certain characteristics rather than exact data.}
{}

\subsubsection{Timelessness}
The timelessness dimension describes whether the age af data is suitable for the purpose.

\weighttable
{m_{Freq}(g)}
{Frequency of updates}
{1}
{Data should be current to the extent that it is represented in current research to enable comparison of results and ease implementation efforts.}
{}

\weighttable
{m_{Validity}(g)}
{Specification of temporal validity}
{3}
{The project is focused on temporal embedding methods and as such the availability of temporal information is central to the analysis.}
{}

\weighttable
{m_{Change}(g)}
{Specification of last modification}
{0}
{In order to eliminate possible sources of error models are trained on static datasets and as such the modification date is irrelevant.}
{}

\subsection{Representational}
The representational category describes the format of data.

\subsubsection{Ease of Understanding}
The ease of understanding dimension describes whether the data is without ambiguity for a human.

\weighttable
{m_{Descr}(g)}
{Entities are labelled with descriptions.}
{0}
{The models do not need this information to train and the information is not valuable in the context of any hypotheses. As such it is irrelevant.}
{}

\weighttable
{m_{Lang}(g)}
{Contains labels in multiple languages}
{0}
{The models do not need this information to train and the information is not valuable in the context of any hypotheses. As such it is irrelevant.}
{}

\weighttable
{m_{uSer}(h_g)}
{More human-readable serialization formats are available.}
{0}
{The \gls{kg} used as input is not examined beyond its fulfillment of data quality requirements which are analyzed here. As such it is irrelevant whether it is human-readable or not.}
{}

\weighttable
{m_{uURI}(g)}
{Short, self-describing, non-generic URIs.}
{0}
{Though the value of immediately interpretable URIs to the human reader is obvious the value of clearly distinguishable URIs should not be dismissed and as the interpretation is available through labels the short URIs are irrelevant.}
{}

\subsubsection{Interoperability}
The interoperability dimension describes whether the data is without ambiguity for a computer.

\weighttable
{m_{Reif}(g)}
{Avoidance of blank nodes and reification statements.}
{0}
{The usage of blank nodes and reification statements could both improve and impair model performance. As it is not immediately obvious what impact their use has on model performance it is not possible to select the option allowing better model performance and as no hypothesis concerns this aspect whether they are used or not is irrelevant.}
{The source is inconsistent in its reports of $m_{Reif}(DBPEDIA)$ and reports both values of 0.5 and 1. The 0.5 value is selected as it is consistent with the textual analysis and the values reported for remaining \glspl{kg}.}

\weighttable
{m_{iSerial}(g)}
{The availability of other serialization formats.}
{1}
{Embedding methods typically require the input data in the format of tuples. While it is convenient to have the data supplied in the right format converting it is simple as long as it is syntactically valid and as such the availability of other formats is interesting, but likely irrelevant.}
{}

\weighttable
{m_{extVoc}(g)}
{Usage of external vocabulary for relations.}
{1}
{Multiple \glspl{kg} are analyzed in order to generalize observations. If all sources used the same vocabulary it would allow more fine-grained analysis and might simplify parts of the analysis, but the focus of our analysis is generally of a higher abstraction level and thus external vocabulary is interesting, but likely irrelevant.}
{The source is inconsistent in its reports of $m_{extVoc}(OpenCyc)$ and reports both values of 0.415 and 0.41. As this appears to be a rounding error the value 0.415 is used.}

\weighttable
{m_{propVoc}(g)}
{Linking the vocabulary to that of external sources.}
{0}
{In principle the reasoning is very similar to that of $m_{extVoc}(g)$. However this adds an additional layer of complexity to an already unlikely task, making the information irrelevant.}
{}

\subsection{Accessibility}
The accessibility category describes the accessibility of the data.

\subsubsection{Accessibility}
The accessibility dimension describes whether the data is available as well as the ease and speed of retrieval.

\weighttable
{m_{Deref}(h_g)}
{URIs are resolvable via HTTP requests.}
{2}
{It is important to have access to additional information about the resources in order to discover patterns.}
{The source has switched the values of $m_{Deref}(h_{FreeBase})$ and $m_{Deref}(h_{OpenCyc})$ in at least one instance. $m_{Deref}(h_{FreeBase})$ is set to 0.437 and $m_{Deref}(h_{OpenCyc})$ is set to 1 as those are the most commonly reported numbers and consistent with the textual analysis.}

\weighttable
{m_{Avai}(h_g)}
{Uptime of the \gls{kg}.}
{0}
{As only static data is queried it is irrelevant.}
{The source has switched the values of $m_{Avai}(h_{FreeBase})$ and $m_{Avai}(h_{YAGO})$ in at least one instance. $m_{Avai}(h_{FreeBase})$ is set to 0.9998 and $m_{Avai}(h_{YAGO})$ is set to 0.7306 as those are the most commonly reported numbers and consistent with the textual analysis.}

\weighttable
{m_{SPARQL}(h_g)}
{The availability of a SPARQL endpoint.}
{1}
{In a production setting this would likely be necessary, but as that is outside the scope of this project it is interesting, but likely irrelevant.}
{The source has switched the values of $m_{SPARQL}(h_{FreeBase})$ and $m_{SPARQL}(h_{YAGO})$ in at least one instance. $m_{SPARQL}(h_{FreeBase})$ is set to 0 and $m_{SPAQRL}(h_{YAGO})$ is set to 1 as those are the most commonly reported numbers and consistent with the textual analysis.}

\weighttable
{m_{Export}(h_g)}
{The ability to export the data in RDF format.}
{3}
{In order to eliminate possible sources of error models are trained on static datasets which are exported from the \gls{kg} and the ability to so is therefore central.}
{}

\weighttable
{m_{Negot}(h_g)}
{Support of content negotiation that is dereferencing sources in client's preferred serialization format.}
{0}
{As long as content is returned in a valid serialization format the exact format is largely irrelevant.}
{The source has switched the values of $m_{Negot}(h_{FreeBase})$ and $m_{Negot}(h_{YAGO})$ in at least one instance. $m_{Negot}(h_{FreeBase})$ is set to 0 and $m_{Negot}(h_{YAGO})$ is set to 1 as those are the most commonly reported numbers and consistent with the textual analysis.}

\weighttable
{m_{HTMLRDF}(h_g)}
{Linking HTML descriptions and RDF serializations to facilitate automatic link discovery.}
{0}
{While possibly useful to improve the performance of embedding models, we are unaware of any embedding methods utilizing it and it is therefore irrelevant.}
{The source has switched the values of $m_{HTMLRDF}(h_{OpenCyc})$ and $m_{HTMLRDF}(h_{YAGO})$ in at least one instance. $m_{HTMLRDF}(h_{OpenCyc})$ is set to 0 and $m_{HTMLRDF}(h_{YAGO})$ is set to 1 as those are the most commonly reported numbers and consistent with the textual analysis.}

\weighttable
{m_{Meta}(g)}
{Machine-readable metadata via sitemaps and VoID.}
{1}
{Some embedding methods make use of metadata to improve the performance of the model. As this aspect is not covered in any hypothesis, the information is interesting, but likely irrelevant.}
{The source has switched the values of $m_{Meta}(OpenCyc)$ and $m_{Meta}(YAGO)$ in at least one instance. $m_{Meta}(OpenCyc)$ is set to 1 and $m_{Meta}(YAGO)$ is set to 0 as those are the most commonly reported numbers and consistent with the textual analysis.}

\subsubsection{License}
The license dimension describes the license the data is subject to.

\weighttable
{m_{macLicense}(g)}
{Availability of machine-readable license.}
{1}
{It is important that there is a license permitting our use of the data but as a human-readable version is sufficient it is interesting, but likely irrelevant whether a machine-readable one exists.}
{}

\subsubsection{Interlinking}
The interlinking dimension describes whether data representing the same entities is linked.

\weighttable
{m_{Inst}(g)}
{Equivalence links to external sources.}
{0}
{This would only be relevant if models were trained on combined graphs. As that is not the case it is irrelevant.}
{The source is inconsistent in its reports of $m_{Inst}(DBPEDIA)$ and $m_{Inst}(OpenCyc)$. $m_{Inst}(DBPEDIA)$ is reported as 0.592 and 0.251 and $m_{Inst}(OpenCyc)$ is reported as 0.443 and 0.382.
$m_{Inst}(DBPEDIA)$ is set to 0.251 and $m_{Inst}(OpenCyc)$ is set to 0.382 as those are the most commonly reported numbers and consistent with the textual analysis.}

\weighttable
{m_{URI}(g)}
{Validity of of external URIs.}
{0}
{This is irrelevant as the external URIs are not used.}
{The source is inconsistent in its reports of $m_{URI}(DBPEDIA)$ and $m_{URI}(FreeBase)$ and the textual analysis does not resolve these inconsistencies. $m_{URI}(DBPEDIA)$ is set to 0.929 and $m_{URI}(FreeBase)$ is set to 0.908 as those are the most common values. It is also noted that greatest value difference is 0.06.}
\endgroup
\clearpage
\section{Time Prediction Error Distribution}
\label{time_prediction_error_distribution}
\input{content/appendix/figures/time_error_distribution_smooth/de_transe_icews14_original}
\input{content/appendix/figures/time_error_distribution_smooth/de_distmult_icews14_original}
\input{content/appendix/figures/time_error_distribution_smooth/de_simple_icews14_original}
\input{content/appendix/figures/time_error_distribution_smooth/atise_icews14_original}
\input{content/appendix/figures/time_error_distribution_smooth/tero_icews14_original}
\input{content/appendix/figures/time_error_distribution_smooth/timeplex_icews14_original}

\clearpage
\input{content/appendix/figures/time_error_distribution_smooth/de_transe_wikidata12k_original}
\input{content/appendix/figures/time_error_distribution_smooth/de_distmult_wikidata12k_original}
\input{content/appendix/figures/time_error_distribution_smooth/de_simple_wikidata12k_original}
\input{content/appendix/figures/time_error_distribution_smooth/tero_wikidata12k_original}
\input{content/appendix/figures/time_error_distribution_smooth/atise_wikidata12k_original}
\input{content/appendix/figures/time_error_distribution_smooth/timeplex_wikidata12k_original}

\clearpage
\input{content/appendix/figures/time_error_distribution_smooth/de_transe_yago11k_original}
\input{content/appendix/figures/time_error_distribution_smooth/de_distmult_yago11k_original}
\input{content/appendix/figures/time_error_distribution_smooth/de_simple_yago11k_original}
\input{content/appendix/figures/time_error_distribution_smooth/tero_yago11k_original}
\input{content/appendix/figures/time_error_distribution_smooth/atise_yago11k_original}
\input{content/appendix/figures/time_error_distribution_smooth/timeplex_yago11k_original}
\clearpage

\section{Comparisons between datasets and splits}
\label{app:dataset_split_comparisons}

\input{content/appendix/figures/compare/icews14_original}
\input{content/appendix/figures/compare/icews14_1}
\input{content/appendix/figures/compare/icews14_2}
\input{content/appendix/figures/compare/icews14_3}

\input{content/appendix/figures/compare/wikidata12k_original}
\input{content/appendix/figures/compare/wikidata12k_1}
\input{content/appendix/figures/compare/wikidata12k_2}
\input{content/appendix/figures/compare/wikidata12k_3}

\input{content/appendix/figures/compare/yago11k_original}
\input{content/appendix/figures/compare/yago11k_1}
\input{content/appendix/figures/compare/yago11k_2}
\input{content/appendix/figures/compare/yago11k_3}

\end{document}
